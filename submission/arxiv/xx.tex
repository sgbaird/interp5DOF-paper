\begin{filecontents}{main.aux}
\relax
\providecommand\hyper@newdestlabel[2]{}
\providecommand\zref@newlabel[2]{}
\providecommand\HyperFirstAtBeginDocument{\AtBeginDocument}
\HyperFirstAtBeginDocument{\ifx\hyper@anchor\@undefined
\global\let\oldcontentsline\contentsline
\gdef\contentsline#1#2#3#4{\oldcontentsline{#1}{#2}{#3}}
\global\let\oldnewlabel\newlabel
\gdef\newlabel#1#2{\newlabelxx{#1}#2}
\gdef\newlabelxx#1#2#3#4#5#6{\oldnewlabel{#1}{{#2}{#3}}}
\AtEndDocument{\ifx\hyper@anchor\@undefined
\let\contentsline\oldcontentsline
\let\newlabel\oldnewlabel
\fi}
\fi}
\global\let\hyper@last\relax
\gdef\HyperFirstAtBeginDocument#1{#1}
\providecommand\HyField@AuxAddToFields[1]{}
\providecommand\HyField@AuxAddToCoFields[2]{}
\providecommand\babel@aux[2]{}
\@nameuse{bbl@beforestart}
\emailauthor{ster.g.baird@gmail.com}{Sterling G. Baird\corref {cor1}}
\Newlabel{cor1}{1}
\citation{jinColossalGrainGrowth2018}
\citation{brandenburgMigrationFacetingLowangle2014}
\citation{huangGrainRotationLattice2015,trauttCapillarydrivenGrainBoundary2014,sharmaObservationChangingCrystal2012,wareGrainBoundaryPlane2018}
\citation{liAnisotropyHydrogenDiffusion2017,oudrissGrainSizeGrainboundary2012}
\citation{metsueHydrogenSolubilityVacancy2016}
\citation{huangHydrogenEmbrittlementGrain2017}
\citation{xiaApplingGrainBoundary2011,demkowiczThresholdDensityHelium2020,hansonCrystallographicCharacterGrain2018,jothiInvestigationMicromechanismsHydrogen2016,zhouChemomechanicalOriginHydrogen2016}
\citation{huangNaturalImpactresistantBicontinuous2020,wangAdditivelyManufacturedHierarchical2018,linMeasuringNonlinearStresses2016}
\citation{yinCeramicPhasesOnedimensional2019,guanAnalysisThreedimensionalMicrostructure2011}
\citation{vlassioukEvolutionarySelectionGrowth2018,hanSubnanometreChannelsEmbedded2018}
\citation{sunEnhancingPowerFactor2020}
\citation{hanSubnanometreChannelsEmbedded2018,huangNaturalImpactresistantBicontinuous2020}
\citation{johnsonInferringGrainBoundary2015,yangMeasuringRelativeGrain2001,zhangGrainBoundaryMobilities2020}
\citation{hanGrainboundaryMetastabilityIts2016,weiDirectImagingAtomistic2021}
\citation{bostanabadStochasticMicrostructureCharacterization2016,Homer2019c,Jothi2015h,pirgaziAlignment3DEBSD2019,pirgaziThreedimensionalCharacterizationGrain2015,speidelCrystallographicTextureCan2018,zhangGrainBoundaryMobilities2020,zhengGrainBoundaryProperties2020}
\citation{keinanIntegratedImagingThree2018,Seita2016,speidelCrystallographicTextureCan2018,winiarskiBroadIonBeam2017,zhangGrainBoundaryMobilities2020}
\citation{kimIdentificationSchemeGrain2011,liAtomisticSimulationsEnergies2019,liRelativeGrainBoundary2009,olmstedSurveyComputedGrain2009,olmstedSurveyComputedGrain2009a,pirgaziThreedimensionalCharacterizationGrain2015,randleFiveparameterGrainBoundary2008,saylorMisorientationDependenceGrain2000,saylorRelativeFreeEnergies2003,yangAtomisticSimulationsEnergies2019,zhengGrainBoundaryProperties2020}
\Newlabel{myu}{a}
\@LN@col{1}
\providecommand\@newglossary[4]{}
\@newglossary{main}{glg}{gls}{glo}
\providecommand\@glsxtr@savepreloctag[2]{}
\providecommand\@glsorder[1]{}
\providecommand\@istfilename[1]{}
\@istfilename{main.ist}
\@glsorder{word}
\babel@aux{english}{}
\@writefile{toc}{\contentsline {section}{\numberline {1}Introduction}{1}{section.1}\protected@file@percent }
\newlabel{sec:intro}{{1}{1}{Introduction}{section.1}{}}
\newlabel{sec:intro@cref}{{[section][1][]1}{[1][1][]1}}
\@writefile{toc}{\contentsline {subsection}{\numberline {1.1}Motivation}{1}{subsection.1.1}\protected@file@percent }
\newlabel{sec:intro:motivation}{{1.1}{1}{Motivation}{subsection.1.1}{}}
\newlabel{sec:intro:motivation@cref}{{[subsection][1][1]1.1}{[1][1][]1}}
\@LN@col{2}
\citation{liRelativeGrainBoundary2009}
\citation{dillonCharacterizationGrainboundaryCharacter2009}
\citation{randleFiveparameterGrainBoundary2008}
\citation{restrepoUsingArtificialNeural2014}
\citation{kimIdentificationSchemeGrain2011}
\citation{guziewskiMicroscopicMacroscopicCharacterization2021,huGeneticAlgorithmguidedDeep2020}
\citation{francisGeodesicOctonionMetric2019}
\citation{chesserLearningGrainBoundary2020}
\citation{francisGeodesicOctonionMetric2019}
\citation{francisGeodesicOctonionMetric2019}
\citation{chesserLearningGrainBoundary2020}
\citation{olmstedSurveyComputedGrain2009a}
\citation{chesserLearningGrainBoundary2020}
\citation{morawiecDistancesGrainInterfaces2019}
\citation{morawiecDistancesGrainInterfaces2019}
\citation{francisGeodesicOctonionMetric2019}
\citation{barberQuickhullAlgorithmConvex1996}
\citation{francisGeodesicOctonionMetric2019,chesserLearningGrainBoundary2020}
\citation{francisGeodesicOctonionMetric2019}
\@LN@col{1}
\@writefile{toc}{\contentsline {subsection}{\numberline {1.2}Prior Work}{2}{subsection.1.2}\protected@file@percent }
\newlabel{sec:intro:prior}{{1.2}{2}{Prior Work}{subsection.1.2}{}}
\newlabel{sec:intro:prior@cref}{{[subsection][2][1]1.2}{[1][2][]2}}
\@LN@col{2}
\@writefile{toc}{\contentsline {subsection}{\numberline {1.3}\@glsentrytitlecase {vfzo}{long} Framework}{2}{subsection.1.3}\protected@file@percent }
\newlabel{sec:intro:vfzo}{{1.3}{2}{\glsentrytitlecase {vfzo}{long} Framework}{subsection.1.3}{}}
\newlabel{sec:intro:vfzo@cref}{{[subsection][3][1]1.3}{[1][2][]2}}
\citation{heinzRepresentationOrientationDisorientation1991}
\citation{grimmerUniqueDescriptionRelative1980,heinzRepresentationOrientationDisorientation1991}
\@LN@col{1}
\@writefile{toc}{\contentsline {section}{\numberline {2}Methods}{3}{section.2}\protected@file@percent }
\newlabel{sec:methods}{{2}{3}{Methods}{section.2}{}}
\newlabel{sec:methods@cref}{{[section][2][]2}{[1][3][]3}}
\@writefile{toc}{\contentsline {subsection}{\numberline {2.1}The \@glsentrytitlecase {vfzo}{long} Framework}{3}{subsection.2.1}\protected@file@percent }
\newlabel{sec:methods:framework}{{2.1}{3}{The \glsentrytitlecase {vfzo}{long} Framework}{subsection.2.1}{}}
\newlabel{sec:methods:framework@cref}{{[subsection][1][2]2.1}{[1][3][]3}}
\@LN@col{2}
\@writefile{toc}{\contentsline {subsubsection}{\numberline {2.1.1}Defining the \@glsentrytitlecase {vfz}{long}}{3}{subsubsection.2.1.1}\protected@file@percent }
\newlabel{sec:methods:framework:vfz}{{2.1.1}{3}{Defining the \glsentrytitlecase {vfz}{long}}{subsubsection.2.1.1}{}}
\newlabel{sec:methods:framework:vfz@cref}{{[subsubsection][1][2,1]2.1.1}{[1][3][]3}}
\citation{luongVoronoiSphere2020}
\citation{luongVoronoiSphere2020}
\citation{francisGeodesicOctonionMetric2019}
\citation{morawiecDistancesGrainInterfaces2019}
\citation{bairdFiveDegreeofFreedom5DOF2020}
\citation{francisGeodesicOctonionMetric2019}
\@LN@col{1}
\newlabel{eq:8Deuclidean_dist}{{1}{4}{Defining the \glsentrytitlecase {vfz}{long}}{equation.2.1}{}}
\newlabel{eq:8Deuclidean_dist@cref}{{[equation][1][]1}{[1][4][]4}}
\newlabel{eq:7sphere_arc_length}{{2}{4}{Defining the \glsentrytitlecase {vfz}{long}}{equation.2.2}{}}
\newlabel{eq:7sphere_arc_length@cref}{{[equation][2][]2}{[1][4][]4}}
\newlabel{eq:omega}{{3}{4}{Defining the \glsentrytitlecase {vfz}{long}}{equation.2.3}{}}
\newlabel{eq:omega@cref}{{[equation][3][]3}{[1][4][]4}}
\@LN@col{2}
\citation{francisGeodesicOctonionMetric2019}
\citation{olmstedSurveyComputedGrain2009a}
\citation{chesserLearningGrainBoundary2020}
\citation{olmstedSurveyComputedGrain2009a}
\citation{chesserLearningGrainBoundary2020}
\citation{patalaSymmetriesRepresentationGrain2013,homerGrainBoundaryPlane2015}
\citation{olmstedSurveyComputedGrain2009}
\@writefile{lof}{\contentsline {figure}{\numberline {1}{\ignorespaces (a) 3D Cartesian analogue to a non-degenerate 7D Cartesian representation of U(1)-symmetrized \cglspl {gbo} and \cglspl {vfzo} (\cglspl {vfzo} are inherently U(1)-symmetrized) which demonstrates the symmetrization of many points relative to a fixed reference point (white circle{}). This produces a 3D Cartesian \cgls {vfz} point set (dark blue points{}). (b) To further illustrate, a single input point (magenta points{}) is symmetrized (dark blue point{}) relative to a fixed reference point (white circle{}), demonstrating that only one symmetrized point is found within the borders (black{}) of each of the Voronoi cells (light blue{}). The Voronoi tessellation is defined by the symmetric images of the reference point, and the spherical Voronoi diagram for this illustration is constructed using a modified version of \cite  {luongVoronoiSphere2020}.\relax }}{5}{figure.caption.1}\protected@file@percent }
\providecommand*\caption@xref[2]{\@setref\relax\@undefined{#1}}
\newlabel{fig:voronoi}{{1}{5}{(a) 3D Cartesian analogue to a non-degenerate 7D Cartesian representation of U(1)-symmetrized \glspl {gbo} and \glspl {vfzo} (\glspl {vfzo} are inherently U(1)-symmetrized) which demonstrates the symmetrization of many points relative to a fixed reference point (\refpt {}). This produces a 3D Cartesian \gls {vfz} point set (\sympt {}). (b) To further illustrate, a single input point (\singlept {}) is symmetrized (\singlesympt {}) relative to a fixed reference point (\refpt {}), demonstrating that only one symmetrized point is found within the borders (\vbordercolor {}) of each of the Voronoi cells (\vcellcolor {}). The Voronoi tessellation is defined by the symmetric images of the reference point, and the spherical Voronoi diagram for this illustration is constructed using a modified version of \cite {luongVoronoiSphere2020}.\relax }{figure.caption.1}{}}
\newlabel{fig:voronoi@cref}{{[figure][1][]1}{[1][4][]5}}
\@LN@col{1}
\@writefile{toc}{\contentsline {subsubsection}{\numberline {2.1.2}Mapping \glsxtrtitleshortpl {gb} to the \@glsentrytitlecase {vfz}{long}}{5}{subsubsection.2.1.2}\protected@file@percent }
\newlabel{sec:methods:framework:proj}{{2.1.2}{5}{Mapping \glsfmtshortpl {gb} to the \glsentrytitlecase {vfz}{long}}{subsubsection.2.1.2}{}}
\newlabel{sec:methods:framework:proj@cref}{{[subsubsection][2][2,1]2.1.2}{[1][5][]5}}
\@writefile{toc}{\contentsline {subsubsection}{\numberline {2.1.3}Distance Calculations in the \@glsentrytitlecase {vfz}{long}}{5}{subsubsection.2.1.3}\protected@file@percent }
\newlabel{sec:methods:framework:vfz-dist}{{2.1.3}{5}{Distance Calculations in the \glsentrytitlecase {vfz}{long}}{subsubsection.2.1.3}{}}
\newlabel{sec:methods:framework:vfz-dist@cref}{{[subsubsection][3][2,1]2.1.3}{[1][5][]5}}
\@LN@col{2}
\citation{chesserLearningGrainBoundary2020}
\citation{morawiecDistancesGrainInterfaces2019}
\citation{olmstedSurveyComputedGrain2009}
\citation{chesserLearningGrainBoundary2020}
\citation{olmstedSurveyComputedGrain2009}
\citation{chesserLearningGrainBoundary2020}
\@writefile{lof}{\contentsline {figure}{\numberline {2}{\ignorespaces (a) Histogram of \cgls {nn} octonion distances ($\omega $) in a \cgls {vfzo} set of \num {50000} points. The average \cgls {nn} distance was \SI {2.8709 \pm 00.69112}{\degree }. (b) The average k-th nearest neighbor distances demonstrate that many nearest neighbors fall within a tight tolerance (less then \SI {10}{\degree }) out of approximately 10 trial runs.\relax }}{6}{figure.caption.2}\protected@file@percent }
\newlabel{fig:nnhist-knn-50000}{{2}{6}{(a) Histogram of \gls {nn} octonion distances ($\omega $) in a \gls {vfzo} set of \num {50000} points. The average \gls {nn} distance was \SI {\nnomega }{\degree }. (b) The average k-th nearest neighbor distances demonstrate that many nearest neighbors fall within a tight tolerance (less then \SI {10}{\degree }) out of approximately 10 trial runs.\relax }{figure.caption.2}{}}
\newlabel{fig:nnhist-knn-50000@cref}{{[figure][2][]2}{[1][5][]6}}
\@LN@col{1}
\@LN@col{2}
\@writefile{lof}{\contentsline {figure}{\numberline {3}{\ignorespaces Hexagonally binned parity plots of pairwise distances of 388 Ni bicrystals \cite  {olmstedSurveyComputedGrain2009a}. Euclidean distance approximation is converted to octonions ($x_{i,j,k}=2\left (\frac  {180}{\pi }\right )|\hat  {o}_{i,k}^{\text  {sym}}-\hat  {o}_{j,k}^{\text  {sym}}|$) for comparison with the traditional octonion metric \cite  {chesserLearningGrainBoundary2020}. The minimum distance among an ensemble of \cgls {vfzo} sets ($\qopname  \relax m{min}_{\forall k \in [1,k_{max}]}x_{i,j,k}$) is used for (a) 1, (b) 2, (c) 10, and (d) 20 \cgls {vfzo} sets. As the number of \cgls {vfzo} sets increases, the correlation between the Euclidean distance and the traditional octonion distance improves.\relax }}{7}{figure.caption.3}\protected@file@percent }
\newlabel{fig:dist-ensemble-k1-2-10-20}{{3}{7}{Hexagonally binned parity plots of pairwise distances of 388 Ni bicrystals \cite {olmstedSurveyComputedGrain2009a}. Euclidean distance approximation is converted to octonions ($x_{i,j,k}=2\left (\frac {180}{\pi }\right )|\hat {o}_{i,k}^{\text {sym}}-\hat {o}_{j,k}^{\text {sym}}|$) for comparison with the traditional octonion metric \cite {chesserLearningGrainBoundary2020}. The minimum distance among an ensemble of \gls {vfzo} sets ($\min _{\forall k \in [1,k_{max}]}x_{i,j,k}$) is used for (a) 1, (b) 2, (c) 10, and (d) 20 \gls {vfzo} sets. As the number of \gls {vfzo} sets increases, the correlation between the Euclidean distance and the traditional octonion distance improves.\relax }{figure.caption.3}{}}
\newlabel{fig:dist-ensemble-k1-2-10-20@cref}{{[figure][3][]3}{[1][5][]7}}
\@LN@col{1}
\@LN@col{2}
\citation{francisGeodesicOctonionMetric2019}
\citation{francisGeodesicOctonionMetric2019}
\citation{morawiecDistancesGrainInterfaces2019}
\citation{francisGeodesicOctonionMetric2019}
\citation{singhOrientationSamplingDictionarybased2016}
\@LN@col{1}
\@writefile{lof}{\contentsline {figure}{\numberline {4}{\ignorespaces \cGls {rmse} and \cgls {mae} of pairwise distance errors for 388 Ni bicrystals \cite  {olmstedSurveyComputedGrain2009} of scaled Euclidean distance approximation relative to the traditional octonion metric \cite  {chesserLearningGrainBoundary2020} (compare with \cref  {fig:dist-ensemble-k1-2-10-20}). The minimum distance among an ensemble of \cgls {vfzo} sets ($\qopname  \relax m{min}_{\forall k \in [1,k_{max}]}x_{i,j,k}$, where $x_{i,j,k}$ is the scaled Euclidean distance) is taken, iteratively adding consecutive sets up to $k_{max} = 20$. As the number of \cgls {vfzo} sets increases, \cgls {rmse} and \cgls {mae} between the scaled Euclidean distance approximation and the traditional octonion distance decreases.\relax }}{8}{figure.caption.4}\protected@file@percent }
\newlabel{fig:dist-ensemble-rmse-mae}{{4}{8}{\Gls {rmse} and \gls {mae} of pairwise distance errors for 388 Ni bicrystals \cite {olmstedSurveyComputedGrain2009} of scaled Euclidean distance approximation relative to the traditional octonion metric \cite {chesserLearningGrainBoundary2020} (compare with \cref {fig:dist-ensemble-k1-2-10-20}). The minimum distance among an ensemble of \gls {vfzo} sets ($\min _{\forall k \in [1,k_{max}]}x_{i,j,k}$, where $x_{i,j,k}$ is the scaled Euclidean distance) is taken, iteratively adding consecutive sets up to $k_{max} = 20$. As the number of \gls {vfzo} sets increases, \gls {rmse} and \gls {mae} between the scaled Euclidean distance approximation and the traditional octonion distance decreases.\relax }{figure.caption.4}{}}
\newlabel{fig:dist-ensemble-rmse-mae@cref}{{[figure][4][]4}{[1][6][]8}}
\@writefile{toc}{\contentsline {subsubsection}{\numberline {2.1.4}Comparison with Traditional Octonion Framework}{8}{subsubsection.2.1.4}\protected@file@percent }
\@LN@col{2}
\@writefile{toc}{\contentsline {subsection}{\numberline {2.2}Generating Random \@glsentrytitlecase {vfzo}{long}s}{8}{subsection.2.2}\protected@file@percent }
\newlabel{sec:methods:rand}{{2.2}{8}{Generating Random \glsentrytitlecase {vfzo}{long}s}{subsection.2.2}{}}
\newlabel{sec:methods:rand@cref}{{[subsection][2][2]2.2}{[1][8][]8}}
\citation{morawiecDistancesGrainInterfaces2019}
\citation{olmstedSurveyComputedGrain2009}
\@writefile{lot}{\contentsline {table}{\numberline {1}{\ignorespaces Comparison between \glsxtrlong {vfzo} and traditional octonion frameworks. *6D Cartesian representation used only for mesh triangulation efficiency in barycentric interpolation and *7D Cartesian representation only required for barycentric interpolation. 7D Cartesian representation is also implemented (though not required) for \cgls {gpr}, \cgls {nn}, and \cgls {idw}. For pairwise distance complexity, $N_p$ is the number of proper rotations ($N_p=24$ for $m\bar {3}m$ \cgls {fcc} point group) and $L$ is the number of \cglspl {gb}.\relax }}{9}{table.caption.5}\protected@file@percent }
\newlabel{tab:closed-mesh-comparison}{{1}{9}{Comparison between \glsxtrlong {vfzo} and traditional octonion frameworks. *6D Cartesian representation used only for mesh triangulation efficiency in barycentric interpolation and *7D Cartesian representation only required for barycentric interpolation. 7D Cartesian representation is also implemented (though not required) for \gls {gpr}, \gls {nn}, and \gls {idw}. For pairwise distance complexity, $N_p$ is the number of proper rotations ($N_p=24$ for $m\Bar {3}m$ \gls {fcc} point group) and $L$ is the number of \glspl {gb}.\relax }{table.caption.5}{}}
\newlabel{tab:closed-mesh-comparison@cref}{{[table][1][]1}{[1][8][]9}}
\@LN@col{1}
\@LN@col{2}
\@writefile{lof}{\contentsline {figure}{\numberline {5}{\ignorespaces \cGls {nn} \cgls {vfzo} ($\omega _{\text  {NN}}$) distances ($^{\circ }$) versus \cgls {vfzo} set size out of 70-80 random \cgls {vfzo} sets per set size.\relax }}{9}{figure.caption.6}\protected@file@percent }
\newlabel{fig:nndist-vs-setsize}{{5}{9}{\Gls {nn} \gls {vfzo} ($\omega _{\text {NN}}$) distances ($^{\circ }$) versus \gls {vfzo} set size out of 70-80 random \gls {vfzo} sets per set size.\relax }{figure.caption.6}{}}
\newlabel{fig:nndist-vs-setsize@cref}{{[figure][5][]5}{[1][9][]9}}
\citation{langerSphericalBarycentricCoordinates2006}
\citation{floaterGeneralizedBarycentricCoordinates2015,meyerGeneralizedBarycentricCoordinates2002,langerSphericalBarycentricCoordinates2006}
\citation{barberQuickhullAlgorithmConvex1996}
\citation{langerSphericalBarycentricCoordinates2006}
\citation{rasmussenGaussianProcessesMachine2006}
\citation{tovarInverseDistanceWeight2020}
\@LN@col{1}
\@writefile{toc}{\contentsline {subsection}{\numberline {2.3}Interpolation in the \@glsentrytitlecase {vfzo}{long} Framework}{10}{subsection.2.3}\protected@file@percent }
\newlabel{sec:methods:interp}{{2.3}{10}{Interpolation in the \glsentrytitlecase {vfzo}{long} Framework}{subsection.2.3}{}}
\newlabel{sec:methods:interp@cref}{{[subsection][3][2]2.3}{[1][9][]10}}
\@writefile{toc}{\contentsline {subsubsection}{\numberline {2.3.1}Barycentric Interpolation}{10}{subsubsection.2.3.1}\protected@file@percent }
\newlabel{sec:methods:interp:bary}{{2.3.1}{10}{Barycentric Interpolation}{subsubsection.2.3.1}{}}
\newlabel{sec:methods:interp:bary@cref}{{[subsubsection][1][2,3]2.3.1}{[1][10][]10}}
\@LN@col{2}
\@writefile{toc}{\contentsline {subsubsection}{\numberline {2.3.2}\@glsentrytitlecase {gpr}{long}}{10}{subsubsection.2.3.2}\protected@file@percent }
\newlabel{sec:methods:interp:gpr}{{2.3.2}{10}{\glsentrytitlecase {gpr}{long}}{subsubsection.2.3.2}{}}
\newlabel{sec:methods:interp:gpr@cref}{{[subsubsection][2][2,3]2.3.2}{[1][10][]10}}
\@writefile{toc}{\contentsline {subsubsection}{\numberline {2.3.3}\@glsentrytitlecase {idw}{long} Interpolation}{10}{subsubsection.2.3.3}\protected@file@percent }
\newlabel{sec:methods:interp:idw}{{2.3.3}{10}{\glsentrytitlecase {idw}{long} Interpolation}{subsubsection.2.3.3}{}}
\newlabel{sec:methods:interp:idw@cref}{{[subsubsection][3][2,3]2.3.3}{[1][10][]10}}
\@writefile{toc}{\contentsline {subsubsection}{\numberline {2.3.4}\@glsentrytitlecase {nn}{long} Interpolation}{10}{subsubsection.2.3.4}\protected@file@percent }
\newlabel{sec:methods:interp:nn}{{2.3.4}{10}{\glsentrytitlecase {nn}{long} Interpolation}{subsubsection.2.3.4}{}}
\newlabel{sec:methods:interp:nn@cref}{{[subsubsection][4][2,3]2.3.4}{[1][10][]10}}
\@writefile{toc}{\contentsline {subsection}{\numberline {2.4}Literature Datasets}{10}{subsection.2.4}\protected@file@percent }
\newlabel{sec:methods:litdata}{{2.4}{10}{Literature Datasets}{subsection.2.4}{}}
\newlabel{sec:methods:litdata@cref}{{[subsection][4][2]2.4}{[1][10][]10}}
\citation{kimPhasefieldModeling3D2014}
\citation{kimIdentificationSchemeGrain2011}
\citation{kimPhasefieldModeling3D2014}
\citation{chesserLearningGrainBoundary2020}
\citation{olmstedSurveyComputedGrain2009}
\citation{chesserGBOctonionCode2019}
\citation{chesserGBOctonionCode2019}
\citation{bulatovGrainBoundaryEnergy2014}
\citation{olmstedSurveyComputedGrain2009}
\citation{kimPhasefieldModeling3D2014}
\citation{olmstedSurveyComputedGrain2009}
\citation{beanHexscatter2020}
\citation{bulatovGrainBoundaryEnergy2014}
\citation{bulatovGrainBoundaryEnergy2014}
\@LN@col{1}
\@writefile{toc}{\contentsline {subsubsection}{\numberline {2.4.1}\@glsentrytitlecase {gpr}{long} for Fe Simulation Dataset}{11}{subsubsection.2.4.1}\protected@file@percent }
\newlabel{sec:methods:gprsim}{{2.4.1}{11}{\glsentrytitlecase {gpr}{long} for Fe Simulation Dataset}{subsubsection.2.4.1}{}}
\newlabel{sec:methods:gprsim@cref}{{[subsubsection][1][2,4]2.4.1}{[1][11][]11}}
\@writefile{toc}{\contentsline {subsubsection}{\numberline {2.4.2}\@glsentrytitlecase {gpr}{long} for Ni Simulation Dataset}{11}{subsubsection.2.4.2}\protected@file@percent }
\newlabel{sec:methods:gprsim-Ni}{{2.4.2}{11}{\glsentrytitlecase {gpr}{long} for Ni Simulation Dataset}{subsubsection.2.4.2}{}}
\newlabel{sec:methods:gprsim-Ni@cref}{{[subsubsection][2][2,4]2.4.2}{[1][11][]11}}
\@LN@col{2}
\@writefile{toc}{\contentsline {subsubsection}{\numberline {2.4.3}\@glsentrytitlecase {gprm}{long} for Fe Simulation Dataset}{11}{subsubsection.2.4.3}\protected@file@percent }
\newlabel{sec:methods:gprmix}{{2.4.3}{11}{\glsentrytitlecase {gprm}{long} for Fe Simulation Dataset}{subsubsection.2.4.3}{}}
\newlabel{sec:methods:gprmix@cref}{{[subsubsection][3][2,4]2.4.3}{[1][11][]11}}
\@writefile{toc}{\contentsline {section}{\numberline {3}Results and Discussion}{11}{section.3}\protected@file@percent }
\newlabel{sec:results}{{3}{11}{Results and Discussion}{section.3}{}}
\newlabel{sec:results@cref}{{[section][3][]3}{[1][11][]11}}
\@writefile{toc}{\contentsline {subsection}{\numberline {3.1}Interpolation Accuracy}{11}{subsection.3.1}\protected@file@percent }
\newlabel{sec:results:accuracy}{{3.1}{11}{Interpolation Accuracy}{subsection.3.1}{}}
\newlabel{sec:results:accuracy@cref}{{[subsection][1][3]3.1}{[1][11][]11}}
\@writefile{toc}{\contentsline {subsubsection}{\numberline {3.1.1}Accuracy of Four Interpolation Methods}{11}{subsubsection.3.1.1}\protected@file@percent }
\newlabel{sec:results:accuracy:interp}{{3.1.1}{11}{Accuracy of Four Interpolation Methods}{subsubsection.3.1.1}{}}
\newlabel{sec:results:accuracy:interp@cref}{{[subsubsection][1][3,1]3.1.1}{[1][11][]11}}
\citation{bulatovGrainBoundaryEnergy2014}
\citation{bulatovGrainBoundaryEnergy2014}
\citation{bulatovGrainBoundaryEnergy2014}
\citation{bulatovGrainBoundaryEnergy2014}
\citation{dolanBenchmarkingOptimizationSoftware2004,ConstrainedElectrostaticNonlinear2020}
\citation{francisGeodesicOctonionMetric2019}
\citation{francisGeodesicOctonionMetric2019}
\citation{restrepoUsingArtificialNeural2014}
\citation{restrepoUsingArtificialNeural2014}
\@LN@col{1}
\@LN@col{2}
\@writefile{toc}{\contentsline {subsubsection}{\numberline {3.1.2}Constant-Valued Control Models}{12}{subsubsection.3.1.2}\protected@file@percent }
\newlabel{sec:results:accuracy:control}{{3.1.2}{12}{Constant-Valued Control Models}{subsubsection.3.1.2}{}}
\newlabel{sec:results:accuracy:control@cref}{{[subsubsection][2][3,1]3.1.2}{[1][12][]12}}
\@writefile{lof}{\contentsline {figure}{\numberline {6}{\ignorespaces Hexagonally binned parity plots for \num {50000} input{} and \num {10000} prediction{} octonions formed via pairs of a random cubochorically sampled quaternion and a spherically sampled random boundary plane normal. Interpolation via (a) \cgls {gpr}, (b) \cgls {idw}, (c) \cgls {nn}, and (d) barycentric coordinates. \cgls {brk} \cgls {gbe} function for \cgls {fcc} Ni \cite  {bulatovGrainBoundaryEnergy2014} was used as the test function.\relax }}{13}{figure.caption.7}\protected@file@percent }
\newlabel{fig:brkparity50000}{{6}{13}{Hexagonally binned parity plots for \num {50000} \inpt {} and \num {10000} \outpt {} octonions formed via pairs of a random cubochorically sampled quaternion and a spherically sampled random boundary plane normal. Interpolation via (a) \gls {gpr}, (b) \gls {idw}, (c) \gls {nn}, and (d) barycentric coordinates. \gls {brk} \gls {gbe} function for \gls {fcc} Ni \cite {bulatovGrainBoundaryEnergy2014} was used as the test function.\relax }{figure.caption.7}{}}
\newlabel{fig:brkparity50000@cref}{{[figure][6][]6}{[1][11][]13}}
\@writefile{lot}{\contentsline {table}{\numberline {2}{\ignorespaces Comparison of average interpolation \cgls {rmse} (approximately 10 trial runs) for each interpolation method in the present work, using \num {50000} points in the definition of the \cgls {vfz} and \cglspl {gbe} obtained by evaluating the \cgls {brk} validation function (\cite  {bulatovGrainBoundaryEnergy2014}) at these points. A constant model (Cst, Avg \cgls {rmse}), whose value was chosen to be the mean of the input{} \cgls {gbe} was used as a control. The last two columns represent the reduction ($\downarrow $) in \cgls {rmse} in absolute units of \SI {}{\J \per \square \meter } and \% relative to the control model, respectively.\relax }}{13}{table.caption.8}\protected@file@percent }
\newlabel{tab:rmse-error-comparison}{{2}{13}{Comparison of average interpolation \gls {rmse} (approximately 10 trial runs) for each interpolation method in the present work, using \num {50000} points in the definition of the \gls {vfz} and \glspl {gbe} obtained by evaluating the \gls {brk} validation function (\cite {bulatovGrainBoundaryEnergy2014}) at these points. A constant model (Cst, Avg \gls {rmse}), whose value was chosen to be the mean of the \inpt {} \gls {gbe} was used as a control. The last two columns represent the reduction ($\downarrow $) in \gls {rmse} in absolute units of \SI {}{\J \per \square \meter } and \% relative to the control model, respectively.\relax }{table.caption.8}{}}
\newlabel{tab:rmse-error-comparison@cref}{{[table][2][]2}{[1][12][]13}}
\@LN@col{1}
\@LN@col{2}
\@writefile{lot}{\contentsline {table}{\numberline {3}{\ignorespaces Comparison of average interpolation \cgls {mae} (approximately 10 trial runs) for each interpolation method in the present work, using \num {50000} points in the definition of the \cgls {vfz} and \cglspl {gbe} obtained by evaluating the \cgls {brk} validation function (\cite  {bulatovGrainBoundaryEnergy2014}) at these points. A constant model (Cst, Avg \cgls {mae}), whose value was chosen to be the mean of the input{} \cgls {gbe} was used as a control. The last two columns represent the reduction ($\downarrow $) in \cgls {mae} in absolute units of \SI {}{\J \per \square \meter } and \% relative to the control model, respectively.\relax }}{14}{table.caption.9}\protected@file@percent }
\newlabel{tab:mae-error-comparison}{{3}{14}{Comparison of average interpolation \gls {mae} (approximately 10 trial runs) for each interpolation method in the present work, using \num {50000} points in the definition of the \gls {vfz} and \glspl {gbe} obtained by evaluating the \gls {brk} validation function (\cite {bulatovGrainBoundaryEnergy2014}) at these points. A constant model (Cst, Avg \gls {mae}), whose value was chosen to be the mean of the \inpt {} \gls {gbe} was used as a control. The last two columns represent the reduction ($\downarrow $) in \gls {mae} in absolute units of \SI {}{\J \per \square \meter } and \% relative to the control model, respectively.\relax }{table.caption.9}{}}
\newlabel{tab:mae-error-comparison@cref}{{[table][3][]3}{[1][12][]14}}
\@writefile{lof}{\contentsline {figure}{\numberline {7}{\ignorespaces (a) Average \cgls {rmse} and (b) average \cgls {mae} vs. number of input{} points for (planar) barycentric (blue), \cgls {gpr} (orange), \cgls {idw} (yellow), and \cgls {nn} (purple) interpolation for approximately 10 random runs with different input{} and prediction{} points. Standard deviations of approximately 10 runs are also included. Compare with approximately \SI {0.1283{}}{\J \per \square \meter } and \SI {0.0955{}}{\J \per \square \meter } \cgls {rmse} and \cgls {mae}, respectively, for a constant, average model (green) using the average of the input{} properties (approximately \SI {1.16}{\J \per \square \meter }).\relax }}{14}{figure.caption.10}\protected@file@percent }
\newlabel{fig:brkerror}{{7}{14}{(a) Average \gls {rmse} and (b) average \gls {mae} vs. number of \inpt {} points for (planar) barycentric (blue), \gls {gpr} (orange), \gls {idw} (yellow), and \gls {nn} (purple) interpolation for approximately 10 random runs with different \inpt {} and \outpt {} points. Standard deviations of approximately 10 runs are also included. Compare with approximately \SI {\avgrmse {}}{\J \per \square \meter } and \SI {\avgmae {}}{\J \per \square \meter } \gls {rmse} and \gls {mae}, respectively, for a constant, average model (green) using the average of the \inpt {} properties (approximately \SI {1.16}{\J \per \square \meter }).\relax }{figure.caption.10}{}}
\newlabel{fig:brkerror@cref}{{[figure][7][]7}{[1][12][]14}}
\@writefile{lot}{\contentsline {table}{\numberline {4}{\ignorespaces Approximate coordinates of \cglspl {vfzo} A and B used for the interpolation in \cref  {fig:tunnel-50000}. Individual quaternions of each octonion are given in the active sense and in the laboratory reference frame with an assumed \cgls {gb} normal pointing in the +z direction, also in the laboratory reference frame.\relax }}{14}{table.caption.11}\protected@file@percent }
\newlabel{tab:tunnel-AB}{{4}{14}{Approximate coordinates of \glspl {vfzo} A and B used for the interpolation in \cref {fig:tunnel-50000}. Individual quaternions of each octonion are given in the active sense and in the laboratory reference frame with an assumed \gls {gb} normal pointing in the +z direction, also in the laboratory reference frame.\relax }{table.caption.11}{}}
\newlabel{tab:tunnel-AB@cref}{{[table][4][]4}{[1][12][]14}}
\citation{shenDeterminingGrainBoundary2019}
\citation{shenDeterminingGrainBoundary2019}
\citation{shenDeterminingGrainBoundary2019}
\citation{shenDeterminingGrainBoundary2019}
\citation{shenDeterminingGrainBoundary2019}
\@LN@col{1}
\@writefile{lof}{\contentsline {figure}{\numberline {8}{\ignorespaces Predictions of \cgls {gpr} (blue circles), barycentric (red circles), \cgls {nn} (magenta circles), and \cgls {idw} (green circles) as a function of distance along a 1D arc ($\overline  {AB}$) between two \cglspl {vfzo} ($A$ and $B$). The true, underlying \cgls {brk} function is also shown (black line). \num {50000} random input{} \cglspl {vfzo} were generated and used for each of the models. \num {150} equally spaced points between $A$ and $B$ obtained via \cgls {oslerp} \cite  {francisGeodesicOctonionMetric2019} were used as prediction{} points. \cgls {gpr} uncertainty standard deviation is plotted as shaded error band.\relax }}{15}{figure.caption.12}\protected@file@percent }
\newlabel{fig:tunnel-50000}{{8}{15}{Predictions of \gls {gpr} (blue circles), barycentric (red circles), \gls {nn} (magenta circles), and \gls {idw} (green circles) as a function of distance along a 1D arc ($\overline {AB}$) between two \glspl {vfzo} ($A$ and $B$). The true, underlying \gls {brk} function is also shown (black line). \num {50000} random \inpt {} \glspl {vfzo} were generated and used for each of the models. \num {150} equally spaced points between $A$ and $B$ obtained via \gls {oslerp} \cite {francisGeodesicOctonionMetric2019} were used as \outpt {} points. \gls {gpr} uncertainty standard deviation is plotted as shaded error band.\relax }{figure.caption.12}{}}
\newlabel{fig:tunnel-50000@cref}{{[figure][8][]8}{[1][12][]15}}
\@writefile{toc}{\contentsline {subsubsection}{\numberline {3.1.3}Experimental and Simulation Error}{15}{subsubsection.3.1.3}\protected@file@percent }
\newlabel{sec:results:accuracy:exp-sim}{{3.1.3}{15}{Experimental and Simulation Error}{subsubsection.3.1.3}{}}
\newlabel{sec:results:accuracy:exp-sim@cref}{{[subsubsection][3][3,1]3.1.3}{[1][15][]15}}
\@LN@col{2}
\citation{bulatovGrainBoundaryEnergy2014}
\citation{bulatovGrainBoundaryEnergy2014}
\citation{francisGeodesicOctonionMetric2019}
\citation{chesserLearningGrainBoundary2020}
\citation{degraefEMSoft2020}
\citation{bairdFiveDegreeofFreedom5DOF2020}
\citation{degraefEMSoft2020}
\citation{chesserLearningGrainBoundary2020}
\citation{morawiecDistancesGrainInterfaces2019}
\citation{kimPhasefieldModeling3D2014,dimokratiSPFMModelIdeal2020}
\citation{miyoshiLargescalePhasefieldStudy2021}
\@LN@col{1}
\@writefile{toc}{\contentsline {subsection}{\numberline {3.2}Interpolation Efficiency}{16}{subsection.3.2}\protected@file@percent }
\newlabel{sec:results:efficiency}{{3.2}{16}{Interpolation Efficiency}{subsection.3.2}{}}
\newlabel{sec:results:efficiency@cref}{{[subsection][2][3]3.2}{[1][15][]16}}
\@writefile{toc}{\contentsline {subsubsection}{\numberline {3.2.1}Efficiency of Four Interpolation Methods}{16}{subsubsection.3.2.1}\protected@file@percent }
\newlabel{sec:results:efficiency:methods}{{3.2.1}{16}{Efficiency of Four Interpolation Methods}{subsubsection.3.2.1}{}}
\newlabel{sec:results:efficiency:methods@cref}{{[subsubsection][1][3,2]3.2.1}{[1][16][]16}}
\@LN@col{2}
\@writefile{toc}{\contentsline {subsubsection}{\numberline {3.2.2}Symmetrization Runtime Comparison with Traditional Octonion Metric}{16}{subsubsection.3.2.2}\protected@file@percent }
\newlabel{sec:results:efficiency:symruntime}{{3.2.2}{16}{Symmetrization Runtime Comparison with Traditional Octonion Metric}{subsubsection.3.2.2}{}}
\newlabel{sec:results:efficiency:symruntime@cref}{{[subsubsection][2][3,2]3.2.2}{[1][16][]16}}
\citation{chesserLearningGrainBoundary2020}
\citation{bairdFiveDegreeofFreedom5DOF2020}
\@writefile{lot}{\contentsline {table}{\numberline {5}{\ignorespaces Comparison of average~runtime~(\SI {}{\second }) for \num {10} trials for barycentric, \cgls {gpr}, \cgls {idw}, and \cgls {nn} interpolation methods for various input{} \cgls {vfzo} set sizes using 12 cores and evaluated on \num {10000} prediction{} \cglspl {vfzo}. Because \cgls {gpr}, \cgls {idw}, and \cgls {nn} method defaults do not use \mbox  {\lstinline [style=Matlab-editor]{parfor}} loops but may have internal multi-core vectorization, it is unclear to what extent the number of cores affects the runtime of methods other than barycentric interpolation. \cGls {vfzo} symmetrization runtime was not included; however, symmetrization of \num {50000} \cglspl {gbo} takes approximately \SI {76}{seconds} on \SI {6}{cores} (Intel i7-10750H, 2.6 GHz) and is a common step in every interpolation method (i.e. it is fundamental to the \cgls {vfzo} framework). We used the \cgls {brk} validation function for \cgls {gbe} \cite  {bulatovGrainBoundaryEnergy2014}. \relax }}{17}{table.caption.13}\protected@file@percent }
\newlabel{tab:runtime}{{5}{17}{Comparison of average~runtime~(\SI {}{\second }) for \num {10} trials for barycentric, \gls {gpr}, \gls {idw}, and \gls {nn} interpolation methods for various \inpt {} \gls {vfzo} set sizes using 12 cores and evaluated on \num {10000} \outpt {} \glspl {vfzo}. Because \gls {gpr}, \gls {idw}, and \gls {nn} method defaults do not use \matlab {parfor} loops but may have internal multi-core vectorization, it is unclear to what extent the number of cores affects the runtime of methods other than barycentric interpolation. \Gls {vfzo} symmetrization runtime was not included; however, symmetrization of \num {50000} \glspl {gbo} takes approximately \SI {\symtime }{seconds} on \SI {6}{cores} (Intel i7-10750H, 2.6 GHz) and is a common step in every interpolation method (i.e. it is fundamental to the \gls {vfzo} framework). We used the \gls {brk} validation function for \gls {gbe} \cite {bulatovGrainBoundaryEnergy2014}. \relax }{table.caption.13}{}}
\newlabel{tab:runtime@cref}{{[table][5][]5}{[1][16][]17}}
\@LN@col{1}
\@LN@col{2}
\citation{francisGeodesicOctonionMetric2019}
\citation{francisGeodesicOctonionMetric2019}
\citation{chesserLearningGrainBoundary2020}
\citation{morawiecDistancesGrainInterfaces2019}
\@LN@col{1}
\@writefile{toc}{\contentsline {subsection}{\numberline {3.3}Interpolation Visualization}{18}{subsection.3.3}\protected@file@percent }
\@writefile{toc}{\contentsline {subsubsection}{\numberline {3.3.1}Interpolation Along a 1D Arc}{18}{subsubsection.3.3.1}\protected@file@percent }
\newlabel{sec:results:vis:arc}{{3.3.1}{18}{Interpolation Along a 1D Arc}{subsubsection.3.3.1}{}}
\newlabel{sec:results:vis:arc@cref}{{[subsubsection][1][3,3]3.3.1}{[1][18][]18}}
\@LN@col{2}
\@writefile{toc}{\contentsline {subsubsection}{\numberline {3.3.2}Potential for Numerical Derivatives}{18}{subsubsection.3.3.2}\protected@file@percent }
\newlabel{sec:results:vis:apps}{{3.3.2}{18}{Potential for Numerical Derivatives}{subsubsection.3.3.2}{}}
\newlabel{sec:results:vis:apps@cref}{{[subsubsection][2][3,3]3.3.2}{[1][18][]18}}
\@writefile{toc}{\contentsline {subsection}{\numberline {3.4}Literature Datasets}{18}{subsection.3.4}\protected@file@percent }
\newlabel{sec:results:simulation}{{3.4}{18}{Literature Datasets}{subsection.3.4}{}}
\newlabel{sec:results:simulation@cref}{{[subsection][4][3]3.4}{[1][18][]18}}
\citation{restrepoUsingArtificialNeural2014}
\citation{chesserLearningGrainBoundary2020}
\citation{restrepoUsingArtificialNeural2014}
\citation{chesserLearningGrainBoundary2020}
\citation{restrepoUsingArtificialNeural2014}
\citation{restrepoUsingArtificialNeural2014}
\citation{restrepoUsingArtificialNeural2014}
\citation{chesserLearningGrainBoundary2020}
\citation{restrepoUsingArtificialNeural2014}
\citation{restrepoUsingArtificialNeural2014}
\citation{restrepoUsingArtificialNeural2014}
\citation{chesserLearningGrainBoundary2020}
\citation{restrepoUsingArtificialNeural2014}
\citation{kimPhasefieldModeling3D2014}
\citation{kimPhasefieldModeling3D2014}
\citation{kimPhasefieldModeling3D2014}
\@LN@col{1}
\@writefile{toc}{\contentsline {subsubsection}{\numberline {3.4.1}Comparison with Prior Work}{19}{subsubsection.3.4.1}\protected@file@percent }
\newlabel{sec:results:simulation:compare}{{3.4.1}{19}{Comparison with Prior Work}{subsubsection.3.4.1}{}}
\newlabel{sec:results:simulation:compare@cref}{{[subsubsection][1][3,4]3.4.1}{[1][19][]19}}
\@LN@col{2}
\@writefile{toc}{\contentsline {subsubsection}{\numberline {3.4.2}\@glsentrytitlecase {gprm}{long} Applied to Metastable Fe Simulation Data}{19}{subsubsection.3.4.2}\protected@file@percent }
\newlabel{sec:results:simulation:gprm}{{3.4.2}{19}{\glsentrytitlecase {gprm}{long} Applied to Metastable Fe Simulation Data}{subsubsection.3.4.2}{}}
\newlabel{sec:results:simulation:gprm@cref}{{[subsubsection][2][3,4]3.4.2}{[1][19][]19}}
\@writefile{lot}{\contentsline {table}{\numberline {6}{\ignorespaces Comparison of interpolation \cgls {mae} (1 trial run) for \SI {0}{\kelvin } \glsxtrfull {ms} datasets. A constant model (Cst, Avg \cgls {mae}), whose value was chosen to be the mean of the input{} \cgls {gbe} was used as a control. The last two columns, \cgls {mae} $\downarrow $ (\SI {}{\J \per \square \meter }) and \cgls {mae} $\downarrow $ (\%)), represent the reduction in \cgls {mae} in units of \SI {}{\J \per \square \meter } and \% relative to the control model, respectively. Non-sym refers to distances calculated in \citet  {restrepoUsingArtificialNeural2014} without regard for crystal symmetries. \relax }}{20}{table.caption.14}\protected@file@percent }
\newlabel{tab:mae-error-simulation}{{6}{20}{Comparison of interpolation \gls {mae} (1 trial run) for \SI {0}{\kelvin } \glsxtrfull {ms} datasets. A constant model (Cst, Avg \gls {mae}), whose value was chosen to be the mean of the \inpt {} \gls {gbe} was used as a control. The last two columns, \gls {mae} $\downarrow $ (\SI {}{\J \per \square \meter }) and \gls {mae} $\downarrow $ (\%)), represent the reduction in \gls {mae} in units of \SI {}{\J \per \square \meter } and \% relative to the control model, respectively. Non-sym refers to distances calculated in \citet {restrepoUsingArtificialNeural2014} without regard for crystal symmetries. \relax }{table.caption.14}{}}
\newlabel{tab:mae-error-simulation@cref}{{[table][6][]6}{[1][19][]20}}
\@writefile{lot}{\contentsline {table}{\numberline {7}{\ignorespaces Comparison of interpolation \cgls {rmse} (1 trial run) for \SI {0}{\kelvin } \glsxtrfull {ms} datasets. A constant model (Cst, Avg \cgls {rmse}), whose value was chosen to be the mean of the input{} \cgls {gbe} was used as a control. The last two columns, \cgls {rmse} $\downarrow $ (\SI {}{\J \per \square \meter }) and \cgls {rmse} $\downarrow $ (\%)), represent the reduction in \cgls {rmse} in units of \SI {}{\J \per \square \meter } and \% relative to the control model, respectively. Non-sym refers to distances calculated in \citet  {restrepoUsingArtificialNeural2014} without regard for crystal symmetries. \relax }}{20}{table.caption.15}\protected@file@percent }
\newlabel{tab:rmse-error-simulation}{{7}{20}{Comparison of interpolation \gls {rmse} (1 trial run) for \SI {0}{\kelvin } \glsxtrfull {ms} datasets. A constant model (Cst, Avg \gls {rmse}), whose value was chosen to be the mean of the \inpt {} \gls {gbe} was used as a control. The last two columns, \gls {rmse} $\downarrow $ (\SI {}{\J \per \square \meter }) and \gls {rmse} $\downarrow $ (\%)), represent the reduction in \gls {rmse} in units of \SI {}{\J \per \square \meter } and \% relative to the control model, respectively. Non-sym refers to distances calculated in \citet {restrepoUsingArtificialNeural2014} without regard for crystal symmetries. \relax }{table.caption.15}{}}
\newlabel{tab:rmse-error-simulation@cref}{{[table][7][]7}{[1][19][]20}}
\@writefile{lof}{\contentsline {figure}{\numberline {9}{\ignorespaces Hexagonally binned parity plots of (a) \cgls {brk} and (b) \cgls {gpr} model \cglspl {gbe} fitted using Olmsted Ni simulation data vs. Olmsted Ni simulation \cglspl {gbe}. \cGls {mae} is \SIlist {0.00975;0.03626}{\J \per \square \m } for (a) and (b), respectively. Likewise, \cgls {rmse} is \SIlist {0.01727;0.04972}{\J \per \square \m }, respectively.\relax }}{20}{figure.caption.16}\protected@file@percent }
\newlabel{fig:resubloss-ni}{{9}{20}{Hexagonally binned parity plots of (a) \gls {brk} and (b) \gls {gpr} model \glspl {gbe} fitted using Olmsted Ni simulation data vs. Olmsted Ni simulation \glspl {gbe}. \Gls {mae} is \SIlist {0.00975;0.03626}{\J \per \square \m } for (a) and (b), respectively. Likewise, \gls {rmse} is \SIlist {0.01727;0.04972}{\J \per \square \m }, respectively.\relax }{figure.caption.16}{}}
\newlabel{fig:resubloss-ni@cref}{{[figure][9][]9}{[1][19][]20}}
\@writefile{lot}{\contentsline {table}{\numberline {8}{\ignorespaces Approximate coordinates of \cglspl {vfzo} $A$ and $B$ used for the \cgls {ms} Fe simulation dataset interpolation in \cref  {fig:kim-interp}. Individual quaternions of each octonion are given in the laboratory reference frame with an assumed \cgls {gb} normal pointing in the +z direction, also in the laboratory reference frame.\relax }}{20}{table.caption.18}\protected@file@percent }
\newlabel{tab:tunnel-AB2}{{8}{20}{Approximate coordinates of \glspl {vfzo} $A$ and $B$ used for the \gls {ms} Fe simulation dataset interpolation in \cref {fig:kim-interp}. Individual quaternions of each octonion are given in the laboratory reference frame with an assumed \gls {gb} normal pointing in the +z direction, also in the laboratory reference frame.\relax }{table.caption.18}{}}
\newlabel{tab:tunnel-AB2@cref}{{[table][8][]8}{[1][19][]20}}
\citation{hanGrainboundaryMetastabilityIts2016}
\citation{hanGrainboundaryMetastabilityIts2016}
\citation{weiDirectImagingAtomistic2021}
\@writefile{lof}{\contentsline {figure}{\numberline {10}{\ignorespaces Interpolation results for a large Fe simulation database \cite  {kimPhasefieldModeling3D2014} using \num {46883} input{} \cglspl {gb} and \num {11721} prediction{} \cglspl {gb} in an 80\%/20\% split and a \cgls {gprm} model to better approximate low \cglspl {gbe}. Use of a \cgls {gprm} model predicts low \cgls {gbe} better than the standard \cgls {gpr} model (compare with \cref  {fig:kim-interp-teach}d). (a) Hexagonally binned parity plot of the \cgls {gpr} mixing model with \cgls {rmse} and \cgls {mae} of \SIlist {0.055035;0.039185}{\J \per \square \meter }, respectively, relative to typical, constant average models of \SIlist {0.0854;0.0617}{\joule \per \square \meter }, respectively. (b) Predictions of \cgls {gprm} model (blue circles) as a function of distance along a 1D arc ($\overline  {AB}$) between two \cglspl {vfzo} ($A$ and $B$). \relax }}{21}{figure.caption.17}\protected@file@percent }
\newlabel{fig:kim-interp}{{10}{21}{Interpolation results for a large Fe simulation database \cite {kimPhasefieldModeling3D2014} using \num {46883} \inpt {} \glspl {gb} and \num {11721} \outpt {} \glspl {gb} in an 80\%/20\% split and a \gls {gprm} model to better approximate low \glspl {gbe}. Use of a \gls {gprm} model predicts low \gls {gbe} better than the standard \gls {gpr} model (compare with \cref {fig:kim-interp-teach}d). (a) Hexagonally binned parity plot of the \gls {gpr} mixing model with \gls {rmse} and \gls {mae} of \SIlist {0.055035;0.039185}{\J \per \square \meter }, respectively, relative to typical, constant average models of \SIlist {0.0854;0.0617}{\joule \per \square \meter }, respectively. (b) Predictions of \gls {gprm} model (blue circles) as a function of distance along a 1D arc ($\overline {AB}$) between two \glspl {vfzo} ($A$ and $B$). \relax }{figure.caption.17}{}}
\newlabel{fig:kim-interp@cref}{{[figure][10][]10}{[1][19][]21}}
\@LN@col{1}
\@LN@col{2}
\citation{bairdFiveDegreeofFreedom5DOF2020}
\citation{morawiecDistancesGrainInterfaces2019}
\citation{shenDeterminingGrainBoundary2019}
\@LN@col{1}
\@writefile{toc}{\contentsline {section}{\numberline {4}Conclusion}{22}{section.4}\protected@file@percent }
\newlabel{sec:conclusion}{{4}{22}{Conclusion}{section.4}{}}
\newlabel{sec:conclusion@cref}{{[section][4][]4}{[1][21][]22}}
\@LN@col{2}
\newlabel{sec:acknowledgement}{{4}{22}{Acknowledgement}{section*.19}{}}
\newlabel{sec:acknowledgement@cref}{{[section][4][]4}{[1][22][]22}}
\citation{francisGeodesicOctonionMetric2019}
\citation{rowenhorstConsistentRepresentationsConversions2015}
\citation{rowenhorstConsistentRepresentationsConversions2015}
\citation{barberQuickhullAlgorithmConvex1996}
\@LN@col{1}
\@writefile{toc}{\contentsline {section}{\numberline {A}Active vs. Passive Convention}{23}{appendix.alph1.A}\protected@file@percent }
\newlabel{sec:app:convention}{{A}{23}{Active vs. Passive Convention}{appendix.alph1.A}{}}
\newlabel{sec:app:convention@cref}{{[appsec][1][]A}{[1][23][]23}}
\@writefile{toc}{\contentsline {section}{\numberline {B}Detailed Barycentric Interpolation Method}{23}{appendix.alph1.B}\protected@file@percent }
\newlabel{sec:app:bary}{{B}{23}{Detailed Barycentric Interpolation Method}{appendix.alph1.B}{}}
\newlabel{sec:app:bary@cref}{{[appsec][2][]B}{[1][23][]23}}
\@LN@col{2}
\@writefile{toc}{\contentsline {subsection}{\numberline {B.1}Triangulating a \@glsentrytitlecase {vfz}{long} Mesh}{23}{subsection.alph1.B.1}\protected@file@percent }
\newlabel{sec:app:bary:tri}{{B.1}{23}{Triangulating a \glsentrytitlecase {vfz}{long} Mesh}{subsection.alph1.B.1}{}}
\newlabel{sec:app:bary:tri@cref}{{[appsec][1][2]B.1}{[1][23][]23}}
\citation{barberQuickhullAlgorithmConvex1996}
\citation{francisGeodesicOctonionMetric2019}
\citation{barberQuickhullAlgorithmConvex1996}
\citation{morawiecDistancesGrainInterfaces2019}
\citation{connorHighdimensionalSimplexesSupermetric2017,boissonnatOnlyDistancesAre2017}
\@LN@col{1}
\@writefile{toc}{\contentsline {subsubsection}{\numberline {B.1.1}\@glsentrytitlecase {svd}{long} Transformation from 8D Cartesian to 7D Cartesian}{24}{subsubsection.alph1.B.1.1}\protected@file@percent }
\newlabel{sec:app:bary:tri:svd1}{{B.1.1}{24}{\glsentrytitlecase {svd}{long} Transformation from 8D Cartesian to 7D Cartesian}{subsubsection.alph1.B.1.1}{}}
\newlabel{sec:app:bary:tri:svd1@cref}{{[appsec][1][2,1]B.1.1}{[1][24][]24}}
\@writefile{toc}{\contentsline {subsubsection}{\numberline {B.1.2}Linearly Project onto Hyperplane}{24}{subsubsection.alph1.B.1.2}\protected@file@percent }
\newlabel{sec:app:bary:tri:project}{{B.1.2}{24}{Linearly Project onto Hyperplane}{subsubsection.alph1.B.1.2}{}}
\newlabel{sec:app:bary:tri:project@cref}{{[appsec][2][2,1]B.1.2}{[1][24][]24}}
\@LN@col{2}
\@writefile{toc}{\contentsline {subsubsection}{\numberline {B.1.3}\@glsentrytitlecase {svd}{long} Transformation from 7D Cartesian to 6D Cartesian}{24}{subsubsection.alph1.B.1.3}\protected@file@percent }
\newlabel{sec:app:bary:tri:svd2}{{B.1.3}{24}{\glsentrytitlecase {svd}{long} Transformation from 7D Cartesian to 6D Cartesian}{subsubsection.alph1.B.1.3}{}}
\newlabel{sec:app:bary:tri:svd2@cref}{{[appsec][3][2,1]B.1.3}{[1][24][]24}}
\@writefile{toc}{\contentsline {subsection}{\numberline {B.2}Intersections in a \@glsentrytitlecase {vfz}{long} Mesh}{24}{subsection.alph1.B.2}\protected@file@percent }
\newlabel{sec:app:bary:int}{{B.2}{24}{Intersections in a \glsentrytitlecase {vfz}{long} Mesh}{subsection.alph1.B.2}{}}
\newlabel{sec:app:bary:int@cref}{{[appsec][2][2]B.2}{[1][24][]24}}
\@writefile{toc}{\contentsline {subsubsection}{\numberline {B.2.1}Apply Same \@glsentrytitlecase {svd}{long} to Input and Prediction Points}{24}{subsubsection.alph1.B.2.1}\protected@file@percent }
\newlabel{sec:app:bary:int:out-svd}{{B.2.1}{24}{Apply Same \glsentrytitlecase {svd}{long} to Input and Prediction Points}{subsubsection.alph1.B.2.1}{}}
\newlabel{sec:app:bary:int:out-svd@cref}{{[appsec][1][2,2]B.2.1}{[1][24][]24}}
\@writefile{lof}{\contentsline {figure}{\numberline {B.1}{\ignorespaces 3D Cartesian to 2D Cartesian analogue of 8D Cartesian to 7D Cartesian degeneracy removal via rigid \cgls {svd} transformation as used in barycentric interpolation approach. (a) Starting spherical arc points on surface of 2-sphere, (b) rotational symmetrization applied w.r.t. z-axis (analogous to U(1) symmetrization), and (c) degenerate dimension removed via \glsxtrlong {svd} transformation to 2D Cartesian with either the origin (black plus) preserved (black asterisks, \mbox  {\lstinline [style=Matlab-editor]{zeroQ=T}}) for triangulation or ignored (red asterisks, \mbox  {\lstinline [style=Matlab-editor]{zeroQ=F}}) for mesh intersection. The spheres (a,b) and circle (c) each have a radius of 0.8 and are used as a visualization aid only.\relax }}{25}{figure.caption.21}\protected@file@percent }
\newlabel{fig:bary-remove-deg}{{B.1}{25}{3D Cartesian to 2D Cartesian analogue of 8D Cartesian to 7D Cartesian degeneracy removal via rigid \gls {svd} transformation as used in barycentric interpolation approach. (a) Starting spherical arc points on surface of 2-sphere, (b) rotational symmetrization applied w.r.t. z-axis (analogous to U(1) symmetrization), and (c) degenerate dimension removed via \glsxtrlong {svd} transformation to 2D Cartesian with either the origin (black plus) preserved (black asterisks, \matlab {zeroQ=T}) for triangulation or ignored (red asterisks, \matlab {zeroQ=F}) for mesh intersection. The spheres (a,b) and circle (c) each have a radius of 0.8 and are used as a visualization aid only.\relax }{figure.caption.21}{}}
\newlabel{fig:bary-remove-deg@cref}{{[figure][1][]B.1}{[1][24][]25}}
\@writefile{lof}{\contentsline {figure}{\numberline {B.2}{\ignorespaces 3D Cartesian to 2D Cartesian analogue of 7D Cartesian to 6D Cartesian mesh triangulation used in barycentric interpolation approach. (a) 3D Cartesian input{} points are (b) linearly projected onto hyperplane that is tangent to mean of starting points. (c) The degenerate dimension is removed via a rigid \cgls {svd} transformation to 2D Cartesian and the Delaunay triangulation (black lines) is calculated, with input{} vertices (red). Delaunay triangulation superimposed onto normalized input{} points (d). The spheres in (a), (b), and (d) have a radius of 0.8 and are used for visualization only.\relax }}{25}{figure.caption.22}\protected@file@percent }
\newlabel{fig:bary-delaunay}{{B.2}{25}{3D Cartesian to 2D Cartesian analogue of 7D Cartesian to 6D Cartesian mesh triangulation used in barycentric interpolation approach. (a) 3D Cartesian \inpt {} points are (b) linearly projected onto hyperplane that is tangent to mean of starting points. (c) The degenerate dimension is removed via a rigid \gls {svd} transformation to 2D Cartesian and the Delaunay triangulation (black lines) is calculated, with \inpt {} vertices (red). Delaunay triangulation superimposed onto normalized \inpt {} points (d). The spheres in (a), (b), and (d) have a radius of 0.8 and are used for visualization only.\relax }{figure.caption.22}{}}
\newlabel{fig:bary-delaunay@cref}{{[figure][2][]B.2}{[1][24][]25}}
\citation{anatoliyCheckIfRay2015,skalaRobustBarycentricCoordinates2013}
\citation{langerSphericalBarycentricCoordinates2006}
\citation{anisimovSubdividingBarycentricCoordinates2016,budninskiyPowerCoordinatesGeometric2016,dyerBarycentricCoordinateNeighbourhoods2016,floaterGeneralizedBarycentricCoordinates2015,floaterInjectivityWachspressMean2010,hormannDiscretizingWachspressKernels2017,hormannMaximumEntropyCoordinates2008,langerHigherOrderBarycentric2008,langerSphericalBarycentricCoordinates2006,leiNewCoordinateSystem2020,meyerGeneralizedBarycentricCoordinates2002,peixotoVectorFieldReconstructions2014,pihajokiBarycentricInterpolationRiemannian2019,rustamovBarycentricCoordinatesSurfaces2010,skalaRobustBarycentricCoordinates2013,taoFastNumericalSolver2019,warrenBarycentricCoordinatesConvex2007}
\@LN@col{1}
\@writefile{toc}{\contentsline {subsubsection}{\numberline {B.2.2}Testing Nearby Facets for Intersections}{26}{subsubsection.alph1.B.2.2}\protected@file@percent }
\newlabel{sec:app:bary:int:facets}{{B.2.2}{26}{Testing Nearby Facets for Intersections}{subsubsection.alph1.B.2.2}{}}
\newlabel{sec:app:bary:int:facets@cref}{{[appsec][2][2,2]B.2.2}{[1][26][]26}}
\@LN@col{2}
\@writefile{lof}{\contentsline {figure}{\numberline {B.3}{\ignorespaces A ray (red line) is linearly projected from the 2-sphere onto the hyperplane of a mesh facet (transparent black), shown as a red asterisk. The barycentric coordinates are computed as $\lambda _{i \in [1,3]} = \frac  {1}{3}$. Because all barycentric coordinates are positive, it is determined that the projected point is an intersection with the mesh. Given vertex values of \num {8.183}, \num {3.446}, and \num {3.188} for vertices 1, 2, and 3, respectively, the interpolated value is calculated as \num {4.94} via \cref  {eq:bary-interp}.\relax }}{26}{figure.caption.23}\protected@file@percent }
\newlabel{fig:bary-interp}{{B.3}{26}{A ray (red line) is linearly projected from the 2-sphere onto the hyperplane of a mesh facet (transparent black), shown as a red asterisk. The barycentric coordinates are computed as $\lambda _{i \in [1,3]} = \frac {1}{3}$. Because all barycentric coordinates are positive, it is determined that the projected point is an intersection with the mesh. Given vertex values of \num {8.183}, \num {3.446}, and \num {3.188} for vertices 1, 2, and 3, respectively, the interpolated value is calculated as \num {4.94} via \cref {eq:bary-interp}.\relax }{figure.caption.23}{}}
\newlabel{fig:bary-interp@cref}{{[figure][3][]B.3}{[1][26][]26}}
\@LN@col{1}
\@writefile{toc}{\contentsline {subsection}{\numberline {B.3}Interpolation via Barycentric Coordinates}{27}{subsection.alph1.B.3}\protected@file@percent }
\newlabel{sec:app:bary-interp}{{B.3}{27}{Interpolation via Barycentric Coordinates}{subsection.alph1.B.3}{}}
\newlabel{sec:app:bary-interp@cref}{{[appsec][3][2]B.3}{[1][27][]27}}
\@LN@col{2}
\newlabel{eq:bary-interp}{{6}{27}{Interpolation via Barycentric Coordinates}{equation.alph1.B.6}{}}
\newlabel{eq:bary-interp@cref}{{[equation][6][]6}{[1][27][]27}}
\@writefile{toc}{\contentsline {section}{Glossary}{27}{section*.24}\protected@file@percent }
\bibstyle{elsarticle-num-names}
\bibdata{5dof-gb-energy.bib}
\providecommand*{\@gls@entry@count}[2]{}
\@gls@entry@count{5dof}{9}
\@gls@entry@count{3dof}{1}
\@gls@entry@count{dof}{5}
\@gls@entry@count{gb}{108}
\@gls@entry@count{fcc}{3}
\@gls@entry@count{tj}{2}
\@gls@entry@count{gpr}{57}
\@gls@entry@count{gprm}{13}
\@gls@entry@count{ann}{6}
\@gls@entry@count{nn}{50}
\@gls@entry@count{rmse}{44}
\@gls@entry@count{mae}{36}
\@gls@entry@count{brk}{17}
\@gls@entry@count{gbed}{1}
\@gls@entry@count{mfz}{2}
\@gls@entry@count{bp}{3}
\@gls@entry@count{bpfz}{2}
\@gls@entry@count{knn}{2}
\@gls@entry@count{gbe}{59}
\@gls@entry@count{gbo}{31}
\@gls@entry@count{nbo}{1}
\@gls@entry@count{oslerp}{5}
\@gls@entry@count{loocv}{3}
\@gls@entry@count{kfcv}{1}
\@gls@entry@count{seo}{20}
\@gls@entry@count{idw}{26}
\@gls@entry@count{fic}{1}
\@gls@entry@count{svd}{14}
\@gls@entry@count{gbc}{4}
\@gls@entry@count{fz}{5}
\@gls@entry@count{vfz}{61}
\@gls@entry@count{vfzo}{128}
\@gls@entry@count{lobpcg}{2}
\@gls@entry@count{lkr}{3}
\@gls@entry@count{ms}{10}
\@LN@col{1}
\@LN@col{2}

\end{filecontents}
