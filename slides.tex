\section{Introduction} 
\subsection{Motivation}
\subsection{Prior Work}
\subsection{\glsentrytitlecase{vfzo}{long} Framework}
\section{Methods} 
\subsection{The \glsentrytitlecase{vfzo}{long} Framework}
%\subsubsection{Defining the \glsentrytitlecase{vfz}{long}}
%\subsubsection{Mapping \glsfmtshortpl{gb} to the \glsentrytitlecase{vfz}{long}}
%\subsubsection{Distance Calculations in the \glsentrytitlecase{vfz}{long}}
\subsection{Generating Random \glsentrytitlecase{vfzo}{long}s}
\subsection{Interpolation in the \glsentrytitlecase{vfzo}{long} Framework}
%\subsubsection{Barycentric Interpolation}
%\subsubsection{\glsentrytitlecase{gpr}{long}}
%\subsubsection{\glsentrytitlecase{idw}{long} Interpolation}
%\subsubsection{\glsentrytitlecase{nn}{long} Interpolation}
\subsection{Use of Interpolation Function}
\subsection{Literature Datasets}
%\subsubsection{\glsentrytitlecase{gpr}{long} for Fe Simulation Dataset}
%\subsubsection{\glsentrytitlecase{gpr}{long} for Ni Simulation Dataset}
%\subsubsection{\glsentrytitlecase{gprm}{long} for Fe Simulation Dataset}
\section{Results and Discussion} 
\subsection{Interpolation Accuracy}
%\subsubsection{Constant-Valued Control Models}
%\subsubsection{Experimental and Simulation Error}
%\subsubsection{Our Four Interpolation Methods}
%\subsubsection{Plotting \glsentrytitlecase{gbe}{long} between Arbitrary GBs}
\subsection{Interpolation Efficiency}
%\subsubsection{Comparison of Methods}
%\subsubsection{Symmetrization Runtime Comparison with Traditional Octonion Metric}
\subsection{Literature Datasets}
%\subsubsection{Comparison with Prior Work}
%\subsubsection{\glsentrytitlecase{gprm}{long} Applied to Fe Simulation Data}
\section{Conclusion} 
\section{Active vs. Passive Convention}
\section{Detailed Barycentric Interpolation Method}
\subsection{Triangulating a \glsentrytitlecase{vfz}{long} Mesh}
%\subsubsection{\glsentrytitlecase{svd}{long} Transformation from 8D Cartesian to 7D Cartesian}
%\subsubsection{Linearly Project onto Hyperplane}
%\subsubsection{\glsentrytitlecase{svd}{long} Transformation from 7D Cartesian to 6D Cartesian}
\subsection{Intersections in a \glsentrytitlecase{vfz}{long} Mesh}
%\subsubsection{Apply Same \glsentrytitlecase{svd}{long} to Input and Prediction Points}
%\subsubsection{Testing Nearby Facets for Intersections}
\subsection{Interpolation via Barycentric Coordinates}
