\begin{enumerate}
    \item First mesh-based property interpolation scheme that incorporates all five macroscopic crystallographic degrees of freedom.
    \item Past work predicting grain boundary energy
    \begin{enumerate}
        \item \gls{brk} function \cite{Bulatov2014GrainMetals}
        \item Rohrer \gls{gbed} based on binning in nickel \cite{Li2009RelativeNickel}, yttria \cite{Dillon2009CharacterizationFIB}, and copper \cite{Randle2008Five-parameterCopper}. Also \cite{Rohrer2010DerivingData}
        \item other five-parameter articles: titanium \cite{Farabi2018Five-parameterTitanium}, BCC materials \cite{Ratanaphan2015GrainMetals}.
        \item a non-discretizing \gls{knn} approach to determine GB energy \cite{Shen2019DeterminingSpace}.
        \item Neural network predictions \cite{EcheverriRestrepo2014UsingEnergies}
        \item grain boundary octonion theory \cite{Francis2019ABoundaries,Chesser2020LearningProperties}
         \begin{enumerate}
             \item Smooth interpolation between two arbitrary grain boundaries (oSLERP) \cite{Francis2019ABoundaries}
             \item inverse distance weighting approach \cite{Chesser2020LearningProperties}
         \end{enumerate}
    \end{enumerate}
    
    \item extension of octonion representation using barycentric coordinates
    \begin{enumerate}
        \item barycentric coordinates commonly used for interpolation within a simplex or other convex polygon.
        \begin{enumerate}
            \item basic equations (positivity, partition of unity, linear precision) \cite{Langer2006SphericalCoordinates}
        \end{enumerate}
        \item spherical barycentric coordinates that preserve linear precision \cite{Langer2006SphericalCoordinates}, which is important for interpolation, or preserve partition of unity \cite{Lei2020ASystems}, which is better for even subdivision of spherical surfaces.
        \item we incorporate linear precision preserving barycentric coordinates with an octonion representation of GBs to do property interpolation.
    \end{enumerate}
    \item application of this octonion framework using Gaussian process regression
    \begin{enumerate}
        \item Gaussian process regression is a machine learning technique geared towards predicting data (along with prediction intervals and standard deviations of the predictions) for arbitrary regions using a set of input data consisting of predictors and responses.
        \item By supplying a closed octonion mesh, we compute covariance matrices efficiently (i.e. arbitrary distance calculations are very quick), which is important for a Gaussian process regression.
    \end{enumerate}
\end{enumerate}

\section{Results and Discussion} \label{sec:resultsDiscussion}
\begin{enumerate}
    \item parity plots
    \begin{enumerate}
        \item 388 and 50,000 mesh points
        \item tile1 - spherical barycentric interpolation
        \item tile2 - planar barycentric interpolation
        \item tile3 - \gls{gpr}
        \item tile4 - pure \gls{nn} interpolation
    \end{enumerate}
    \begin{figure}
        \centering
        \includegraphics{}
        \caption{parity plots for 388 (red) and 50,000 (black) octonions formed via pairs of random cubochorically sampled quaternion and spherically sampled boundary plane normals. Property values sampled via \acrlong{brk} function and interpolation via barycentric (a), pure \acrlong{nn} (b), and \acrlong{gpr} (c).}
        \label{fig:brk-parity1}
    \end{figure}
    \begin{figure}
        \centering
        \includegraphics{}
        \caption{parity plots for 388 Olmsted \acrfullpl{gb} (red) and 50,000 Kim \acrshortpl{gb} (black) with property values sampled via \acrlong{brk} function and interpolation via barycentric (a), pure \acrlong{nn} (b), and \acrlong{gpr} (c).}
        \label{fig:brk-parity2}
    \end{figure}
    \item \Gls{rmse} vs. number mesh points for barycentric interpolation (black), \gls{gpr}, and pure \gls{nn} interpolation (red). Range from 1e2 to 1e5 mesh points
    \begin{figure}
        \centering
        \includegraphics{}
        \caption{\acrlong{rmse} vs. number of mesh points for spherical barycentric interpolation (black), planar barycentric interpolation, spherical \acrlong{gpr} (red), planar \gls{gpr}, pure \acrlong{nn} interpolation (orange), and average model with standard deviations from 10 random samples.}
        \label{fig:brk-rmse}
    \end{figure}
    \item histograms of \gls{nn} and next \gls{nn} distances (arc length) for mesh
    \begin{figure}
        \centering
        \includegraphics{}
        \caption{histograms of \acrfull{nn} (red) and next \acrlong{nn} (black) arc lengths for meshes of 388 (a), 10,000 (b), and 50,000 (c) octonions formed via pairs of random, cubochorically sampled quaternions.}
        \label{fig:nndist}
    \end{figure}
    \item timing and efficiency
    \begin{enumerate}
        \item fitrgp produces lower error in less time than barycentric interpolation even when using sparse approximation methods.
        \item barycentric interpolation is very fast once the triangulation is computed (i.e. still fast even if responses change, but needs to be recomputed if input data, i.e. predictors, change)
        \item fitrgp is fast and has lower error compared to barycentric interpolation; however, the entire process has to be rerun (in current implementation) if the responses or the predictors change.
    \end{enumerate}
    \item comparison with Chesser \cite{Chesser2020LearningProperties} (few points, Ni data, both high- and low-symmetry sets). Use 388 Ni bicrystals instead of \gls{brk} function. Barycentric and \gls{gpr}
    \begin{enumerate}
        \item table
        \begin{table}[]
            \centering
            \begin{tabular}{c c c c}
                \hline
                 Method & Minimizing Distance & \acrshort{rmse} & \acrshort{mae} \\
                 \hline
                 Barycentric & Euclidean Norm & & \\
                 \acrshort{nn} & Euclidean Norm & & \\
                 \acrshort{gpr} & Euclidean Norm & & \\
                 Pairwise-distance & Arc Length & &
            \end{tabular}
            \caption{Comparison of \acrfull{rmse} and \acrfull{mae} for closed-mesh barycentric interpolation, closed-mesh \acrfull{nn} interpolation, closed-mesh \acrfull{gpr}, and pairwise-distance inverse weighting for 388 Ni bicrystal simulations with 10-fold cross validation for each method.}
            \label{tab:chesser-comp}
        \end{table}
        \item parity plot
        \item \gls{5dof} visualizations
        \begin{figure}
            \centering
            \includegraphics{}
            \caption{\acrfull{gb} energy plotted in the $\Sigma{3}$, $\Sigma{5}$, $\Sigma{7}$, and $\Sigma{11}$ \acrlong{bp} spaces for closed-mesh spherical barycentric (a) and \acrlong{nn} (b) interpolation, closed-mesh \acrlong{gpr} (c), and pairwise-distance inverse weighting using 388 Ni bicrystal simulations.}
            \label{fig:chesser-5dof}
        \end{figure}
        \item 1DOF curves ([100] and [110] symmetric tilt boundaries ?)
    \end{enumerate}
    \item comparison with Restrepo \cite{EcheverriRestrepo2014UsingEnergies} (many points, Fe data, both high- and low-symmetry sets). Use simulated Fe data instead of \gls{brk} function. Barycentric and \gls{gpr}
    \begin{enumerate}
        \item table
        \begin{table}[]
            \centering
            \begin{tabular}{c c c c}
                \hline
                 Method & Minimizing Distance & \acrshort{rmse} & \acrshort{mae} \\
                 \hline
                 Barycentric & Euclidean Norm & & \\
                 \acrshort{nn} & Euclidean Norm & & \\
                 \acrshort{gpr} & Euclidean Norm & & \\
                 \acrshort{ann} & N/A & &
            \end{tabular}
            \caption{Comparison of \acrfull{rmse} and \acrfull{mae} for closed-mesh barycentric interpolation, closed-mesh \acrlong{gpr}, and a non-symmetry considering \acrfull{ann} using 50,000 Fe bicrystal simulations.}
            \label{tab:restrepo-comp}
        \end{table}
        \item parity plot
        \item \gls{5dof} visualizations
        \begin{figure}
            \centering
            \includegraphics{}
            \caption{\acrfull{gb} energy plotted in the $\Sigma{3}$, $\Sigma{5}$, $\Sigma{7}$, and $\Sigma{11}$ \acrlong{bp} spaces for closed-mesh spherical barycentric (a) and \acrlong{nn} (b) interpolation, closed-mesh \acrlong{gpr} (c), and a non-symmetry considering \acrlong{ann} (d) using 50,000 Fe bicrystal simulations.}
            \label{fig:restrepo-5dof}
        \end{figure}
        \item 1DOF curves ([100] and [110] symmetric tilt boundaries ?)
    \end{enumerate}
\end{enumerate}


\section{Methods} \label{sec:methods}

\begin{enumerate}
    \item barycentric interpolation
    \begin{enumerate}
        \item commonly used for interpolation within a simplex or other convex polygon.
        \begin{enumerate}
            \item basic equations (positivity, partition of unity, linear precision) \cite{Langer2006SphericalCoordinates}
        \end{enumerate}
        \item spherical barycentric coordinates that preserve linear precision \cite{Langer2006SphericalCoordinates}, which is important for interpolation, or preserve partition of unity \cite{Lei2020ASystems}, which is better for even subdivision of spherical surfaces.
    \end{enumerate}
    \item Gaussian process regression
    \begin{enumerate}
        \item MATLAB fitrgp
        \begin{enumerate}
            \item squared exponential covariance function
            \item bcd fit method (others are sd, fic etc.)
            \item constant basis function
            \item exact or bcd predict methods (also fic)
            \item quasi-newton optimizer
            \item optimize KernelScale and Sigma hyperparameters with bayesopt Optimizer
            \item ActiveSetMethod - entropy (others, log likelihood, sparse greedy approximation)
        \end{enumerate}
    \end{enumerate}
\item combining octonion representation with barycentric interpolation or Gaussian process regression
    \begin{enumerate}
        \item choosing a symmetrically equivalent representation based on minimum euclidean norm using a single reference octonion (i.e. kept constant)
        \begin{enumerate}
            \item contrast with traditional approach \cite{Francis2019ABoundaries}
            \begin{enumerate}
                \item traditional: allows either octonion to vary for efficiency; ours: one octonion held constant
                \item traditional: use arc length as minimizing metric; ours: use euclidean norm
                \item traditional: consider a subset of symmetrically equivalent GBOs; ours: consider all symmetrically equivalent GBOs
                \item traditional: always stay in 8D representation; ours: remove degenerate dimension (to 7D) for interpolation and employ projections (to 6D) for triangulation efficiency (for barycentric interpolation)
            \end{enumerate}
        \end{enumerate}
        \item triangulating a set of octonions using qhull (for barycentric interpolation)
        \item projections and rotations (projection to hyperplane, singular value decomposition)
        \item spherical barycentric coordinates \cite{Langer2006SphericalCoordinates}
        \begin{enumerate}
            \item equations
        \end{enumerate}
        \item barycentric interpolation
        \begin{enumerate}
            \item equations
            \item Determine intersecting facet using facets connected to \gls{nn}s and positivity constraint of planar barycentric coordinates
            \item intersecting facet vertices projected onto hyperplane tangent to hypersphere at datapoint
            \item containing facet and barycentric coordinates only need to be computed once for each datapoint, can be easily reused
        \end{enumerate}
    \end{enumerate}
    \item random cubochoric sampling \cite{Singh2016OrientationMethods}
    \item use of \gls{brk} function \cite{Bulatov2014GrainMetals}
    
    \item training, validation, and test partitioning \cite{Wang2020MachinePractices} for Kim data set \cite{Kim2011AnDatabase}
    \begin{enumerate}
        \item
    \end{enumerate}
    
\end{enumerate}


\section{Conclusion} \label{sec:conclusion}

\begin{enumerate}
    \item high fidelity interpolation approach is demonstrated
    \item approach is faster than pairwise distance matrices
    \item approach produces lower error than machine learning approach without consideration of symmetrically equivalent GBs
    \item lower error than pure \gls{nn} interpolation
    \item high intersection rate (~99\%) for barycentric interpolation
    \item approach is general to any crystal system and can be controlled by a single parameter (pgnum)
    \item future work
    \begin{enumerate}
        \item extension from linear interpolation to hyperspherical spline interpolation \cite{Taijeron1994SplineHyperspheres}
        \item interpolate within restricted regions of \gls{5dof} space
        \item define true FZ borders (high-symmetry GBs on perimeter)
        \item influence of choice of reference point
        \item influence of regularly spaced points vs. random sampling
        \item extension to generalized spherical barycentric coordinates (i.e. non-simplicial interpolation) \cite{Langer2006SphericalCoordinates}
        \item application to interpolating a GBCD (i.e. like fitting a curve to a histogram) and interpolating other GB properties (e.g. diffusivity, mobility)
        \item use of octonion facet vertices with other distance metrics (i.e. simplex reconstruction using edge lengths \cite{Connor2017High-dimensionalSearch})
        \item distance-based construction of submanifold \cite{Boissonnat2017OnlySubmanifolds}
        \item exploration of other parameters within \gls{gpr}
        \item exploration of machine learning methods other than \gls{gpr}
        \item linking with non-bicrystal data such as Rohrer 3D \gls{tj} data sets (e.g. Ni \cite{Li2009RelativeNickel})
    \end{enumerate}
\end{enumerate}


\section{Supplemental}
\begin{enumerate}
    \item conversion from octonion to \gls{5dof}
    \item high-aspect ratio facets
    \item facet subdivision
    \item spherical vs. planar barycentric interpolation
    \item tolerances of interpolation and intersecting facets
    \item non-intersection percentage vs. number of mesh points
    \item "excess" arc length in pairwise distance matrices
    \begin{enumerate}
        \item gradient optimization and global optimization of U(1) twist symmetry with marginal improvement
    \end{enumerate}
    \item identification and subdivision of hull exterior
    \item distribution of data in \gls{mfz} and \gls{bp} spaces
    \item comments on \gls{gpr} hyperparameters
\end{enumerate}