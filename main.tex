% start the document

% specify the document layout and font size
\documentclass[preprint,12pt]{elsarticle}
\usepackage[margin=1.5cm,includefoot]{geometry}
\usepackage{setspace}

% uploading packages
\usepackage{graphicx}
\usepackage{amssymb}
\usepackage{textcomp} % https://latex.org/forum/viewtopic.php?f=4&t=3364#p13124, https://tex.stackexchange.com/questions/165115
\usepackage{gensymb}
\usepackage{lineno}
\usepackage{mathtools}
\usepackage[title]{appendix}
\usepackage[separate-uncertainty=true]{siunitx}
\usepackage[colorlinks]{hyperref}
\usepackage[nameinlink,capitalise]{cleveref} %needs to appear after hyperref, https://tex.stackexchange.com/questions/396728/my-equations-referencing-not-working
\Crefname{figure}{Figure}{Figures} %needs to appear after hyperref and cleveref
\crefname{appsec}{Appendix}{Appendices}
\newcommand\crefrangeconjunction{--} % modify the reference style
\usepackage{mathrsfs}
\usepackage{url}
\usepackage{enumitem}
\usepackage{tabulary}
\usepackage{caption}
\usepackage{subcaption}
\usepackage{multirow}
\usepackage{makecell} % https://tex.stackexchange.com/questions/2441/how-to-add-a-forced-line-break-inside-a-table-cell
\newcommand{\NA}{---} % holds an m-dash
\graphicspath{{figures/}} %Setting the graphicspath
% ---------to deal with the double quotes----------- 
\usepackage [english]{babel}
\usepackage [autostyle, english = american]{csquotes}
\MakeOuterQuote{"}
%alternatively can use `` '' format for double quotes
\usepackage{booktabs}
\setlength{\abovetopsep}{1ex}
\usepackage[shortcuts,abbreviations]{glossaries-extra}
\newcommand*{\TCac}[1]{\ecapitalisewords{\glsentrylong{#1}}}

% remove the "Prepring submitted to Elsevier" footer on the first page
\makeatletter
\def\ps@pprintTitle{%
   \let\@oddhead\@empty
   \let\@evenhead\@empty
   \def\@oddfoot{\reset@font\hfil\thepage\hfil}
   \let\@evenfoot\@oddfoot
}
\makeatother

% figure info, etc. that can dynamically change (color of points, etc.)
\newcommand{\startpt}{red points}
\newcommand{\sympt}{dark blue points}
\newcommand{\refpt}{white circle}
\newcommand{\vbordercolor}{black}
\newcommand{\vcellcolor}{light blue}
\newcommand{\inpt}{input}
\newcommand{\outpt}{prediction}
\newcommand{\inptvar}{nmeshpts}


\glssetcategoryattribute{abbreviation}{indexonlyfirst}{true}
\glssetcategoryattribute{abbreviation}{nohyper}{true}
\makeglossaries

\newabbreviation{5dof}{5DOF}{five degree-of-freedom}
\newabbreviation{ebsd}{EBSD}{electron backscatter diffraction}
\newabbreviation[longplural={grain boundaries}]{gb}{GB}{grain boundary}
\newabbreviation{fcc}{FCC}{face-centered cubic}
\newabbreviation{mfc}{MFC}{mass flow controller}
\newabbreviation{sem}{SEM}{scanning electron microscope}
\newabbreviation{fea}{FEA}{finite element analysis}
\newabbreviation{bcs}{BCs}{boundary conditions}
\newabbreviation[longplural={triple junctions}]{tj}{TJ}{triple junction}
\newabbreviation{gpr}{GPR}{Gaussian process regression}
\newabbreviation{ann}{ANN}{artificial neural network}
\newabbreviation{nn}{NN}{nearest neighbor}
\newabbreviation{rmse}{RMSE}{root mean square error}
\newabbreviation{mae}{MAE}{mean absolute error}
\newabbreviation{brk}{BRK}{Bulatov Reed Kumar}
\newabbreviation{gbed}{GBED}{grain boundary energy distribution}
\newabbreviation{gbcd}{GBCD}{grain boundary character distribution}
\newabbreviation{mfz}{MFZ}{misorientation fundamental zone}
\newabbreviation{bp}{BP}{boundary plane}
\newabbreviation{knn}{kNN}{k-nearest neighbor}
\newabbreviation{gbe}{GBE}{grain boundary energy}
\newabbreviation{gbo}{GBO}{grain boundary octonion}
\newabbreviation{oslerp}{oSLERP}{octonion Spherical Linear Interpolation}
\newabbreviation{loocv}{LOOCV}{leave-one-out cross validation}
\newabbreviation{kfcv}{kFCV}{k-fold cross validation}
\newabbreviation{seo}{SEO}{symmetrically equivalent octonion}
\newabbreviation{fex}{FEX}{file exchange}
\newabbreviation{idw}{IDW}{inverse-distance weighting}
\newabbreviation{fic}{FIC}{fully independent conditional}
\newabbreviation{svd}{SVD}{singular value decomposition}
\newabbreviation{gbc}{GBC}{grain boundary character}
\newabbreviation{fz}{FZ}{fundamental zone}
% \newabbreviation{pfz}{pFZ}{pseudo fundamental zone} % pfz replaced by vfz
% \newabbreviation{cmo}{CMO}{closed-mesh octonion} % cmo replaced by vfzo
\newabbreviation{vfz}{VFZ}{Voronoi Fundamental Zone}
\newabbreviation{vfzo}{VFZO}{Voronoi Fundamental Zone octonion}
\newabbreviation{lobpcg}{LOBPCG}{locally optimal block preconditioned conjugate gradient}
\newabbreviation{lkr}{LKR}{Laplacian kernel regression}
% example abbreviations
% \newabbreviation{seo}{SEO}{symmetrically equivalent octonions}
%\newabbreviation[longplural={grain boundaries}]{gb}{GB}{grain boundary}

%example usage: \gls{gpr}
%example usage: \Gls{gpr} (capitalize first letter, only meaningful for first usage)
% \glspl{seo} --> symmetrically equivalent octonions OR SEOs
%^^^^^^^^^^^^^^^^^^^^^^^^^^^^^^^^^^^^^^^^^^^^^^^^^^^

\biboptions{sort&compress}

% Double Spacing
% \doublespacing
\begin{document}

\begin{frontmatter}

%\title{Grain Boundary Octonion Meshing and Interpolation}
\title{Five Degree-of-Freedom Property Interpolation of Arbitrary Grain Boundaries via \glsentrytitlecase{vfzo}{long} Framework}

\author[myu]{Sterling G. Baird}
\author[myu]{Eric R. Homer}
\author[myu]{David T. Fullwood}
\author[myu]{Oliver K. Johnson\corref{cor1}}
\ead{ojohnson@byu.edu}

\address[myu]{Department of Mechanical Engineering, Brigham Young University, Provo, UT 84602, USA}

\cortext[cor1]{Corresponding author.}

\date{October 2020}

\begin{abstract}
    In this work we introduce the \gls{vfzo} interpolation framework for \gls{gb} structure-property models and surrogates. The \gls{vfzo} framework offers an advantage over other \gls{5dof} based property interpolation methods because it is constructed as a \gls{vfz} point set in a Riemannian manifold, for which directly computed Euclidean and arc length distances significantly reduce computation time compared with other approaches. This increased efficiency facilitates the use of significantly more input data and therefore leads to a reduction of the interpolation error. As a demonstration, we present \gls{gbe} interpolation results for a non-smooth validation function (Bulatov, Reed, Kumar) \cite{bulatovGrainBoundaryEnergy2014}.
    %and simulated bi-crystal datasets from the literature for Ni and Fe.
    Four interpolation methods built on this framework --- barycentric interpolation, \gls{gpr} or Kriging, \gls{idw}, and \gls{nn} interpolation --- are presented and compared against a constant, average model (\gls{rmse} $\simeq$ \SI{0.13}{\J\per\square\meter}), resulting in \gls{rmse} values of 0.063, 0.056, 0.068, and \SI{0.082}{\J\per\square\meter}, respectively, for \num{50000} randomly sampled input \glspl{gb} and evaluated for \num{10000} randomly sampled \outpt{} \glspl{gb}. A vectorized, parallelized, MATLAB interpolation function is made available (\url{github.com/sgbaird-5dof/interp}). The \gls{vfzo} framework offers advantages for computing distances between GBs, estimating property values for arbitrary \glspl{gb}, and modeling surrogates of computationally expensive \gls{5dof} functions and simulations.
\end{abstract}

\begin{keyword}
Grain Boundary \sep Structure-Property Model \sep Interpolation \sep Octonion \sep Energy
\end{keyword}

% 2020-10-24  The closed-octonion interpolation framework offers an advantage over other five-degree-of-freedom based property interpolation methods because it's defined as a closed mesh in a Riemannian manifold. One can triangulate a mesh using standard routines (e.g. quickhull, qhull.org) and interpolate using barycentric coordinates or machine learning methods such as Gaussian Process Regression. Euclidean and arc length distances take on meaning in this framework and are trivial computations compared with other distance metrics, thereby addressing a limitation in previous work. The ability to use significantly more input data lends itself to lower interpolation error and is demonstrated by grain boundary energy interpolation results for a non-smooth validation function (Bulatov, Reed, Kumar) and simulated bi-crystal datasets from the literature for Ni and Fe. Four interpolation methods built on this framework --- barycentric interpolation, Gaussian Process Regression or Kriging, inverse-distance weighting, nearest neighbor interpolation --- are presented and compared against a constant, average model (RMSE = 0.013 \J\per\square\meter), resulting in RMSE values of 0.063, 0.056, 0.068, and 0.082 \J\per\square\meter, respectively, for \num{50000} randomly input sampled bicrystals and evaluated for \num{10000} randomly sampled output bicrystals. A vectorized, parallelized, MATLAB implementation is made available (github.com/sgbaird-5dof/interp) with similar input/output structure of built-in MATLAB interpolation functions (e.g. \textit{interpn}) and placeholders for custom interpolation schemes. The closed-octonion framework offers a great advantage in estimating property values for arbitrary grain boundaries based on experimental or simulated data and modeling surrogates of computationally expensive 5DOF functions and simulations.

\end{frontmatter}

\section{Introduction} \label{sec:intro}

In this work, we present a new method for interpolation and prediction of grain boundary (GB) properties from a set of measured/calculated values. Our approach, called the \gls{vfzo} framework is highly efficient, and thereby facilitates the use of large data sets to enhance prediction accuracy.

\subsection{Previous Work}
In previous work, a number of strategies have been developed for predicting \gls{5dof} \gls{gb} properties from experimental or simulated data. Binning and gradient descent were used to produce a \gls{5dof} \gls{gbed} in nickel \cite{liRelativeGrainBoundary2009}, yttria \cite{dillonCharacterizationGrainboundaryCharacter2009}, and copper \cite{randleFiveparameterGrainBoundary2008} based on experimentally characterized 3D microstructures. 

A non-discretizing approach was recently introduced \cite{shenDeterminingGrainBoundary2019} that utilizes regularization imposed on \gls{tj} equilibrium equations, the \gls{lobpcg} method, and \gls{knn} distances. In this approach, \num{60000} \glspl{tj} (\num{180000} \glspl{gb}) and a custom, non-smooth validation function were used to obtain \gls{gbe} \gls{rmse} values of \SI{0.0076}{\J\per\square\meter} and \SI{0.0277}{\J\per\square\meter} for \gls{gbe} values greater than \SI{0.9}{\J\per\square\meter} and less than \SI{0.9}{\J\per\square\meter}, respectively. 

\citet{restrepoUsingArtificialNeural2014} used an \gls{ann} and approximately \num{17000} and \num{51000} Fe bicrystal simulations from \citet{kimIdentificationSchemeGrain2011} as training and validation data, respectively, to achieve \glspl{mae} of \SI{0.0486}{\J\per\square\meter} and approximately \SI{0.09}{\J\per\square\meter} in the best fitted \glspl{ann} for randomly selected and special \glspl{gb}, respectively. 

Recently, a new \gls{gb} representation termed \glspl{gbo} was reported \cite{francisGeodesicOctonionMetric2019} and tested \cite{chesserLearningGrainBoundary2020}. The \gls{gbo} representation is valuable for a number of applications. Most relevant to the present work is the resulting distance metric. The \gls{gbo} distance metric offers an advantage over other metrics in that it "correctly determines the angular distances between \glspl{gb} with a common normal or misorientation" and "closely approximates the geodesic metric on $SO(3) \times SO(3)$ \textit{for all grain boundary pairs} while maintaining the ability to be analytically minimized with respect to the $U(1)$ symmetry" \cite{francisGeodesicOctonionMetric2019}. In this context, \citet{francisGeodesicOctonionMetric2019} derived \gls{oslerp} and provided examples showing that \gls{oslerp} produces smooth, minimum distance paths through \gls{gb} character space between two arbitrary \glspl{gb}. 

\Gls{lkr} (a type of \gls{idw}) involving scaled pairwise distance matrices was later used with \glspl{gbo} to predict properties of arbitrary \glspl{gb} from a set of known values \cite{chesserLearningGrainBoundary2020}. Using \gls{kfcv} with $k=10$ for \num{388} Ni \gls{gbe} simulations \cite{olmstedSurveyComputedGrain2009a} and an optimized scaling parameter, an \gls{rmse} of \SI{0.0977}{\J\per\square\meter} was obtained. Due to computation time of pairwise distance matrices, this approach is currently "limited to datasets with several thousand or fewer" \glspl{gb} \cite{chesserLearningGrainBoundary2020}.

\subsection{\glsentrytitlecase{vfzo}{long} Framework}
The \gls{vfzo} interpolation framework introduced in this work offers an advantage over other methods because it is defined as a \gls{vfz} point set in a Riemannian manifold. This is evidenced by the ability to triangulate a mesh using standard routines (e.g. quickhull \cite{barberQuickhullAlgorithmConvex1996}) and interpolate using barycentric coordinates or machine learning methods such as \gls{gpr}. Building on previous work on \glspl{gbo} \cite{francisGeodesicOctonionMetric2019,chesserLearningGrainBoundary2020}, we create a \gls{vfz} point set by obtaining a set of octonions minimized with respect to Euclidean distance and an arbitrary reference octonion after considering all \glspl{seo}. Because \glspl{gbo} are guaranteed to reside on the surface of a hypersphere \cite{francisGeodesicOctonionMetric2019} (a type of Riemannian manifold) a point set which locally resembles Euclidean space is the result (\cref{sec:methods:vfz-dist}). Below we mention a few benefits and applications of this approach, after which we provide the detailed description of the method (\cref{sec:methods}) and numerical test results (\cref{sec:results}).

\subsection{Benefits of \glsentrytitlecase{vfzo}{long} Framework}
\subsubsection{Distance Calculations}
Directly computed, scaled Euclidean and arc length distances in the \gls{vfzo} framework approximate the original octonion distance by \citet{francisGeodesicOctonionMetric2019}, and the calculation speed is even higher than explicit \gls{gbo} distance calculations using the original octonion distance.. For example, \num{50000} \glspl{gbo} can by symmetrized into \glspl{vfzo} in approximately \SI{60}{\second} using 6 cores (e.g. via \texttt{get\_octpairs(get\_ocubo(50000))}), and the corresponding \num{50000} $\times$ \num{50000} pairwise-distance matrix can be computed in approximately \SI{10}{\second} using built-in MATLAB \textit{Statistics and Machine Learning Toolbox} function \texttt{pdist}, giving a total runtime of approximately \SI{70}{\second}. This was computationally challenging before and a limitation of previous work. Compared to original octonion metric distance calculations \cite{chesserLearningGrainBoundary2020}, this represents an improvement in computational speed by five orders of magnitude.\footnote{Improvement per distance calculation per core is about \num{5e5} relative to the EMSoft \cite{degraefEMSoft2020} metric of 26 minutes using 8 cores for a 388 x 388 pairwise distance matrix. This EMSoft timing information is directly reported in \cite{chesserLearningGrainBoundary2020}.}

This significant speed up stems from the fact that in the \gls{vfzo} framework \glspl{seo} only need to be considered once per \gls{gb}, $O(L)$, rather than once per distance calculation, $O(L^2)$,
%per \gls{gb} in a \gls{gb}-pair
and that \glspl{seo} only need to be considered once in a \gls{gb} pair, $O(N_p^2)$, rather than for every combination between the two \glspl{gb} ($O(N_p^4)$). The \gls{seo} computation complexity is thus $O(N_p^2L)$, a significant improvement compared with the original \gls{seo} complexity of $O(N_p^4L^2)$ \cite{chesserLearningGrainBoundary2020}, where $N_p$ is the symmetry cardinality ($N_p=24$ for $m\Bar{3}m$ \gls{fcc} point group) and $L$ is the number of \glspl{gb}. Empirically, the full $O(Np^2L)$ operations take about $60*6 = \SI{360}{\second}$ for \num{50000} \glspl{vfzo}, whereas the final pairwise-distance computation is $O_{\text{pd}}(L^2)$ and takes approximately \SI{10}{\second} for a \num{50000} $\times$ \num{50000} matrix, despite $Np^2L\ll L^2$. In other words, $O(Np^2L)+O_\text{pd}(L^2)\approx O(Np^2L)$. This is because the operation of considering an \gls{seo} is computationally much more expensive than the direct (Euclidean) pairwise distance computation in this work. Because Euclidean distances are employed, which can be computed faster than trigonometric inverse functions, and built-in, vectorized MATLAB functions are utilized where possible, there is a further speed enhancement in the \gls{vfzo} approach. % $O(N^2L)$ (this work) vs. $O(N^4L^2$ (octonion paper)

%\footnote{\texttt{GBdist.m} in \cite{chesserGBOctonionCode2019} uses the assumption that if a minimum angle is repeated 9 times, the symmetrization loop is exited, whereas the \texttt{GBmod.f90} implementation in EMSoft \cite{degraefEMSoft2020} does not appear to use this assumption.}

\subsubsection{Interpolation Error} \label{sec:intro:interp-error}
The ability to use significantly more input data lends itself to lower interpolation error. This will be demonstrated by \gls{gbe} interpolation results for a non-smooth validation function in \cref{sec:results}.
%and simulated bi-crystal datasets from the literature (388 Olmsted Ni \cite{olmstedSurveyComputedGrain2009} and \num{50000} Kim Fe \cite{kimIdentificationSchemeGrain2011} bicrystals).
Four interpolation methods built on this framework --- barycentric (\cref{sec:methods:interp:bary}), \gls{gpr} or Kriging (\cref{sec:methods:interp:gpr}), \gls{idw} (\cref{sec:methods:interp:idw}), \gls{nn} (\cref{sec:methods:interp:nn}) --- will be presented and compared against a constant, average model (\cref{sec:results:accuracy}). To facilitate easy application of the presented method, a vectorized, parallelized, MATLAB implementation, \texttt{interp5DOF()}, is made available \cite{bairdFiveDegreeofFreedom5DOF2020} with similar input/output structure to that of built-in MATLAB interpolation functions (e.g. \texttt{scatteredInterpolant(), griddatan()}) and placeholders for custom interpolation schemes. A typical function call is as follows: \texttt{ypred = interp5DOF(qm,nA,propList,qm2,nA2,method)}. The argument \texttt{propList} is a vector of known property values corresponding to the GBs defined by (\texttt{qm},\texttt{nA}), which respectively denote pairs of GB misorientation quaternions and \gls{bp} normals. The result, \texttt{ypred}, is a vector of predicted/interpolated property values corresponding to the query GBs defined by (\texttt{qm2},\texttt{nA2}). Internally, these are converted to octonions and interpolation is performed using the selected \texttt{method}. The methods used in this work are \texttt{'pbary'}, \texttt{'gpr'}, \texttt{'idw'}, and \texttt{'nn'}, corresponding to planar barycentric, \gls{gpr}, \gls{idw}, and \gls{nn} interpolation, respectively. A placeholder template with instructions for implementing additional interpolation schemes is also provided in \texttt{interp5DOF()}. Misorientation quaternions are represented in the active sense\footnote{The passive convention is used in \cite{francisGeodesicOctonionMetric2019}}:
\begin{equation}
    q_m = {q_A}^{-1}q_B
\end{equation}
where $q_m$, $q_A$, and $q_B$ represent the misorientation quaternion, orientation quaternion of grain A in the sample frame, and orientation quaternion of grain B in the sample frame, respectively. The $^{-1}$ operator denotes a quaternion inverse. \Gls{bp} unit normals are expressed pointing away from grain A and in the reference frame of grain A (i.e. the outward-pointing normal convention). See \cite{francisGeodesicOctonionMetric2019} for a treatment of conversions to octonion coordinates.

\subsubsection{Applications}

The \gls{vfzo} framework offers a great advantage in estimating property values for arbitrary \glspl{gb} based on experimental or simulated data such as energy, mobility, and diffusivity. The framework can also enable efficient surrogate modeling of computationally expensive 5DOF functions and simulations such as in evaluation of the \gls{brk} function for use in anisotropic grain growth simulations. In other words, one can evaluate the \gls{vfz} surrogate model in a fraction of the time of the true property model, thereby facilitating larger scale iterative simulations, which require repetitive evaluation of a computationally expensive structure-property model.

% We think it is straightforward to apply these methods to full or restricted regions of \gls{5dof} space for \gls{gb} models and surrogates. In addition to interpolations involving \gls{gbe}, \gls{gbcd} (i.e. similar to fitting a curve to a histogram) and other \gls{gb} properties such as diffusivity and mobility could also be interpolated. Finally, linking this interpolation scheme with polycrystal data such as 3D microstructural \gls{tj} datasets (e.g. Ni \cite{liRelativeGrainBoundary2009}) via \textit{TJ2GBE} \cite{shenDeterminingGrainBoundary2019} or similar approaches geared towards large datasets is especially promising.

\section{Methods} \label{sec:methods}

The core operations of the \gls{vfzo} framework are (i) the generation of, and (ii) mapping of points into, a \gls{vfz}, and (iii) distance calculations within the \gls{vfz}. Rather than relying on an analytically defined \gls{fz}, we employ a numerical strategy to define what we will refer to as a \gls{vfz}.

The construction of the \gls{vfz} dramatically reduces the computational burden of pairwise distance calculations. The mechanism by which this reduction is achieved can best be illustrated with an example. Let $o_1$ and $o_2$ denote two \glspl{gb} represented as octonions (\glspl{gbo}). 
To perform a traditional distance calculation it is necessary to compare all \glspl{seo} of $o_1$ to all of the \glspl{seo} of $o_2$ and take the smallest distance. If $N_p$ is the cardinality of the crystallographic point group, this single minimum distance calculation requires a total of $N_p^4$ \glspl{seo} to be considered. If one desires to compute a pairwise distance matrix between $L$ GBs, all \glspl{seo} of each GB are compared against all \glspl{seo} of all others so that the total number of \gls{seo} computations will be $O(N_p^4L^2)$.

In contrast, for a single distance calculation using the \gls{vfzo} framework, $o_1$ and $o_2$ are first mapped into the \gls{vfz}, and then only a single distance calculation is required between them. Mapping $o_1$ into the \gls{vfz} requires comparison of all $N_p^2$ \glspl{seo} of $o_1$ with a fixed reference \gls{gb} in the interior of the \gls{vfz}; and likewise for $o_2$. Consequently, a single distance calculation between $o_1$ and $o_2$ under the \gls{vfzo} framework requires $2N_p^2$ \gls{seo} computations. If one desires to compute a pairwise distance matrix between $L$ \glspl{gb}, the total computational cost will be $O(N_p^2L)$, which represents a dramatic reduction compared to the traditional approach (\cref{fig:runtime}). A summary of the differences between the two approaches is provided in \cref{tab:closed-mesh-comparison}.

\begin{table}
\caption{Comparison between \acrlong{vfzo} and traditional octonion frameworks. *6D Cartesian representation used only for mesh triangulation efficiency in barycentric interpolation and *7D Cartesian representation only required for barycentric interpolation. For pairwise distance complexity, $N_p$ is the symmetry cardinality ($N_p=24$ for $m\Bar{3}m$ \gls{fcc} point group) and $L$ is the number of \glspl{gb}.}
\centering
\begin{tabular}{ccc}
\toprule
Property & Traditional & This Work \\
\midrule
Symmetrizing Distance & Octonion & Euclidean \\
% Considered \glspl{seo} & Subset & All \\
Dimensionality & 8D Cartesian & 6*/7*/8D Cartesian \\
Bounded by \Gls{fz} & No & Yes \\
% Euclidean Approximation Valid & No & Yes \\
Pairwise Distance Complexity & $O(N_p^4L^2)$ & $O(N_p^2L)$ \\
\bottomrule
\end{tabular}
\label{tab:closed-mesh-comparison}
\end{table}

In this section we describe our methods for each of the \gls{vfzo} framework operations (i)-(iii) named above. We then describe our implementation of 4 different interpolation schemes for prediction of \gls{gb} properties based on the \gls{vfzo} approach.

\subsection{The \glsentrytitlecase{vfzo}{long} Framework}
\label{sec:methods:framework}

\subsubsection{Defining the \glsentrytitlecase{vfz}{long}}
\label{sec:methods:vfz}

% Explain the "voronoi" type definition of the vfz by choosing a reference GB and then taking the symmetric image of each GB that is closest to that reference GB. You will need to convince the reader (probably by reason, or proof, or numerical tests) that the resulting vfz does in fact contain one and ONLY one SEO for every GB (i.e. that it does satisfy the definition of a FZ). This is probably where you note that for it to work, the reference GB needs to be something that isn't a high-symmetry GB, and why. This discussion will probably benefit from a figure illustrating the procedure in 2D.

To define a \gls{vfz}, an arbitrary, fixed, low-symmetry reference \gls{gbo} is chosen ($o_{\text{ref}}$) and the \gls{vfz} is formally defined as the region of the unit 8-sphere closer to $o_{\text{ref}}$ than any of its symmetric images. However, use of the \gls{vfz} does not require its explicit construction. Rather, practical calculations require only the selection of the single point $o_{\text{ref}}$ (which completes the definition of the \gls{vfz}), followed by mapping of query points into the \gls{vfz} by comparison of their \glspl{seo} with $o_{\text{ref}}$.

To illustrate the process of mapping points into the \gls{vfz}, we describe a 3D Cartesian analogue (\cref{fig:voronoi}) to a 7D Cartesian non-degenerate (i.e. U(1) degeneracy removed) representation of a \gls{vfz}. A set of \num{500} points ($p_i, i\in[1,500]$) randomly scattered on the surface of the 2-sphere comprise the data (\startpt{}). A random point, $p_{\text{ref}}$, also on the surface of the 2-sphere, is chosen as the reference point (\refpt{}). In this illustration, $O_h$ or $m\bar{3}m$ point group rotations are used as symmetry operators, $S_j,j\in[1,N_p]$, where $N_p$ is the cardinality as before and $N_p = 24$ for the $O_h$  point group. For each data point, \num{24} symmetrically equivalent representations ($p^{\text{sym}}_{i,k},k\in[1,24]$) are produced by applying the appropriate rotation transformations ($S_j(p_i),j\in[1,24]$). After calculating the Euclidean distance between $p_{\text{ref}}$ and $p^{\text{sym}}_{i,k},k\in[1,24]$, the point ($p^{NN}_i,i\in[1,500]$) closest to $p_{\text{ref}}$ is chosen and retained as the unique representative of $p^{\text{sym}}_{i,k}$. As illustrated in \cref{fig:voronoi}(a), the projected points $p^{NN}_i$ (dark blue points) all fall in the \gls{vfz} without ever having to construct or define it explicitly, we call this group of projected points a \textit{\gls{vfz} point set}. Note also that there is only one $p^{NN}_i$ in \gls{vfz} for each $p^{\text{sym}}_{i,k}$ (see \cref{fig:voronoi}(b)).
\begin{figure}
    \centering
    \includegraphics[scale=1]{voronoi.png}
    \caption{3D Cartesian analogue to a non-degenerate 7D Cartesian representation of \glspl{gbo} and \glspl{vfzo} demonstrates the symmetrization of many points relative to a fixed, reference point (\refpt{}), leading to a 3D Cartesian \gls{vfz} point set (\sympt{}). A single input point (\startpt{}) is symmetrized (\sympt{}) relative to a fixed, reference point (\refpt{}) (b) demonstrating that only one symmetrized point is found within the borders (\vbordercolor{}) of each of the Voronoi cells (\vcellcolor{}). The Voronoi tessellation is defined by the symmetric images of the reference point, and the spherical Voronoi diagram for this illustration is constructed using a modified version of \cite{luongVoronoiSphere2020}.}
    \label{fig:voronoi}
\end{figure}

The minimum Euclidean distance \gls{seo} will be the same as the minimum arc length distance \gls{seo} because arc length ($s(x)$) is a monotonically increasing function of Euclidean distance ($x$) in $s(x)\in[0,\pi]$ (\cref{fig:dist-parity}). Without loss of generality, $s(2)=\pi$ gives the maximum arc length that can be obtained from two points on a unit-sphere and $2$ is the maximum Euclidean distance (diameter of a unit-sphere).

% The expectation that a single, unique \gls{seo} will be found (within numerical tolerance and given a low-symmetry reference \gls{gbo}) is verified by several manual tests and internally within the symmetrization sub-routine \texttt{get\_octpairs()} \cite{bairdFiveDegreeofFreedom5DOF2020} that is part of the \texttt{interp5DOF()} package. Similar numerical tests reveal that inappropriately selecting a high-symmetry reference \gls{gbo} results in many degenerate minimum distance \glspl{seo}, with the identity octonion ($\{1,0,0,0,0,0,0,0\}\in\mathbb{R}^8$) \cite{francisGeodesicOctonionMetric2019} giving the highest degeneracy.

\subsubsection{Mapping \glsfmtshortpl{gb} to the \glsentrytitlecase{vfz}{long}}
\label{sec:methods:proj}

% With the reference GB chosen, and consequently the vfz defined, explain that a GB is mapped into the vfz simply by finding among its \glspl{seo} the one that is closest to the reference GB.

As described above in the 3D analogy, with a reference \gls{gbo} chosen ($o_{\text{ref}}$), and consequently the \gls{vfz} defined (\cref{sec:methods:vfz}), a \gls{gb} is mapped into the \gls{vfz} by finding among its \glspl{seo} the one that is closest to $o_{\text{ref}}$. This is performed for all \inpt{} and \outpt{} points w.r.t. $o_{\text{ref}}$ resulting in a \gls{vfz} point set.

% In the \gls{vfzo} framework, \glspl{seo} are chosen based on minimum Euclidean distance relative to a fixed, arbitrary reference octonion rather than allowing either octonion in a pair to vary in the traditional octonion approach with arc-length distances \cite{francisGeodesicOctonionMetric2019}. We consider all symmetrically equivalent \glspl{gbo} rather than a subset. For barycentric interpolation (\cref{sec:methods:bary}), we also remove a degenerate dimension via a rigid \gls{svd} transformation to 7D Cartesian and employ a further projection to 6D Cartesian only for the triangulation. These differences are summarized in Table \cref{tab:closed-mesh-comparison}. By choosing \glspl{seo} based on Euclidean distance with respect to a fixed octonion, a "closed-mesh" is obtained in the sense that directly computed Euclidean and arc length distances become meaningful and a unique octonion is found within numerical tolerance as long as the reference \gls{gbo} is low-symmetry (very likely based on random sampling and can be verified). This facilitates the use of standard triangulation and interpolation routines rather than needing to rely on pairwise-distance matrices where each distance calculation requires consideration of \glspl{seo}.

% \subsubsection{Generating Random \glsentrytitlecase{vfzo}{long}s} \label{sec:methods:rand} First, random \glspl{gbo} are formed by taking a random, cubochorically sampled quaternion \cite{singhOrientationSamplingDictionarybased2016} \texttt{qm} and random \gls{bp} unit vector \texttt{nA} pair and converting these to an octonion representation \texttt{o} using a modified version \cite{bairdFiveDegreeofFreedom5DOF2020} of \texttt{GBfive2oct()} \cite{chesserGBOctonionCode2019} via \texttt{o=GBfive2oct(qm,nA)}. The conventions used for \texttt{qm} and \texttt{nA} are given in \cref{sec:intro:interp-error}. After mapping random \glspl{gbo} (\cref{sec:methods:rand}) into the \gls{vfz} (\cref{sec:methods:proj}) to obtain \glspl{vfzo} %

\subsubsection{Distance Calculations in the \glsentrytitlecase{vfz}{long}}
\label{sec:methods:vfz-dist}

% Explain how to compute distances in the vfz, and why you can now just use euclidean distances.
Euclidean distances are an accurate approximation of arc length distances in a \gls{vfz} because the difference between the two metrics for the maximum pairwise distance ($pd_{max} = ~\SI{60}{\degree}$) in a \gls{vfz} is small as shown in \cref{fig:dist-parity}. However, when compared with the traditional octonion distance \cite{francisGeodesicOctonionMetric2019}, due to the presence of low-symmetry \glspl{gb} near the exterior of a \gls{vfz}, some \glspl{gb} pairs will exhibit larger Euclidean or arc length distances than is truly representative (\cref{fig:dist-ensemble-k1-2-5-10}a). In other words, moving "past" the low-symmetry border of a \gls{vfz} will result in an instantaneous relocation to a possibly distant point in the \gls{vfz} that in reality is highly correlated. Future work involving the use of ensemble \gls{vfzo} sets with the same \glspl{gb} but different reference octonions can address this issue. We have demonstrated that a low number of ensemble \gls{vfzo} sets results in significant improvement to the Euclidean distance approximation (\cref{fig:dist-ensemble-k1-2-5-10},\cref{fig:dist-ensemble-rmse-mae}) of the traditional octonion metric.

\begin{figure}
\centering
\includegraphics[scale=1]{dist-parity.png}
\caption{Parity plot of 8D Cartesian hyperspherical arc length vs. 8D Cartesian Euclidean distance for pairwise distances in a symmetrized ($m\bar{3}m$), \gls{vfzo} set of \num{10000} randomly sampled \glspl{gbo}. The max arc length is approximately \SI{0.58}{\radian}, indicating a max octonion distance of approximately \SI{1.16}{\radian} or \SI{66.5}{\degree} between any two points in the \gls{vfz}. The close correlation between arc length and Euclidean distance supports the validity of using Euclidean distance instead of arc length in the interpolation methods. This is \textit{separate} from the correlation between \gls{vfzo} Euclidean or arc length distances with the traditional octonion distance.}
\label{fig:dist-parity}
\end{figure}

This is a limitation of the \gls{vfzo} framework which generates a \gls{vfz} with low-symmetry \glspl{gb} at the borders in contrast to typical \glspl{fz} \cite{patalaSymmetriesRepresentationGrain2013,homerGrainBoundaryPlane2015}. While defining a \gls{fz} with high-symmetry \glspl{gb} at the borders will certainly increase interpolation accuracy, the favorable interpolation results presented in this work are obtained because overestimation is infrequent within a small correlation length (e.g. \SI{10}{\degree} \cite{olmstedSurveyComputedGrain2009}, for which many \glspl{nn} fall within for a \num{50000} \gls{vfzo} set (\cref{fig:nnhist-knn-50000}b).

Overestimation imposes an arbitrary "sparseness" of data within a local region of influence common to the interpolation methods in this work, whereas underestimation would give erroneous high correlations between uncorrelated \glspl{gb}. Because only overestimation relative to traditional octonion distances exist in this work (\cref{fig:dist-ensemble-k1-2-5-10}), it is expected that large errors will occur infrequently, as shown in \cref{fig:brkparity50000}, and that overall errors will remain low as shown in \cref{fig:brkerror}, and \cref{tab:rmse-error-comparison,tab:mae-error-comparison}. We find that taking the minimum distance among several \gls{vfzo} sets defined by separate reference octonions leads to better correlation between the Euclidean approximation and the traditional octonion metric (\cref{fig:dist-ensemble-k1-2-5-10}). Additionally, \cref{fig:dist-ensemble-rmse-mae}) shows that the error decreases rapidly as the number of ensemble \gls{vfzo} sets increases. In future work, we plan to explore the incorporation of ensemble \gls{vfzo} sets to interpolation methods and expect increased accuracy at a reasonable computational cost.

\begin{figure}
    \centering
    \includegraphics[scale=1]{figures/dist-ensemble-k1-2-5-10.png}
    \caption{Hexagonally binned parity plots of pairwise distances of 388 Ni bicrystals \cite{olmstedSurveyComputedGrain2009a}. Euclidean distance approximation is converted to octonions ($x_{i,j,k}=2\left(\frac{180}{\pi}\right)|\hat{o}_{i,k}^{\text{sym}}-\hat{o}_{j,k}^{\text{sym}}|$) for comparison with the traditional octonion metric \cite{chesserLearningGrainBoundary2020}. The minimum distance among an ensemble of \gls{vfzo} sets ($\min_{\forall k \in [1,k_{max}]}x_{i,j,k}$) is used for (a) 1, (b) 2, (c) 5, and (d) 10 \gls{vfzo} sets. As the number of \gls{vfzo} sets increases, the correlation between the Euclidean distance and the traditional octonion distance improves.}
    \label{fig:dist-ensemble-k1-2-5-10}
\end{figure}

\begin{figure}
    \centering
    \includegraphics[scale=1]{figures/dist-ensemble-rmse-mae.png}
    \caption{\Gls{rmse} and \gls{mae} of pairwise distance errors for 388 Ni bicrystals \cite{olmstedSurveyComputedGrain2009a} of scaled Euclidean distance approximation relative to the traditional octonion metric \cite{chesserLearningGrainBoundary2020}. The minimum distance among an ensemble of \gls{vfzo} sets ($\min_{\forall k \in [1,k_{max}]}x_{i,j,k}$, where $x_{i,j,k}$ is the scaled Euclidean distance) is taken, iteratively adding consecutive sets up to $k_{max} = 10$. As the number of \gls{vfzo} sets increases, \gls{rmse} and \gls{mae} between the scaled Euclidean distance approximation and the traditional octonion distance decreases.}
    \label{fig:dist-ensemble-rmse-mae}
\end{figure}

\subsection{Generating Random \glsentrytitlecase{vfzo}{long}s}
\label{sec:methods:rand}
In addition to the 3 core operations of the \gls{vfzo} framework described in \cref{sec:methods:framework}, it will be necessary for our tests, and useful for other applications, to be able to generate points in the \gls{vfz}. We briefly explain here our process for accomplishing this. 

First, random \glspl{gbo} are formed by taking random, cubochorically sampled \cite{singhOrientationSamplingDictionarybased2016}, quaternion \texttt{qm} (via \texttt{get\_cubo()}) and random \gls{bp} unit vector \texttt{nA} (uniformly sampled in $\mathbb{R}^3 \in [-0.5,0.5]$ and normalized via \texttt{normr(rand(1,3)-0.5)}) pairs. These qm/nA pairs are then converted to an octonion representation \texttt{o} using a modified version \cite{bairdFiveDegreeofFreedom5DOF2020} of \texttt{GBfive2oct()} \cite{chesserGBOctonionCode2019} via \texttt{o=GBfive2oct(qm,nA)} and symmetrized via using \texttt{get\_octpairs()} (\cref{sec:methods:proj}) via \texttt{osym=get\_octpairs(o)} for a default reference octonion (this work, except for $k\in[2,10]$ in \cref{fig:dist-ensemble-k1-2-5-10} and \cref{fig:dist-ensemble-rmse-mae}) or by using a random reference octonion via \texttt{osym=get\_octpairs(o,'oref',get\_ocubo())}. The conventions used for \texttt{qm} and \texttt{nA} are given in \cref{sec:intro:interp-error}.

\subsection{Interpolation in the \glsentrytitlecase{vfzo}{long} Framework}
\label{sec:methods:interp}

With the \gls{vfzo} framework established, it is possible to define interpolation schemes over the \gls{vfz} to predict the properties of new \glspl{gb} from the known properties of other \glspl{gb}. For one application of interest to us, it is necessary to evaluate multiple different functions over a fixed set of \inpt{} and \outpt{} \glspl{gbo}. In this section we first present a barycentric interpolation method that we have developed to efficiently accomplish this task by pre-computing the interpolation weights (which remain fixed only when the function being evaluated changes). We then present adaptations of three other interpolation methods (\gls{gpr}, \gls{idw}, and \gls{nn}) that are useful for other applications. We recommend the \gls{gpr} interpolation method for the \gls{vfzo} framework for most applications because it provides the best combination of accuracy and speed (\cref{sec:results}).

\subsubsection{Barycentric Interpolation}
\label{sec:methods:interp:bary}

% Give brief explanation of standard barycentric interpolation: you find which simplex the point is in, then you calculate the weights. The advantage of this approach is that the weight matrix only has to be computed once. If you interpolate different functions over the same query points, the weights don't change, so re-evaluation with a new function is rapid via a simple matrix multiplication.

% Explain how to adapt barycentric interpolation to the vfz: how to identify which simplex the query point falls in, and then how to compute the weights. Also explain and justify why you can use euclidean distances (i.e. because the vfz occupies a small portion of the 7-sphere it is nearly planar so euclidean distances and octonion distances are approximately equal and you can show tha figure here), and how that speeds things up.

Barycentric coordinates are a type of homogeneous coordinate system that reference a \outpt{} point within a simplex \cite{langerSphericalBarycentricCoordinates2006} or convex polytope \cite{floaterGeneralizedBarycentricCoordinates2015,meyerGeneralizedBarycentricCoordinates2002,langerSphericalBarycentricCoordinates2006} based on "masses" or weights at the vertices. The \outpt{} point is assumed to be the barycenter (center of mass) of the simplex or convex polytope, and weights at the vertices necessary to make this assumption true are determined. We utilize rigid \gls{svd} transformations and a standard triangulation algorithm to define a simplicial mesh (\cref{app:bary-tri}). We then use barycentric weights (i.e. coordinates) for computing intersections of a point within a simplicial facet (\cref{app:bary-int}) and for interpolation (\cref{app:bary-interp}) \cite{langerSphericalBarycentricCoordinates2006}. A detailed explanation of the process is provided in \cref{sec:app}. The barycentric interpolation method is invoked in \texttt{interp5DOF()} by setting the \texttt{method} argument to \texttt{'pbary'}.

\subsubsection{\glsentrytitlecase{gpr}{long}}
\label{sec:methods:interp:gpr}

% You don't need to give a detailed explanation of GPR, but you do need 1 sentence summarizing the idea and then point to REF for a general treatment. Explain any assumptions or adaptations that need to be done to use it with the vfz (e.g. presumably GPR is not actually operating on the surface of the 7-sphere, rather it is actually operating on all of $\mathbb{R}^8$, but because the vfz occupies a small portion of the 7-sphere it is nearly planar so euclidean distances and octonion distances are approximately equal, as mentioned previously, so it works without modification). Finally just explain that we used matlab's built-in \texttt{fitrgp()} function and the options used like you already have written.

\Gls{gpr} or kriging uses the notion of similarity between points to fit Gaussian processes (random variables) to data based on prior information and provides uncertainty information in addition to interpolated or inferred values. For a general treatment of \gls{gpr}, see \cite{rasmussenGaussianProcessesMachine2006}. We use MATLAB's built-in function, \texttt{fitrgp()}, with all default parameters (MATLAB R2020b) except that \texttt{PredictMethod = 'exact'} regardless of the number of \inpt{} points, and we assume a Euclidean approximation of the \gls{vfz} (see \cref{sec:methods:vfz-dist}, and \cref{fig:dist-parity}). A faster, less memory-intensive \gls{fic} approximation is also available (\texttt{PredictMethod = 'fic'}). The \gls{gpr} interpolation method is invoked in \texttt{interp5DOF()} by setting the \texttt{method} argument to \texttt{'gpr'}.

\subsubsection{\glsentrytitlecase{idw}{long} Interpolation}
\label{sec:methods:interp:idw}

% Explain the idea behind IDW in 1 sentence, then give the details of how you implemented it and any adaptations that are necessary for the present situation.

\Gls{idw} interpolation applies a weighted average to points within a neighborhood of a query point to obtain an interpolated value. \texttt{interp5DOF()} implements a simple \gls{idw} approach based on \cite{tovarInverseDistanceWeight2020}. A default radius of influence of $r=\sqrt{2} \mu$ is used, where $\mu$ represents mean \gls{nn} distance, and where octonion distance is approximated by the Euclidean distance or 2-norm (see \cref{sec:methods:vfz-dist}, and \cref{fig:dist-parity}). \gls{nn} interpolation (\cref{sec:methods:interp:nn}) is used for a given query point when there are no points in the radius of influence. The \gls{idw} interpolation method is invoked in \texttt{interp5DOF()} by setting the \texttt{method} argument to \texttt{'idw'}.

\subsubsection{\glsentrytitlecase{nn}{long} Interpolation}
\label{sec:methods:interp:nn}

% Again, explain the idea in 1 sentence, and then give any special adaptations necessary.

\Gls{nn} interpolation takes the nearest \inpt{} point relative to a query point and assigns the value of the \gls{nn} \inpt{} point to the query point. This is implemented via the built-in MATLAB function \texttt{dsearchn()} using a Euclidean approximation of octonion distance (see \cref{sec:methods:vfz-dist}, and \cref{fig:dist-parity}). The \gls{nn} interpolation method is invoked in \texttt{interp5DOF()} by setting the \texttt{method} argument to \texttt{'nn'}.

\section{Results and Discussion} \label{sec:results}

To illustrate the utility of the \gls{vfzo} framework for one application, namely interpolation, we compare the (i) accuracy, and (ii) efficiency of the four previously described interpolation methods implemented over the \gls{vfz} with each other and with existing methods from the literature (see \cref{sec:intro}).

For these tests, we used the 5DOF GB energy function developed by Bulatov, Reed and Kumar \cite{bulatovGrainBoundaryEnergy2014}, as the test function. First, a \gls{vfz} point set is randomly generated containing between \num{100} to \num{50000} points (different for each trial run), corresponding to resolutions of \SI{10.7175 \pm 0.3684}{\degree} and \SI{2.87 \pm 0.7}{\degree}, respectively, out of approximately \num{70} trial runs per \gls{vfzo} set size (\cref{fig:nndist-vs-setsize}). For a specific \num{50000} \gls{vfzo} set, the \gls{nn} distance is \SI{2.87 \pm 0.7}{\degree} (\cref{fig:nnhist-knn-50000}a) and the 100th \glspl{nn} distance is under \SI{10}{\degree} (\cref{fig:nnhist-knn-50000}b). We denote this set of \glspl{gb} by \texttt{gb\_v}. The test function was then evaluated at each of these points and the values were stored in the vector \texttt{f\_v}. We then generated \num{10000} query \glspl{vfzo}, denoted by \texttt{gb\_q}. Finally, we used \texttt{interp5DOF()} \cite{bairdFiveDegreeofFreedom5DOF2020} to predict the function value at the query points, which were stored in the vector \texttt{f\_q}. We also evaluated the test function at the query points to obtain the true function values, \texttt{f\_true}. We repeated this process for each of the interpolation methods, and compared the predictions, \texttt{f\_q}, to the true values, \texttt{f\_true}.

\begin{figure}
    \centering
    \includegraphics[scale=1]{nndist-vs-setsize.png}
    \caption{\Gls{nn} \gls{vfzo} ($\omega_{\text{NN}}$ distances ($^{\circ}$) versus \gls{vfzo} set size out of approximately 70 trial runs.}
    \label{fig:my_label}
\end{figure}

\begin{figure}
\centering
\includegraphics[scale=1]{nnhist-knn-50000.png}
\caption{Histogram of \gls{nn} octonion distances ($\omega$) in a \gls{vfzo} set of \num{50000} points. The average \gls{nn} distance was \SI{2.87 \pm 0.7}{\degree}. the average k-th nearest neighbor distances demonstrate many nearest neighbors fall within a tight yolerance (less then \SI{10}{\degree}) (b).}
\label{fig:nnhist-knn-50000}
\end{figure}

\subsection{Interpolation Accuracy}
\label{sec:results:accuracy}

\Cref{fig:brkparity50000} provides hexagonally binned parity plots \cite{beanHexscatter2020} for each of the four interpolation methods.
\begin{figure}
    \centering
    \includegraphics[scale=1]{brkparity50000.png}
    \caption{Hexagonally binned parity plots for \num{50000} mesh and \num{10000} \outpt{} octonions formed via pairs of a random cubochorically sampled quaternion and a spherically sampled random boundary plane normal. Interpolation via (a) \acrlong{gpr}, (b) \acrlong{idw}, (c) \acrlong{nn}, and (d) barycentric coordinates.  \acrlong{brk} \acrlong{gbe} function for \gls{fcc} Ni was used as the test function.}
    \label{fig:brkparity50000}
\end{figure}

All of the methods permit successful interpolation, and the highest density region in all cases falls squarely on the parity line. The \Gls{gpr} and barycentric results show a slight asymmetry such that low energy values are overpredicted more often than they are underpredicted. The width of the point clouds provides a qualitative indication of the dispersion in the prediction errors. Quantitative measures of the overall accuracy are presented for \gls{rmse} (\cref{tab:rmse-error-comparison}) and \gls{mae} (\cref{tab:mae-error-comparison}) along with results of prior work where available.

\begin{table}
\caption{Comparison of average interpolation error for each interpolation method in the present work, using \num{50000} points in the definition of the \gls{vfz}. Comparison to the work of other authors is also included. A constant model (Cst, Avg \gls{rmse}), whose value was chosen to be the mean of the \inpt{} \gls{gbe} was used as a control, except in the case of \gls{lobpcg} which used the mean of the true \gls{gbe} values instead. Additionally, \gls{lobpcg} is not interpolation, but rather reconstruction of specific \glspl{gbe}. The last two columns, \gls{rmse} $\downarrow$ (\SI{}{\J\per\square\meter} and \gls{rmse}   $\downarrow$ (\%)), represent the reduction in \gls{rmse} in units of \SI{}{\J\per\square\meter} and \% relative to the control model, respectively.}
\centering
\begin{tabular}{lccccc}
\toprule
Method &
  \thead{\# \glspl{gb}} &
  \thead{\gls{rmse} \\   (\SI{}{\J\per\square\meter})} &
  \thead{Cst, Avg \gls{rmse} \\   (\SI{}{\J\per\square\meter})} &
  \thead{\gls{rmse} $\downarrow$ \\   (\SI{}{\J\per\square\meter})} &
  \thead{\gls{rmse}   $\downarrow$ \\ (\%)} \\ \midrule
\Gls{gpr}                                                     & \num{50000}  & \num{0.0541} & \num{0.1296} & \num{0.0755} & \num{58.3} \\
Barycentric                                                   & \num{50000}  & \num{0.0607} & \num{0.1296} & \num{0.0689} & \num{53.2} \\
\gls{idw}                                                     & \num{50000}  & \num{0.068}  & \num{0.1296} & \num{0.0616} & \num{47.5} \\
\gls{nn}                                                      & \num{50000}  & \num{0.0821} & \num{0.1296} & \num{0.0475} & \num{36.7} \\
\gls{lobpcg}   \cite{shenDeterminingGrainBoundary2019}        & \num{180000} & \num{0.0092} & \num{0.0976} & \num{0.0884} & \num{90.6} \\
\gls{ann}   \cite{restrepoUsingArtificialNeural2014} & \num{17176}  & \NA          & \num{0.0854} & \NA          & \NA        \\
\gls{lkr}   \cite{chesserLearningGrainBoundary2020}           & \num{388}    & \num{0.0977} & \num{0.2243} & \num{0.1266} & \num{56.4} \\ \bottomrule
\end{tabular}
\label{tab:rmse-error-comparison}
\end{table}


\begin{table}
\caption{Comparison of average interpolation error for each interpolation method in the present work, using \num{50000} points in the definition of the \gls{vfz}. Comparison to the work of other authors is also included. A constant model (Cst, Avg \gls{mae}), whose value was chosen to be the mean of the \inpt{} \gls{gbe} was used as a control, except in the case of \gls{lobpcg} which used the mean of the true \gls{gbe} values instead. Additionally, \gls{lobpcg} is not interpolation, but rather reconstruction of specific \glspl{gbe}. The last two columns represent the reduction ($\downarrow$) in \gls{mae} in absolute units of \SI{}{\J\per\square\meter} and \% relative to the control model, respectively.}
\centering
\begin{tabular}{lccccc}
\toprule
Method &
  \# \glspl{gb} &
  \thead{\gls{mae} \\   (\SI{}{\J\per\square\meter})} &
  \thead{Cst, Avg \gls{mae} \\   (\SI{}{\J\per\square\meter})} &
  \thead{\gls{mae} $\downarrow$ \\   (\SI{}{\J\per\square\meter})} &
  \thead{\gls{mae}   $\downarrow$ \\ (\%)} \\ \midrule
\Gls{gpr}                                                     & \num{50000}  & \num{0.0372} & \num{0.0964} & \num{0.0592} & \num{61.4} \\
Barycentric                                                   & \num{50000}  & \num{0.0418} & \num{0.0964} & \num{0.0546} & \num{56.6} \\
\gls{idw}                                                     & \num{50000}  & \num{0.047}  & \num{0.0964} & \num{0.0494} & \num{51.2} \\
\gls{nn}                                                      & \num{50000}  & \num{0.0573} & \num{0.0964} & \num{0.0391} & \num{40.6} \\
\gls{lobpcg}   \cite{shenDeterminingGrainBoundary2019}        & \num{180000} & \NA          & \num{0.0466} & \NA          & \NA        \\
\gls{ann}   \cite{restrepoUsingArtificialNeural2014} & \num{17176}  & \num{0.0486} & \num{0.0617} & \num{0.0131} & \num{21.2} \\
\gls{lkr}   \cite{chesserLearningGrainBoundary2020}           & \num{388}    & \num{0.0977} & \num{0.1752} & \num{0.0775} & \num{44.2} \\ \bottomrule
\end{tabular}
\label{tab:mae-error-comparison}
\end{table}

% \begin{table}
% \caption{Comparison of average runtime for each interpolation method in the present work, using \num{50000} points in the definition of the \gls{vfz}. Comparison to the work of other authors is also included. A constant model, whose value was chosen to be the mean of \texttt{f\_v}, was used as a control.}
% \centering
% \begin{tabular}{lccc}
% \toprule
% Method & \# \glspl{gb} & Runtime (\SI{}{\second}) \\
% \midrule
% \Gls{gpr} & x & x \\
% Barycentric & x & x \\
% \gls{idw} & x & x \\
% \gls{nn} & x & x \\
% Constant & 0.13 & 0.097 \\
% LOBPCG \cite{shenDeterminingGrainBoundary2019} & 0.01--0.03 & --- \\
% ANN \cite{restrepoUsingArtificialNeural2014} & --- & 0.05--0.09 \\
% LKR \cite{chesserLearningGrainBoundary2020} & 0.0977 & --- \\
% \bottomrule
% \end{tabular}
% \label{tab:runtime-comparison}
% \end{table}

To aid in objective interpretation of the error metrics, comparison is made to a constant valued control model, whose value is chosen to be the average of \texttt{f\_v} (approximately \SI{1.16}{\J\per\square\meter} in the limit of $\texttt{\inptvar{}} \rightarrow \infty$) resulting in \gls{rmse} and \gls{mae} values of approximately \num{0.130} and \SI{0.097}{\J\per\square\meter}. Because the true models differed for other articles, equivalent errors obtained by a constant valued control model (mean of \inpt{} \gls{gbe} values\footnote{Because \cite{shenDeterminingGrainBoundary2019} uses polycrystal rather than bicrystal data, for which there is no direct measure of the \gls{gbe}, the mean of the true \glspl{gbe} is used instead.}) were also included for prior work in \cref{tab:rmse-error-comparison,tab:mae-error-comparison}. Additionally, \gls{lobpcg} is not interpolation, but rather reconstruction of specific \glspl{gbe} and is considered a complementary rather than competing technique. These metrics give a sense of the complexity and variability of the true function being inferred and allow for a more objective comparison between differing works.

Of the four interpolation methods from this work, \Gls{gpr} has the lowest error, both in terms of \gls{rmse} and \gls{mae}, while \gls{nn} has the highest error. Compared to a constant valued control model, \gls{gpr} interpolation reduced the prediction \gls{rmse} by \SI{58.3}{\percent}, which outperforms all of the interpolation methods in this work with respect to accuracy, as well as those considered from the literature, except for \gls{lobpcg} (which is a reconstruction, not an interpolation technique). After \gls{lobpcg} and \gls{gpr} the next most accurate methods are barycentric and \gls{lkr} interpolation (with the order depending on whether \gls{rmse} or \gls{mae} are used as the error measure), followed by \gls{idw}, \gls{nn} and then \gls{ann}.

The accuracy of the predictions made using the \gls{vfz} methods depends on the resolution of the \gls{vfzo} set. \Cref{fig:brkerror} compares the prediction accuracy for each of the 4 methods to the constant valued control model, as a function of the number of \inpt{} \glspl{vfzo} (\texttt{\inptvar{}}). As expected, higher density \gls{vfzo} sets result in lower error and diminishing returns (\cref{fig:brkerror}). Moreover, the standard deviations produced via multiple runs are tightly constrained and generally shrink as the \gls{vfzo} set size increases. 

\Gls{gpr} consistently gives lower error than the other three interpolation methods for all \gls{vfzo} set sizes. It is interesting to see that despite qualitative differences in the parity plots for barycentric and \gls{idw} interpolation, these two methods produce similar \gls{rmse} and \gls{mae} values. \Gls{nn} interpolation produces the worst error of the four methods, but is better than the constant valued control model so long as \texttt{\inptvar{}} exceeds a few hundred \inpt{} points. 

It is worthwhile to note that both \gls{gpr} and \gls{idw} are kernel-based in that a model parameter controls the size of the region that can influence the interpolation results. In the \gls{gpr} case, this is automatically calculated via an internal fitting routine of \texttt{fitrgp()}. \gls{nn} distance distributions (\cref{fig:nnhist-knn-50000}) can lead to insight about correlation lengths in a given \gls{vfzo} set and is used in the \gls{idw} implementation. For \gls{idw}, the radius of influence is set to $r=\sqrt{2} \mu$, where $\mu$ is the mean \gls{nn} distance. It is likely that better tuning of the kernel parameters in these two methods could further decrease their interpolation errors.

By contrast, barycentric interpolation automatically adjusts its effective region of influence because the size of the simplices in the mesh decreases as the number of vertices increases. More uniformly distributed meshes (such as obtained via constrained optimization \cite{dolanBenchmarkingOptimizationSoftware2004,ConstrainedElectrostaticNonlinear2020}) will likely result in more uniform, decreased interpolation error, especially for this simplex-based approach which is can exhibit some high-aspect ratio facets and non-intersections outside the bounds of the mesh. % However, this may be computationally infeasible for large mesh sizes.

% While not implemented in this work, it is expected that an \gls{ann} such as in \cite{restrepoUsingArtificialNeural2014} could lead to even lower interpolation error than \gls{gpr} for large mesh sizes. % (e.g. https://www.mathworks.com/help/deeplearning/ref/narxnet.html, https://www.mathworks.com/help/deeplearning/function-approximation-and-nonlinear-regression.html)

\begin{figure}
    \centering
    \includegraphics[scale=1]{brkerror.png}
    \caption{Average \gls{rmse} and average \gls{mae} vs. number of \inpt{} points for planar barycentric (\texttt{pbary}, blue), \gls{gpr} (\texttt{gpr}, orange), \gls{idw} (\texttt{idw}, yellow), and \texttt{nn} (\texttt{nn}, purple) interpolation for approximately 10 random runs with different \inpt{} and \outpt{} points. Standard deviations of approximately 10 runs are also included. Compare with approximately \SI{0.13}{\J\per\square\meter} and \SI{0.95}{\J\per\square\meter} \gls{rmse} and \gls{mae}, respectively, for a constant model using the average of the \inpt{} properties (approximately \SI{1.16}{\J\per\square\meter}).}
    \label{fig:brkerror}
\end{figure}

\subsection{Interpolation Efficiency}
\label{sec:results:efficiency}

% general overview statement about the interpolation time and memory requirements

\Gls{gpr} is fast and has lower error compared to barycentric interpolation; however, the entire process has to be reevaluated (in current implementation) if the responses or the predictors change. On the other hand, barycentric interpolation is fast once the triangulation and intersections are computed. In other words, it is fast if only responses (i.e. \glspl{gbe}) change, but slow if the \inpt{} points or predictors (i.e. \glspl{gb}) change.

Computational runtimes of the various interpolation methods are also shown (\cref{fig:runtime}). Barycentric interpolation takes the longest, which is compounded by the fact that it is the only parallelized method by default (not accounted for in \cref{fig:runtime}). In other words, since 12 cores were used to obtain these runtime results, the total runtime across all cores is much higher compared with the other methods; however, it is possible that other methods used multi-threading via built-in vectorized functions. The long computation times result primarily from the large number of facets present in a high-dimensional mesh triangulation and the interconnectedness of facets with respect to each other.

\Gls{gpr} is the second-longest in terms of of runtime, but is more accurate than any of the other three methods. \Gls{nn} and \gls{idw} interpolation have vectorized implementations and are much simpler than the barycentric and \gls{gpr} methods. Consistent with expectations, \gls{nn} and \gls{idw} exhibit almost negligible runtimes; however, this is at the expense of increased error discussed earlier (\cref{sec:results:accuracy}). It should also be noted that barycentric interpolation and \gls{gpr} have much higher memory requirements than \gls{nn} and \gls{idw} due to the need to store large matrices (unless \texttt{fitrgp PredictMethod = 'fic'}).

Because the default implementation of \gls{idw} uses a radius cut-off, the distance and weight matrices can be stored as sparse objects, dramatically reducing both the storage requirements and computational complexity of this method. We expect that a \gls{knn} approach would produce similar results both in terms of runtime and error when a relatively uniform sampling of \gls{gbc} is obtained.

\begin{figure}
    \centering
    \includegraphics[scale=1]{runtime.png}
    \caption{
    Linear (a) and log-scale (b) average runtime (s) and standard deviation (approximately 10 runs) vs. number of \inpt{} points for barycentric (blue) and \acrlong{gpr} (\acrshort{gpr}) (red), \acrlong{idw} (\acrshort{idw}) (yellow), and \acrlong{nn} (\acrshort{nn}) (purple) interpolation using 12 cores compared against runtime (s) for an average model (b, yellow). Because \acrshort{gpr}, \acrshort{idw}, and \acrshort{nn} method defaults do not use parfor loops but may have internal multi-core vectorization, it is unclear to what extent the number of cores affects the runtime of methods other than barycentric interpolation. \Acrlong{gb} symmetrization runtime was not included; however, symmetrization takes approximately one minute on 6 cores (Intel i7-10750H, 2.6 GHz) and is a common step in every interpolation method (i.e. it is fundamental to the \acrlong{vfzo} framework).
    }
    \label{fig:runtime}
\end{figure}

\section{Conclusion} \label{sec:conclusion}

In this work, we presented the \gls{vfzo} framework for computing distances between GBs and predicting the properties of GBs from existing measurements. We found that distance calculations in the \gls{vfzo} framework are dramatically more computationally efficient ($O(N_p^2L)$) than traditional methods ($O(N_p^4L^2)$).

We also developed and tested a barycentric interpolation method, and adapted three other interpolation methods for use in the \gls{vfzo} framework. We provide an easy-to-use, versatile implementation of our methods through an interpolation function \texttt{interp5DOF()} written in Matlab  (\url{github.com/sgbaird-5dof/interp}, \cite{bairdFiveDegreeofFreedom5DOF2020}).  This approach is general and applies to any crystal system (any of the 32 crystallographic point groups can be selected by the parameter \texttt{pgnum}). 

Of the interpolation methods that we present in this work, \Gls{gpr} provided the highest accuracy predictions. It also provided higher accuracy predictions than any of the methods in the literature, except for \gls{lobpcg} \cite{shenDeterminingGrainBoundary2019}. While \gls{idw}, and \gls{nn} interpolation, have the fastest computation times, they also have higher interpolation error. Consequently, we recommend the \gls{gpr} interpolation method for the \gls{vfzo} framework for most applications because it provides the best combination of accuracy and speed; however, the other methods can meet niche needs. For example, barycentric interpolation enables rapid and accurate predictions when the function to be evaluated changes, but the \outpt{} points remain fixed. 

We anticipate that the \gls{vfzo} framework and corresponding implementation will benefit numerous applications related to \gls{gb} structure and properties, including facilitating \gls{gb} structure-property model development, enabling efficient surrogate modeling of \gls{gb} properties, and larger scale iterative simulations that require repetitive evaluation of computationally expensive structure-property models.

\section*{Acknowledgement}
\label{sec:acknowledgement}

The material presented here is based upon work supported by the National Science Foundation under Grant No. 1610077.

\appendix \label{sec:app}
\renewcommand\thefigure{\thesection.\arabic{figure}}
\section{Detailed Barycentric Interpolation Methods}
\setcounter{figure}{0}
\subsection{Triangulating a \glsentrytitlecase{vfz}{long} Mesh}
\label{app:bary-tri}

% Briefly (but clearly and explicitly) explain the steps of generating the cubochorically sampled GBs, converting them to octonions and then mapping all of them into the vfz. Then explain how you do the triangulation. This will require a few figures to show the process in 3D (i.e. generating a triangulation of a point cloud on a region of the surface of the 2-sphere).

% After random \glspl{vfzo} are obtained (\cref{sec:methods:rand}), % ...

In order to reduce the computational complexity of triangulating a high-dimensional mesh \cite{barberQuickhullAlgorithmConvex1996}, some simplifications are made. The degenerate dimension obtained from analytically minimizing $U(1)$ symmetry \cite{francisGeodesicOctonionMetric2019} is first removed via a rigid (i.e. distance- and angle-preserving) \gls{svd} transformation,
%to enable use of MATLAB's quickhull \cite{barberQuickhullAlgorithmConvex1996} implementations such as \texttt{delaunayn} and \texttt{convhulln}. Removal of the degenerate dimension is done via a rigid \gls{svd} transformation,
analogous to a Cartesian rotation and translation
%Thus, a set of octonions originally represented by 8D Cartesian coordinates are collapsed to a 7D Cartesian representation while preserving both distances and angles among the points
As a 3D Cartesian to 2D Cartesian analogue of this 8D Cartesian to 7D Cartesian transformation, we take a set . Next, this 7D Cartesian representation is projected onto a hyperplane tangent to the mean of the input points\footnote{This is \textit{not} a rigid transformation; however, it is sufficiently close to produce a high-quality triangulation in a \gls{vfz}.} (\cref{fig:bary-delaunay}b). The 7D Cartesian \glspl{gbo} undergo another rigid \gls{svd} transformation to 6D Cartesian (see 3D to 2D analogue in \cref{fig:bary-delaunay}c) for which a triangulation is computed via the quickhull algorithm \cite{barberQuickhullAlgorithmConvex1996} (see MATLAB function \texttt{delaunayn}).

\begin{figure}
    \centering
    \includegraphics[scale=1]{bary-remove-deg.png}
    \caption{3D Cartesian to 2D Cartesian analogue of 8D Cartesian to 7D Cartesian degeneracy removal used in barycentric interpolation approach. Starting spherical arc points on surface of 2-sphere (a) and degenerate dimension removed via \acrlong{svd} transformation to 2D Cartesian (b) with either the origin (black plus) preserved (black asterisks, \texttt{zeroQ=T}) or ignored (red asterisks, \texttt{zeroQ=F}). The sphere (a) and circle (b) each have a radius of 0.8 and are used as a visualization aid only.}
    \label{fig:bary-remove-deg}
\end{figure}

Using a separate rigid \gls{svd} transformation from the triangulation, the original mesh and prediction \glspl{gbo} are simultaneously transformed from 8D Cartesian to 7D Cartesian. Because our \gls{svd} approach is rigid, the triangulation is then superimposed onto the newly transformed input points resulting in a 7D Cartesian \gls{vfz}.

\begin{figure}
    \centering
    \includegraphics[scale=1]{bary-delaunay.png}
    \caption{3D Cartesian to 2D Cartesian analogue of 7D Cartesian to 6D Cartesian mesh triangulation used in barycentric interpolation approach. Normalized starting points on surface of 2-sphere (a), points projected onto hyperplane tangent to mean of starting points (b), degenerate dimension removed via rigid \gls{svd} transformation to 2D Cartesian (c), and Delaunay triangulation in 2D Cartesian (d). The sphere in (a) and (b) has a radius of 0.8 and is used as a visualization aid only.}
    \label{fig:bary-delaunay}
\end{figure}

% In order to reduce the computationally complexity of computing barycentric coordinates in a high-dimensional space \cite{barberQuickhullAlgorithmConvex1996}, a single degenerate dimension (originally introduced by analytically minimizing $U(1)$ symmetry) is removed to enable use of MATLAB's quickhull \cite{barberQuickhullAlgorithmConvex1996} implementations such as \texttt{delaunayn} and \texttt{convhulln}. Removal of the degenerate dimension is done via a rigid \gls{svd} transformation, analogous to a Cartesian rotation and translation. Thus, a set of octonions originally represented by 8D Cartesian coordinates are collapsed to a 7D Cartesian representation while preserving both distances and angles among the points (see 3D to 2D analogue in \cref{fig:bary-remove-deg}). To further reduce the "curse of dimensionality" in computing the triangulation, a 7D Cartesian representation of the octonions constrained to lie on the surface of the 6-sphere are first projected onto a hyperplane tangent to the mean of the input points and then rotated/translated again via \gls{svd} to produce a 6D Cartesian representation (see 3D to 2D analogue in Figure \cref{fig:bary-delaunay}). This 6D representation is used to compute a triangulation via the built-in MATLAB routine \texttt{delaunayn} based on the quickhull algorithm \cite{barberQuickhullAlgorithmConvex1996}, giving facet vertices for the 7D Cartesian hypersphere.

\subsection{Intersections in a \glsentrytitlecase{vfz}{long} Mesh}
\label{app:bary-int}
An intersection is calculated by linearly projecting a prediction point onto the hyperplane defined by a mesh facet's vertices, computing barycentric coordinates within the facet, and testing for positivity \cite{langerSphericalBarycentricCoordinates2006}, and repeating this process until an intersection is found or a stop condition is reached (see \texttt{nnMax} below). If the prediction point is contained within the simplex, all of the barycentric weights (i.e. coordinates) are positive. If it outside the simplex, this constraint is violated. Due to the large number of facets per point of a high-dimensional
%simplex-based
triangulation
%and for computational speed,
prediction point intersections are calculated by considering facets connected to up to some number of mesh \glspl{nn} (\texttt{nnMax}) (in this work, \texttt{nnMax = 10}). For prediction points where no intersecting facet is found
%in these connected facets due to high-aspect ratio facets or prediction point falling outside the mesh within a given tolerance,
a \gls{nn} approach is used (\cref{sec:methods:interp:nn}), which represent average non-intersection rates of \SI{0.1207 \pm 1.02}{\percent} and \SI{0.68 \pm 0.11}{\percent} for input meshes of \num{388} and \num{50000}, respectively, for \num{10000} query points out of \num{10} trials.

\subsubsection{Interpolation via Barycentric Coordinates}
\label{app:bary-interp}

Once barycentric coordinates are computed for a prediction point within the input mesh, the interpolated value is found by taking the dot product of the barycentric coordinates and the properties of the corresponding vertices of the intersecting facet via
\begin{equation}
v=\underset{i=1}{\overset{N}{\sum }}\lambda _i v_i
\end{equation}
where $\lambda$, $v$, $v_i$ and $N$, are the barycentric coordinates, interpolated property, property of the $i$th vertex of the intersecting facet, and number of vertices in a given facet ($N = 7$ for the degeneracy-free 6-sphere).

\begin{figure}
    \centering
    \includegraphics[scale=1]{bary-interp.png}
    \caption{Caption}
    \label{fig:my_label}
\end{figure}

% \appendix
\begin{appendices}

\crefalias{section}{appsec}
\crefalias{subsection}{appsec}

\section{Detailed Barycentric Interpolation Method}
\label{sec:app}
\setcounter{figure}{0}
\subsection{Triangulating a \glsentrytitlecase{vfz}{long} Mesh}
\label{app:bary-tri}

% Briefly (but clearly and explicitly) explain the steps of generating the cubochorically sampled GBs, converting them to octonions and then mapping all of them into the vfz. Then explain how you do the triangulation. This will require a few figures to show the process in 3D (i.e. generating a triangulation of a point cloud on a region of the surface of the 2-sphere).

% After random \glspl{vfzo} are obtained (\cref{sec:methods:rand}), % ...

In this section we explain how to generate a simplicial mesh from a \gls{gbo} point cloud (\cref{fig:bary-delaunay}) on the surface of the octonion 7-sphere in 8-dimensions. Throughout the explanation we provide a visual illustration of the process as applied to lower-dimensional scenarios in \cref{fig:bary-remove-deg}, \cref{fig:bary-delaunay}, and \cref{fig:bary-interp}. We note that 3D Cartesian in \cref{fig:bary-remove-deg} corresponds to 8D Cartesian and 3D Cartesian in \cref{fig:bary-delaunay} and \cref{fig:bary-interp} correspond to 7D Cartesian in the \gls{vfz} framework. This is intentional for two reasons:
\begin{itemize}
    \item \cref{fig:bary-remove-deg} illustrates the importance that a symmetrized \gls{gbo} set in the 7-sphere, 8D Cartesian \gls{vfz} is analogous to a geodesic arc on the 2-sphere (\cref{fig:bary-remove-deg}b) with a degenerate dimension which is removed by the first \gls{svd} transformation (\cref{fig:bary-remove-deg}b), which would otherwise be difficult to visualize in the 4D Cartesian 3-sphere to 3D Cartesian 2-sphere case.
    \item \cref{fig:bary-delaunay} illustrates the second \gls{svd} transformation from 7D Cartesian coordinates on the tangent hyperplane to the 6-sphere (\cref{fig:bary-delaunay}a) to 6D Cartesian coordinates on the 5-sphere (\cref{fig:bary-delaunay}b) in a way that retains key features that would otherwise be lost\footnote{Non-intersection issues due to high-aspect ratios and consideration of facets connected up to \texttt{nnMax} \glspl{nn} do not manifest in triangulations on the surface of a 1-sphere.} and gives a more familiar illustration of triangulation (i.e. it uses actual triangles).
\end{itemize}

\begin{figure}
    \centering
    \includegraphics[scale=1]{bary-remove-deg.png}
    \caption{3D Cartesian to 2D Cartesian analogue of 8D Cartesian to 7D Cartesian degeneracy removal used in barycentric interpolation approach. Starting spherical arc points on surface of 2-sphere (a) and degenerate U(1) symmetry removed via \acrlong{svd} transformation to 2D Cartesian (b) with either the origin (black plus) preserved (black asterisks, \texttt{zeroQ=T}) for triangulation or ignored (red asterisks, \texttt{zeroQ=F}) for mesh intersection. The spheres (a,b) and circle (c) each have a radius of 0.8 and are used as a visualization aid only.}
    \label{fig:bary-remove-deg}
\end{figure}

\begin{figure}
    \centering
    \includegraphics[scale=1]{bary-delaunay.png}
    \caption{3D Cartesian to 2D Cartesian analogue of 7D Cartesian to 6D Cartesian mesh triangulation used in barycentric interpolation approach. 3D Cartesian \inpt points (a) linearly projected onto hyperplane tangent to mean of starting points (b), degenerate dimension removed via rigid \gls{svd} transformation to 2D Cartesian and Delaunay triangulation calculated (black lines) with \inpt vertices (red) (c), and Delaunay triangulation superimposed onto normalized \inpt{} points (d). The spheres in (a), (b), and (d) have a radius of 0.8 and are used for visualization only.}
    \label{fig:bary-delaunay}
\end{figure}

To reduce the computational complexity of triangulating a high-dimensional mesh \cite{barberQuickhullAlgorithmConvex1996}, some simplifications are made. First, the degenerate octonion dimension obtained from analytically minimizing $U(1)$ symmetry \cite{francisGeodesicOctonionMetric2019} is removed via a rigid (i.e. distance- and angle-preserving) \gls{svd} transformation,
%to enable use of MATLAB's quickhull \cite{barberQuickhullAlgorithmConvex1996} implementations such as \texttt{delaunayn} and \texttt{convhulln}. Removal of the degenerate dimension is done via a rigid \gls{svd} transformation,
analogous to a Cartesian rotation and translation (see 3D to 2D \gls{svd} transformation in \cref{fig:bary-remove-deg}).
%Thus, a set of octonions originally represented by 8D Cartesian coordinates are collapsed to a 7D Cartesian representation while preserving both distances and angles among the points
Next, the resulting 7D Cartesian representation of each \gls{gbo} is projected onto a hyperplane tangent to the centroid of the \gls{gbo} point cloud\footnote{This is \textit{not} a rigid transformation; however, it approximates one with sufficient accuracy to produce a high-quality triangulation in a \gls{vfz}.} (\cref{fig:bary-delaunay}(a)). The 7D Cartesian \glspl{gbo} are then subjected to another rigid \gls{svd} transformation to remove the now additional degenerate dimension and project them to a 6D Cartesian space (see 3D to 2D projection in \cref{fig:bary-delaunay}(a)-(b)). Finally, the resulting point cloud can be triangulated via the quickhull algorithm \cite{barberQuickhullAlgorithmConvex1996} (see MATLAB function \texttt{delaunayn()}) to produce a simplicial mesh. Because the simplicial mesh is simply defined by a list of edges between vertices for each simplicial facet, this list applies immediately to the \gls{gbo} point cloud in its original coordinates (i.e. no reverse transformation is necessary to use the mesh on the original octonion 7-sphere in 8D).

Other distance metrics \cite{morawiecDistancesGrainInterfaces2019} could also be incorporated into the barycentric approaches described here such as by doing an edge-length based simplex reconstruction with the \gls{vfz} triangulation \cite{connorHighdimensionalSimplexesSupermetric2017,boissonnatOnlyDistancesAre2017} and is a topic of future study.

% In order to reduce the computationally complexity of computing barycentric coordinates in a high-dimensional space \cite{barberQuickhullAlgorithmConvex1996}, a single degenerate dimension (originally introduced by analytically minimizing $U(1)$ symmetry) is removed to enable use of MATLAB's quickhull \cite{barberQuickhullAlgorithmConvex1996} implementations such as \texttt{delaunayn} and \texttt{convhulln}. Removal of the degenerate dimension is done via a rigid \gls{svd} transformation, analogous to a Cartesian rotation and translation. Thus, a set of octonions originally represented by 8D Cartesian coordinates are collapsed to a 7D Cartesian representation while preserving both distances and angles among the points (see 3D to 2D analogue in \cref{fig:bary-remove-deg}). To further reduce the "curse of dimensionality" in computing the triangulation, a 7D Cartesian representation of the octonions constrained to lie on the surface of the 6-sphere are first projected onto a hyperplane tangent to the mean of the \inpt{} points and then rotated/translated again via \gls{svd} to produce a 6D Cartesian representation (see 3D to 2D analogue in Figure \cref{fig:bary-delaunay}). This 6D representation is used to compute a triangulation via the built-in MATLAB routine \texttt{delaunayn} based on the quickhull algorithm \cite{barberQuickhullAlgorithmConvex1996}, giving facet vertices for the 7D Cartesian hypersphere.

\subsection{Intersections in a \glsentrytitlecase{vfz}{long} Mesh}
\label{app:bary-int}

To ensure that the positions of the \outpt{} points relative to the mesh is unchanging after the rigid \gls{svd} transformation, it is crucial to perform the same rigid \gls{svd} transformation (i.e. same rotation and translation) on the \outpt{} points as for the \inpt{} points.
This is easily accomplished by concatenating both \inpt{} and
\outpt{} points together using the \texttt{interp5DOF} sub-routine \texttt{proj\_down} (which depends on MATLAB's built-in \gls{svd} implementation, \texttt{svd()}), performing the transformation, and splitting the points (reverse of concatenation).

To find the barycentric coordinates or weights of a \gls{gbo} \outpt{} point relative to a \gls{vfz} simplicial mesh it is necessary to first identify which simplicial facet the point is contained in, and then to compute the barycentric coordinates of the point relative to the vertices of that facet. These steps are accomplished simultaneously by the following approach.

The facet containing the \outpt{} point is the one for which the point's barycentric coordinates are positive. Consequently, we determine facet affiliation by linearly projecting the \outpt{} point onto the hyperplane defined by a mesh facet's vertices (\cref{fig:bary-interp}), computing the point's barycentric coordinates within the facet, testing that all coordinates are positive \cite{langerSphericalBarycentricCoordinates2006}, and repeating this process until an intersection is found or a stop condition is reached (see \texttt{nnMax} below). 
\begin{figure}
    \centering
    \includegraphics[scale=1]{bary-interp.png}
    \caption{A ray (red line) on the 2-sphere is linearly projected onto the hyperplane of a mesh facet (transparent black), shown as a red asterisk. The barycentric coordinates are computed as $\lambda_{i \in [1,3]} = \frac{1}{3}$. Because all barycentric coordinates are positive, it is determined that the projected point is an intersection with the mesh. Given vertex values of \num{8.183}, \num{3.446}, and \num{3.188} for vertices 1, 2, and 3, respectively, the interpolated value is calculated as \num{4.94} via \cref{eq:bary-interp}.}
    \label{fig:bary-interp}
\end{figure}

Due to the large number of facets per point of a high-dimensional triangulation,
%simplex-based
triangulation,
%and for computational speed,
If every edge length of every facet were equal, only facets connected to the first \gls{nn} would need to be considered to find a proper intersection. However, since the \glspl{gbo} are randomly sampled, edge lengths of facets are non-uniform, and non-unity aspect ratio facets exist. If the facets have high-aspect ratios, the intersecting facets of \outpt{} points can be far from the \glspl{nn} mesh points relative to the \outpt{} points. High-dimensional triangulations have exponentially increasing numbers of facets per vertex. On average, there are approximately \num{2000} facets per vertex for a \num{50000} point \gls{vfz} triangulation (\num{1e8} total facets). Rather than loop through every facet to find an intersection which would be computationally taxing, the \outpt{} point intersections are calculated by considering facets connected to up to some number of \gls{nn} mesh vertices (\texttt{nnMax}) relative to each \outpt{} point (in this work, \texttt{nnMax = 10}). The \gls{nn} mesh vertices to an \outpt{} point are computed via the MATLAB built-in function \texttt{dsearchn} as in the \gls{nn} approach (\cref{sec:methods:interp:nn}), and the facet IDs of facets connected to these \glspl{nn} are computed by simple calls to built-in MATLAB function \texttt{find(K==nn)}, where K is the triangulation from \texttt{sphconvhulln()} and \texttt{nn} is the ID of one of the \gls{nn} mesh vertices. For \outpt{} points where no intersecting facet is found
%(which can occur just outside the piece-wise linear perimeter of the mesh due to finite resolution)
(which can occur due to high-aspect ratio facets or \outpt{} point falling outside the perimeter of the mesh),
properties are interpolated via a \gls{nn} approach (\cref{sec:methods:interp:nn}). In numerical tests, with \num{10} trials using \num{10000} query points for \gls{vfz} meshes composed of \num{388} and \num{50000} vertices, the average rate of non-intersections was \SI{12.07 \pm 1.02}{\percent} and \SI{0.68 \pm 0.11}{\percent}, respectively.

\subsection{Interpolation via Barycentric Coordinates}
\label{app:bary-interp}

Once barycentric coordinates are computed for a \outpt{} point within the \inpt{} mesh, the interpolated value is found by taking the dot product of the \outpt{} point's barycentric coordinates and the properties of the corresponding vertices of the intersecting facet via
\begin{equation}
\label{eq:bary-interp}
v_q=\underset{i=1}{\overset{N}{\sum }}\lambda _i v_{m,i}
\end{equation}
where $\lambda_i$, $v_q$, $v_{m,i}$ and $N$, are the barycentric coordinates of the \outpt{} point, interpolated property at the \outpt{} point, property of the $i$-th vertex of the intersecting facet, and number of vertices in a given facet ($N = 7$ for facets of the simplicial mesh on the degeneracy-free 6-sphere). Interpolation of many \outpt{} points simultaneously can be accomplished by a simple and efficient matrix multiplication
\begin{equation}
    \label{eq:bary-interp_multi}
    \mathbf{v}_q = \Lambda \mathbf{v}_m
\end{equation}
where $\mathbf{v}_m$ is the vector of GB properties at the mesh vertices, $\Lambda$ is the sparse matrix whose rows contain the barycentric weights for each query point, and $\mathbf{v}_q$ is the vector containing the interpolated properties at the query points.

\end{appendices}

\newpage
\printglossaries
%need to manually clear cached files & logs in overleaf to get new abbreviations to appear

\newpage
\bibliographystyle{elsarticle-num-names}
\bibliography{5dof-gb-energy.bib}



\end{document}
