% start the document

% specify the document layout and font size
\documentclass[preprint,12pt]{elsarticle}
% \documentclass[final,twocolumn,12pt]{elsarticle}
% \usepackage[margin=1.5cm,includefoot]{geometry}
\usepackage{setspace}

% uploading packages
\usepackage{graphicx}
\usepackage{amssymb}
\usepackage{textcomp} % https://latex.org/forum/viewtopic.php?f=4&t=3364#p13124, https://tex.stackexchange.com/questions/165115
\usepackage{gensymb}
\usepackage{lineno}
\usepackage{mathtools}
\usepackage[title]{appendix}
\usepackage{pgfmath}
\newcommand{\calcnum}[1]{%
    \pgfmathparse{#1}%
    \num[round-mode=places,round-precision=1]{\pgfmathresult}%
}
\usepackage[separate-uncertainty=true]{siunitx} % \usepackage{xr-hyper} %needs to  be before hyperref
\usepackage{xurl} %needs to be before hyperref
\usepackage[colorlinks]{hyperref}
\hypersetup{breaklinks=true} % set automatically by hyperref?
% \PassOptionsToPackage{hyphens}{url}\usepackage{hyperref} %allow URLs to break across lines
\usepackage[nameinlink,capitalise]{cleveref} %needs to appear after hyperref, https://tex.stackexchange.com/questions/396728/my-equations-referencing-not-working
\Crefname{figure}{Figure}{Figures} %needs to appear after hyperref and cleveref
\crefname{appsec}{Appendix}{Appendices}
\newcommand\crefrangeconjunction{--} % modify the reference style
\usepackage{mathrsfs}
\usepackage{enumitem}
\usepackage{tabulary}
\usepackage{caption}
\usepackage{subcaption}
\usepackage{multirow}
\usepackage{makecell} % https://tex.stackexchange.com/questions/2441/how-to-add-a-forced-line-break-inside-a-table-cell
\newcommand{\NA}{---} % holds an m-dash
\graphicspath{{figures/}} %Setting the graphicspath
% ---------to deal with the double quotes----------- 
\usepackage [english]{babel}
\usepackage [autostyle, english = american]{csquotes}
\MakeOuterQuote{"}
\usepackage{listings}[breakatwhitespace=true,escapechar=\%]
\usepackage{matlab-prettifier}
\newcommand{\matlab}[1]{\mbox{\lstinline[style=Matlab-editor]{#1}}}
%alternatively can use `` '' format for double quotes
\usepackage{booktabs}
\setlength{\abovetopsep}{1ex}
\usepackage[shortcuts,abbreviations]{glossaries-extra}
\newcommand*{\TCac}[1]{\ecapitalisewords{\glsentrylong{#1}}}
% remove the "Preprint submitted to Elsevier" footer on the first page
\makeatletter
\def\ps@pprintTitle{%
   \let\@oddhead\@empty
   \let\@evenhead\@empty
   \def\@oddfoot{\reset@font\hfil\thepage\hfil}
   \let\@evenfoot\@oddfoot
}
\makeatother

% Cross referencing with the xr package in Overleaf (https://www.overleaf.com/learn/how-to/Cross_referencing_with_the_xr_package_in_Overleaf)
\makeatletter
\newcommand*{\addFileDependency}[1]{% argument=file name and extension
  \typeout{(#1)}
  \@addtofilelist{#1}
  \IfFileExists{#1}{}{\typeout{No file #1.}}
}
\makeatother
\newcommand*{\myexternaldocument}[1]{%
    \externaldocument{#1}%
    \addFileDependency{#1.tex}%
    \addFileDependency{#1.aux}%
}

\usepackage{nameref,zref-xr}
\zxrsetup{toltxlabel}
\zexternaldocument*{supp} %https://tex.stackexchange.com/questions/77774/undefined-control-sequence-when-cross-referencing-with-xr-hyper
% \myexternaldocument{supp}

\biboptions{sort&compress}
\interfootnotelinepenalty=10000 %prevent footnotes from getting split across columns/pages 
%\patchcmd{\emailauthor}{(#2)}{(S.G. Baird).}{}{} %Removes/Abbreviates corresponding author name after Email address so that the footnote doesn't take up 2 lines.
% Double Spacing
% \doublespacing

\usepackage[margin=1.5cm,includefoot]{geometry}
\usepackage{auto-paper}
\usepackage{refcheck}
\zexternaldocument*{main-frankenstein-1} %try deleting log files if producing an error, see https://tex.stackexchange.com/questions/131709/unclean-aux-file-causes-file-ended-while-scanning-use-of-newlbel-error-wh
% Concatenate the different "values" .tex files
%RMSE values
% \newcommand{\baryrmse}{0.0242}
% \newcommand{\gprrmse}{0.0220}
% \newcommand{\idwrmse}{0.0345}
% \newcommand{\nnrmse}{0.0448}
% \newcommand{\avgrmse}{0.1302}
%paper-data6
\newcommand{\baryrmse}{0.0238}
\newcommand{\gprrmse}{0.0218}
\newcommand{\idwrmse}{0.0356}
\newcommand{\nnrmse}{0.0445}
\newcommand{\avgrmse}{0.1283}

\newcommand{\gprrmsePercReduction}{83}

%MAE values
% \newcommand{\barymae}{0.0145}
% \newcommand{\gprmae}{0.0145}
% \newcommand{\idwmae}{0.0223}
% \newcommand{\nnmae}{0.0307}
% \newcommand{\avgmae}{0.0965}
%paper-data6
\newcommand{\barymae}{0.0145}
\newcommand{\gprmae}{0.0145}
\newcommand{\idwmae}{0.0225}
\newcommand{\nnmae}{0.0307}
\newcommand{\avgmae}{0.0955}

\newcommand{\nnomega}{2.8709 \pm 00.69112}

\newcommand{\symtime}{76}

%Supplementary
\newcommand{\thr}{\SI{1.1}{\joule\per\square\meter}}
\newcommand{\sigthr}{\SI{1.1}{\joule\per\square\meter}}
\newcommand{\thrtwo}{\SI{1.2}{\joule\per\square\meter}}


%% main-frankenstein-2
\newcommand{\minsymdist}{$\sim$\SI{64.0}{\degree}}
\newcommand{\percExplained}{$\sim$\SI{99.6}{\percent}}
\newcommand{\percFiveVsOne}{$\sim$\SI{70}{\percent}}
\newcommand{\dimOne}{$\sim$\SI{65}{\degree}}

% figure info, etc. that can dynamically change (color of points, etc.)
\newcommand{\startpt}{red points}
\newcommand{\singlept}{magenta points}
\newcommand{\sympt}{dark blue points}
\newcommand{\singlesympt}{dark blue point}
\newcommand{\refpt}{white circle}
\newcommand{\vbordercolor}{black}
\newcommand{\vcellcolor}{light blue}
\newcommand{\inpt}{input}
\newcommand{\outpt}{prediction}
% \newcommand{\inptvar}{ninputpts}
% \newcommand{\distfn}{GBdist4}
\newcommand{\vfzorepo}{\gls{vfz} repository}
\newcommand{\mytitleone}{Five Degree-of-Freedom Property Interpolation of Arbitrary Grain Boundaries via \glsentrytitlecase{vfz}{long} Framework}
% \newcommand{\mytitletwo}{Properties of a \glsentrytitlecase{5dof}{long} \glsentrytitlecase{fz}{long} defined via \glsentrytitlecase{vfz}{long} Framework}
\newcommand{\mytitletwo}{$O_h$ \glsentrytitlecase{5dof}{long} \glsentrytitlecase{fz}{long} Properties via \glsentrytitlecase{vfz}{long} Framework}
\makeglossaries
\GlsXtrEnableEntryCounting{abbreviation}{3}
% \glssetcategoryattribute{abbreviation}{indexonlyfirst}{true}
\glssetcategoryattribute{abbreviation}{nohyper}{true}

% \setabbreviationstyle[abbreviation]{long-short}

% \glsenableentrycount
% \glssetcategoryattribute{abbreviation}{entrycount}{2}

\newabbreviation[longplural=five degrees of freedom]{5dof}{5DOF}{five degree-of-freedom}
\newabbreviation[longplural=three degrees of freedom]{3dof}{3DOF}{three degree-of-freedom}
\newabbreviation[longplural=degrees of freedom]{dof}{DOF}{degree of freedom}
\newabbreviation{ebsd}{EBSD}{electron backscatter diffraction}
\newabbreviation[longplural={grain boundaries}]{gb}{GB}{grain boundary}
\newabbreviation{fcc}{FCC}{face-centered cubic}
\newabbreviation{sem}{SEM}{scanning electron microscope}
\newabbreviation{fea}{FEA}{finite element analysis}
\newabbreviation{bcs}{BCs}{boundary conditions}
\newabbreviation[longplural={triple junctions}]{tj}{TJ}{triple junction}
\newabbreviation{gpr}{GPR}{Gaussian process regression}
\newabbreviation{gprm}{GPRM}{Gaussian process regression mixture}
\newabbreviation{ann}{ANN}{artificial neural network}
\newabbreviation{nn}{NN}{nearest neighbor}
\newabbreviation{rmse}{RMSE}{root mean square error}
\newabbreviation{mae}{MAE}{mean absolute error}
\newabbreviation{brk}{BRK}{Bulatov Reed Kumar}
\newabbreviation{gbed}{GBED}{grain boundary energy distribution}
\newabbreviation{gbcd}{GBCD}{grain boundary character distribution}
\newabbreviation{mfz}{MFZ}{misorientation fundamental zone}
\newabbreviation{bp}{BP}{boundary plane}
\newabbreviation{bpfz}{BPFZ}{boundary plane fundamental zone}
\newabbreviation{knn}{kNN}{k-nearest neighbor}
\newabbreviation{gbe}{GBE}{grain boundary energy}
\newabbreviation{gbo}{GBO}{grain boundary octonion}
\newabbreviation{nbo}{NBO}{no-boundary octonion}
\newabbreviation{oslerp}{oSLERP}{octonion Spherical Linear Interpolation}
\newabbreviation{loocv}{LOOCV}{leave-one-out cross validation}
\newabbreviation{kfcv}{kFCV}{k-fold cross validation}
\newabbreviation{seo}{SEO}{symmetrically equivalent octonion}
\newabbreviation{fex}{FEX}{file exchange}
\newabbreviation{idw}{IDW}{inverse-distance weighting}
\newabbreviation{fic}{FIC}{fully independent conditional}
\newabbreviation{svd}{SVD}{singular value decomposition}
\newabbreviation{pca}{PCA}{principal component analysis}
\newabbreviation{gbc}{GBC}{grain boundary character}
\newabbreviation{fz}{FZ}{fundamental zone}
% \newabbreviation{pfz}{pFZ}{pseudo fundamental zone} % pfz replaced by vfz
% \newabbreviation{cmo}{CMO}{closed-mesh octonion} % cmo replaced by vfzo
\newabbreviation{vfz}{VFZ}{Voronoi fundamental zone}
\newabbreviation{vfzgbo}{VFZ-GBO}{Voronoi fundamental zone grain boundary octonion}
\newabbreviation{lobpcg}{LOBPCG}{locally optimal block preconditioned conjugate gradient}
\newabbreviation{lkr}{LKR}{Laplacian kernel regression}
\newabbreviation{ms}{MS}{molecular statics}
\newabbreviation{sst}{SST}{standard stereographic triangle}
\newabbreviation{ml}{ML}{machine learning}
\newabbreviation{doe}{DoE}{design of experiments}
\newabbreviation{ct}{CT}{coherent-twin}
\newabbreviation{csl}{CSL}{coincident site lattice}
\newabbreviation{dft}{DFT}{density-functional theory}
\newabbreviation{hedm}{HEDM}{high-energy diffraction microscopy}
% example abbreviations
% \newabbreviation{seo}{SEO}{symmetrically equivalent octonions}
%\newabbreviation[longplural={grain boundaries}]{gb}{GB}{grain boundary}

%example usage: \gls{gpr}
%example usage: \Gls{gpr} (capitalize first letter, only meaningful for first usage)
% \glspl{seo} --> symmetrically equivalent octonions OR SEOs
%^^^^^^^^^^^^^^^^^^^^^^^^^^^^^^^^^^^^^^^^^^^^^^^^^^^


% Add "S" to figure captions, sections, and equations
\renewcommand{\thefigure}{S\arabic{figure}}
\renewcommand{\thesection}{S\arabic{section}}
\renewcommand{\theequation}{S\arabic{equation}}

\begin{document}
\sloppy %maybe deals with figure/text spacing. Should deal with text going off the page

\begin{frontmatter}

%\title{Grain Boundary Octonion Meshing and Interpolation}
\title{\mytitle{}: Supplementary Information}

\author[myu]{Sterling G. Baird\corref{cor1}}
\ead{ster.g.baird@gmail.com}
\author[myu]{Eric R. Homer}
\author[myu]{David T. Fullwood}
\author[myu]{Oliver K. Johnson}

\address[myu]{Department of Mechanical Engineering, Brigham Young University, Provo, UT 84602, USA}

\cortext[cor1]{Corresponding author.}

\date{February 2020}

\end{frontmatter}

\tableofcontents

\subsection{Use of Interpolation Function}
\label{sec:methods:repofn}
To facilitate easy application of the presented methods, a vectorized, parallelized, MATLAB implementation, \matlab{interp5DOF.m}, is made available in the \vfzorepo{} \cite{bairdFiveDegreeofFreedom5DOF2020} with similar input/output structure to that of built-in MATLAB interpolation functions (e.g. \matlab{scatteredInterpolant()}, \matlab{griddatan()}). A typical function call is as follows: \matlab{ypred = interp5DOF(qm,nA,y,qm2,nA2,method)}. The argument \matlab{y} is a vector of known property values corresponding to the GBs defined by (\matlab{qm},\matlab{nA}), which respectively denote pairs of GB misorientation quaternions and \gls{bp} normals. The result, \matlab{ypred}, is a vector of predicted/interpolated property values corresponding to the \outpt{} \glspl{gb} defined by (\matlab{qm2},\matlab{nA2}). % and can be compared with the true \gls{brk} values (\matlab{ytrue}) via e.g. \matlab{get\_errmetrics.m} and \matlab{parityplot.m}.

Internally, these are converted to \glspl{gbo} and interpolation is performed using the selected \matlab{method}. For the validation function, these can be compared to the true \glspl{gbe} \matlab{ytrue}. The methods used in this work are \matlab{'pbary'}, \matlab{'gpr'}, \matlab{'idw'}, and \matlab{'nn'}, corresponding to planar barycentric, \gls{gpr}, \gls{idw}, and \gls{nn} interpolation, respectively. A placeholder template with instructions for implementing additional interpolation schemes is also provided in \matlab{interp5DOF.m}. See \citet{francisGeodesicOctonionMetric2019} and \matlab{five2oct.m} \cite{bairdFiveDegreeofFreedom5DOF2020} treatments of conversions to \gls{gbo} coordinates in the passive and active senses, respectively (\cref{sec:app:convention}).

%Additionally, the \gls{gprm} model can be probed at new \glspl{gb} via \matlab{gprmix.m}, which provides both predicted \gls{gbe} and uncertainty standard deviation.


\section{Euclidean and Arc Length Distances}
\label{sec:supp:dist-parity}

In addition to enabling us to leverage the machinery of efficient and established algorithms, the choice of using Euclidean distance as opposed to hyperspherical arclength can be justified by the following observations:
\begin{itemize}
	\item The minimum Euclidean distance \gls{seo} will be the same as the minimum arc length distance \gls{seo} because $d_{\text{S}}$ is a monotonically increasing function of $d_{\text{E}}$, for $d_{\text{S}}\!\left(d_{\text{E}}\right)\in[0,\pi]$ (\cref{fig:dist-parity}). 
	\item For the FCC point group symmetry ($m\bar{3}m$) the portion of $\mathbb{S}^7$ subtended by the \gls{vfz} is sufficiently small that the approximation $d_{\text{E}} \simeq d_{\text{S}}$ holds to very high accuracy\footnote{This is true for a specific pair of \glspl{gbo} within a \gls{vfz}. When calculating the \emph{minimum} distance between \glspl{seo} of two points, there are additional considerations that must be attended to as discussed in detail in \cref{sec:methods:framework:vfz-dist}.} as shown in \cref{fig:dist-parity}. 
	\item Calculation of $d_{\text{E}}$ does not require the use of any inverse trigonometric functions and is about \SI{23}{\percent} faster than calculation of $d_{\text{S}}$ or $d_\Omega$.
\end{itemize}

The close correlation between Euclidean and arc length distances in the \gls{vfzgbo} sense is shown in \cref{fig:dist-parity} using pairwise distances of \num{10000} \glspl{vfzgbo}. This justifies our use of Euclidean distance as an approximation of hyperspherical arc length (and by extension, that a scaled Euclidean distance approximates a non-symmetrized \gls{gbo} distance, see \cref{eq:8Deuclidean_dist,eq:7sphere_arc_length,eq:omega} of the main paper). However, comparison with the original \gls{gbo} metric \cite{francisGeodesicOctonionMetric2019} gives overestimation for some boundaries. This is an inherent feature of the \gls{vfz} framework that can be addressed via use of the ensemble methods described in \cref{sec:methods:framework:vfz-dist} (see also \cref{fig:dist-ensemble-k1-2-10-20,fig:dist-ensemble-rmse-mae}).

\begin{figure}
\centering
\includegraphics[scale=1]{dist-parity.png}
\caption{Parity plot of 8D Cartesian hyperspherical arc length vs. 8D Cartesian Euclidean distance for pairwise distances in a ($m\bar{3}m$) symmetrized set of \num{10000} randomly sampled \glspl{vfzgbo}. The max arc length is approximately \SI{0.58}{\radian}, indicating a max \gls{gbo} distance of approximately \SI{1.16}{\radian} or \SI{66.5}{\degree} between any two points in the \gls{vfz}. The close correlation between arc length and Euclidean distance supports the validity of using Euclidean distance instead of arc length in the interpolation methods. This is \textit{separate} from the correlation between \gls{vfzgbo} Euclidean or arc length distances with the traditional \gls{gbo} distance \cite{chesserLearningGrainBoundary2020}.}
\label{fig:dist-parity}
\end{figure}

Additionally, the use of an isometry equivalence relationship in \citet{morawiecDistancesGrainInterfaces2019} in a non-\gls{vfz} sense results in identical distance results within numerical tolerance (\cref{fig:pd-fix}).

\begin{figure}
    \centering
    \includegraphics{figures/pd-fix.png}
    \caption{\Gls{gb} distances calculated with one \gls{gbo} fixed vs. the traditional calculations in \citet{chesserLearningGrainBoundary2020} show that the isometry equivalence discussed in \citet{morawiecDistancesGrainInterfaces2019} applies to \glspl{gbo}. The pairwise-distance matrix for the Olmsted \glspl{gb} supplied in \cite{chesserGBOctonionCode2019} was used. }
    \label{fig:pd-fix}
\end{figure}

\section{Computational Complexity of \glsentryshort{vfz} vs. \glsentryshort{gbo} Distances }
Let $o_1$ and $o_2$ denote two \glspl{gb} represented in \gls{gbo} coordinates. 
To perform a traditional symmetrized \gls{gbo} distance calculation according to \citet{francisGeodesicOctonionMetric2019}, we compare all \glspl{seo} of $o_1$ to all of the \glspl{seo} of $o_2$ and take the smallest distance. If $N_p$ is the number of proper rotations of the crystallographic point group, this single minimum distance calculation requires a total of $4N_p^4$ \glspl{seo} to be considered (Sections 4.3 and 4.5 of \citet{francisGeodesicOctonionMetric2019}). Thus, the total number of \gls{seo} computations will be $4N_p^4L^2$. However, it is possible to fix a single \gls{gb} in the \gls{gb} pair and still obtain accurate\footnote{Compared with the pairwise distance matrix of the 388 Olmsted \glspl{gb}, we obtained a \gls{rmse} of \SI{1.6566E-7}{\degree} for this computation which completed in \SI{133}{\s} using 6 cores (see \matlab{get_pd_fix.m})} due to isometry equivalence (see Section 7 of \cite{morawiecDistancesGrainInterfaces2019} and \cref{fig:pd-fix}).

In contrast, for a single distance calculation using the \gls{vfz} framework, $o_1$ and $o_2$ are first mapped into the \gls{vfz}, and then only a single distance calculation is required between them. Mapping $o_1$ into the \gls{vfz} requires comparison of $8N_p^2$ \glspl{seo}\footnote{This is 8 instead of 4 because the simplifying assumption that only two of the four double cover cases need to be considered \cite{francisGeodesicOctonionMetric2019} does not apply in the \gls{vfz} framework. This is confirmed by applying \matlab{uniquetol()} on a set of $4608$ \glspl{gbo} which has a final set size of $4608$, where $4608=8\times N_p^2$ and $N_p=24$ (see \matlab{osymset.m}).} of $o_1$ with a fixed reference \gls{gb} in the interior of the \gls{vfz}; and likewise for $o_2$. Consequently, a single distance calculation between $o_1$ and $o_2$ under the \gls{vfz} framework requires $O(N_p^2)$ \gls{seo} computations. If one desires to compute a pairwise distance matrix between $L$ \glspl{gb}, the total computational cost\footnote{See \cref{sec:results:efficiency:symruntime} for a detailed explanation of why this is \emph{not} $O(N_p^2L^2)$.} will be $O(N_p^2L)$, which represents a dramatic reduction compared to the traditional approach.

\section{Additional Interpolation Results}

\subsection{Smaller Set Sizes of Input GBs}
Interpolation results for \num{388} and \num{10000} \glspl{gb} are given in \cref{fig:brkparity388} and \cref{fig:brkparity10000}, respectively.

\begin{figure}[!ht]
    \centering
    \includegraphics[scale=1]{brkparity388.png}
    \caption{Hexagonally binned parity plots for \num{388} \inpt{} and \num{10000} \outpt{} \glspl{gbo} formed via pairs of a random cubochorically sampled quaternion and a spherically sampled random boundary plane normal. Interpolation via (a) \gls{gpr}, (b) \gls{idw}, (c) \gls{nn}, and (d) barycentric coordinates.  \gls{brk} \gls{gbe} function for \gls{fcc} Ni \cite{bulatovGrainBoundaryEnergy2014} was used as the test function.}
    \label{fig:brkparity388}
\end{figure}

\begin{figure}[!ht]
    \centering
    \includegraphics[scale=1]{brkparity10000.png}
    \caption{Hexagonally binned parity plots for \num{10000} \inpt{} and \num{10000} \outpt{} \glspl{gbo} formed via pairs of a random cubochorically sampled quaternion and a spherically sampled random boundary plane normal. Interpolation via (a) \gls{gpr}, (b) \gls{idw}, (c) \gls{nn}, and (d) barycentric coordinates.  \gls{brk} \gls{gbe} function for \gls{fcc} Ni \cite{bulatovGrainBoundaryEnergy2014} was used as the test function.}
    \label{fig:brkparity10000}
\end{figure}

\section{Ensemble Interpolation Results}
\label{sec:ensemble-interp}
Ensemble interpolation is a classic technique that can be used to enhance predictive performance of models. Here we describe our methods (\cref{sec:ensemble-interp:methods}), results (\cref{sec:ensemble-interp:results}), and the potential of integrating ensemble interpolation with a \gls{gprm} scheme (\cref{sec:ensemble-interp:egprm}).

\subsection{Methods}
\label{sec:ensemble-interp:methods}
\Gls{vfzgbo} ensemble\footnote{Ours is a "bagging"-esque ensemble scheme because the same interpolation method (\glsxtrshort{gpr}) is used but with different representations for the \inpt{} data. } interpolation occurs by:
\begin{enumerate}
    \item generating multiple reference \glspl{gbo} to define multiple \glspl{vfz}
    \item obtaining multiple \gls{vfzgbo} representations for a set of \glspl{gb} based on the various reference \glspl{gbo}
    \item performing an interpolation (e.g. \gls{gpr}) for each of the representations
    \item homogenizing the ensemble of models (e.g. by taking the mean or median of the various models)
\end{enumerate}

\subsection{Results}
\label{sec:ensemble-interp:results}

Use of an ensemble interpolation scheme decreases interpolation error for a \gls{gpr} model with \num{50000} \inpt{} and \num{10000} \outpt{} \glspl{vfzgbo}. By using an ensemble size of 10 (i.e. 10 \gls{gpr} models each with different reference \glspl{gbo} and therefore different \glspl{vfz}), \gls{rmse} and \gls{mae} decreased from \SIlist{0.0241;0.0160}{\J\per\square\m} to \SIlist{0.0187;0.0116}{\J\per\square\m}, respectively, using the median homogenization function (\cref{fig:ensemble-interp-rmse-mae}). 
\begin{figure}
    \centering
    \includegraphics[scale=1]{figures/ensemble-interp-rmse-mae.png}
    \caption{(a) \Gls{rmse} and (b) \gls{mae} vs. ensemble size for mean, median, minimum, and maximum homogenization functions. A \gls{gpr} model with \num{50000} \inpt{} and \num{10000} \outpt{} \glspl{vfzgbo} was used. }
    \label{fig:ensemble-interp-rmse-mae}
\end{figure}

\Cref{fig:ensemble-interp} shows the hexagonally binned parity plots for predictions made using the mean, median, minimum, and maximum predicted values over an ensemble of 10 \glspl{vfz}. Qualitatively, the ensemble mean and ensemble median parity plots look similar to those from the main text (\cref{fig:brkparity50000}), though the distributions of the ensemble scheme are somewhat tighter. The ensemble minimum produces better predictions of low \gls{gbe} than any of the other models, but underestimates high \gls{gbe} as expected. Naturally, the ensemble maximum overestimates in general. Diminishing returns manifest in \cref{fig:ensemble-interp-rmse-mae} for mean and median homogenizations. This is to be expected because the original \gls{gbo} distances \cite{francisGeodesicOctonionMetric2019} are well-approximated using an ensemble size of 10 (\cref{fig:dist-ensemble-k1-2-10-20}c and \cref{fig:dist-ensemble-rmse-mae}).
\begin{figure}[h!]
    \centering
    \includegraphics[scale=1]{figures/ensemble-interp.png}
    \caption{Hexagonally binned parity plots for (a) mean, (b) median, (c) minimum, and (d) maximum ensemble homogenization functions. A \gls{gpr} model with \num{50000} \inpt{} and \num{10000} \outpt{} \glspl{vfzgbo} was used. }
    \label{fig:ensemble-interp}
\end{figure}

\subsection{Possibility: Combining Ensemble with \glsentrytitlecase{gpr}{long} Mixture}
\label{sec:ensemble-interp:egprm}

A scheme which preferentially favors the ensemble minimum for low \gls{gbe} predictions and defaults to ensemble mean or median for all other \glspl{gbe} may produce even better results across the full range of \glspl{gbe}. For example, this could be accomplished by combining the ensemble scheme described here with the \gls{gpr} mixture model described in \cref{sec:supp:kim-interp:method}.

\section{Barycentric Interpolation}
\label{sec:supp:bary}

\subsection{High-Aspect Ratios}
\label{sec:supp:bary:artifact}
An artifact of the barycentric interpolation method which occurs due to the presence of high-aspect ratio facets is shown in \cref{fig:high-aspect-non-int}. As the dimensionality increases for a constant number of points and from our numerical tests, the rate of missed facet intersections increases. This artifact and our method for addressing it are discussed in \cref{sec:app:bary:int} of the main text.

\begin{figure*}
    \centering
    \includegraphics[scale=1]{figures/high-aspect-non-int.png}
    \caption{Illustration of two \outpt{} points (red) for which no intersecting facet is found due to being positioned within a high-aspect ratio facet. The inset shows that facets connected to the \gls{nn} do not contain the \outpt{} point. Many \glspl{nn} would need to be considered before an intersection is found. Additionally, it is expected that if found, the interpolation will suffer from higher error due to use of facet vertices far from the interpolation point. Proper intersections of \outpt{} points with the mesh are shown in blue.}
    \label{fig:high-aspect-non-int}
\end{figure*}

% \subsection{Use of Alternative Distance Metrics}


\section{Kim Interpolation}
\label{sec:supp:kim-interp}

A \gls{gpr} mixing model is developed to accommodate the non-uniformly distributed, noisy Fe simulation data \cite{kimPhasefieldModeling3D2014} and better predict low \gls{gbe}. Details of the method (\cref{sec:supp:kim-interp:method}) and an analysis of the input data quality (\cref{sec:supp:kim-interp:quality}) are given. The code implementation is given in \texttt{gprmix.m} and \texttt{gprmix\_test.m} of the \vfzorepo{} \cite{bairdFiveDegreeofFreedom5DOF2020}.

\subsection{Details of \glsentrytitlecase{gpr}{long} Mixture}
\label{sec:supp:kim-interp:method}
As shown in \cref{fig:kim-interp-teach}a, prediction using the standard approach of the main document (termed the $\epsilon_1$ model) overestimates low \glspl{gbe} for this dataset. By training the model on only \glspl{gb} with a \gls{gbe} less than \thrtwo{} (termed the $\epsilon_2$ model) and by using an exponential (\matlab{KernelFunction='exponential'}) rather than a squared exponential kernel, prediction of low \glspl{gbe} improves, but naturally underestimation occurs for higher \glspl{gbe} (\cref{fig:kim-interp-teach}b).

\begin{figure}
    \centering
    \includegraphics[scale=1]{kim-interp-teach.png}
    \caption{(a) Hexagonally binned parity plot of the standard \gls{gpr} model. (b) All prediction \glspl{gb} based on the model using only training \glspl{gb} with a \gls{gbe} less than \thrtwo{}. (c) Combined disjoint model as explained in the text. (d) Hexagonally binned parity plots of the final \gls{gpr} mixing model. Points in (c) are produced by splitting the prediction data into less than and greater than \thr{}. A sigmoid mixing function (\cref{fig:gprmix-sigmoid}) is then applied where the predicted \glspl{gbe} shown in (c) determines the mixing fraction ($f$) to produce a weighted average of models (a) and (b). A large Fe simulation database \cite{kimPhasefieldModeling3D2014} using \num{46883} training datapoints and \num{11721} validation datapoints in an 80\%/20\% split. The \gls{gpr} mixture model decreases error for low \gls{gbe} and changes overall \gls{rmse} and \gls{mae} from \SI{0.057932}{\J\per\square\meter} and \SI{0.039857}{\J\per\square\meter} in the original model (shown in (a)) to \SI{0.055035}{\J\per\square\meter} and \SI{0.039185}{\J\per\square\meter} (shown in (d)), respectively.}
    \label{fig:kim-interp-teach}
\end{figure}

A combined, disjoint model (\cref{fig:kim-interp-teach}c) is taken ($\epsilon_3$) by replacing $\epsilon_1$ \gls{gbe} predictions for \glspl{gb} with \gls{gbe} less than \thr{} with the corresponding $\epsilon_2$ predictions. Finally, a weighted average (\cref{eq:gprmix}) is taken according to:

\begin{equation}
    \epsilon_{mix} = f \epsilon_1+(f-1) \epsilon_2
    \label{eq:gprmix}
\end{equation}
where $\epsilon_1$ and $\epsilon_2$ represent the standard \gls{gpr} model and the \gls{gpr} model trained on the subset of \glspl{gb} with a \gls{gbe} less than \thrtwo{}, respectively, and $f$ is the sigmoid mixing fraction given by:

\begin{equation}
    f=\frac{1}{e^{-m \left(\epsilon_3-b\right)}+1}
        % f\left(\epsilon_3\right)=\frac{1}{e^{-m \left(\epsilon_3-b\right)}+1}
    \label{eq:sigmoid}
\end{equation}
and shown in \cref{fig:gprmix-sigmoid} with $m=30$ and $b=\sigthr{}$, as used in this work. Larger values of $m$ yield a steeper sigmoid function and larger values of $b$ shift the sigmoid function further to the right. Specific values for $m$ and $b$ were chosen by visual inspection and trial and error. This results in a \gls{gpr} mixing model which better predicts low \glspl{gbe} while retaining overall predictive accuracy (\cref{fig:kim-interp-teach}d).

\begin{figure}
    \centering
    \includegraphics[scale=1]{figures/gprmix-sigmoid.png}
    \caption{Sigmoid mixing function used in the \gls{gpr} mixing model with $m=30$ and $b=\sigthr{}$ (\cref{eq:sigmoid}).}
    \label{fig:gprmix-sigmoid}
\end{figure}

Uncertainty of the \gls{gpr} mixing model is similarly obtained by taking a weighted average of the uncertainties of each model according to:

\begin{equation}
    \sigma_{mix} = f \sigma_1+(f-1) \sigma_2
    \label{eq:gprmix-sigma}
\end{equation}
where $\sigma_1$ and $\sigma_2$ are the corresponding uncertainties of $\epsilon_1$ and $\epsilon_2$, respectively, and $f$ is given by \cref{eq:sigmoid}. 

\subsection{Input Data Quality}
\label{sec:supp:kim-interp:quality}
Of the $\sim$\num{60000}\footnote{The "no-boundary" \glspl{gb} (i.e. \glspl{gb} with close to \SI{0}{\joule\per\square\meter} \gls{gbe}) were removed before testing for degeneracy.} \glspl{gb} in \cite{kimPhasefieldModeling3D2014}, $\sim$\num{10000} \glspl{gb} were repeats that were identified by converting to \glspl{vfzgbo} and applying \vfzorepo{} function \texttt{avg\_repeats.m}. In \cite{kimPhasefieldModeling3D2014}, mechanically selected \glspl{gb} were those which involved sampling in equally spaced increments\footnote{In some cases, this was equally spaced increments of the argument of a trigonometric function.} for each \gls{5dof} parameter, and a few thousand intentionally selected \glspl{gb} (i.e. special \glspl{gb}) were also considered. Of mechanically and intentionally selected \glspl{gb}, \numlist{9170;112} are repeats, respectively, with a total of \num{2496} degenerate sets\footnote{A degenerate "set" is distinct from a \gls{vfzgbo} "set", the latter of which is often used in the main text.} (see \cref{fig:kim-interp-degeneracy-sets} for a degeneracy histogram). Thus, on average there is a degeneracy of approximately four per set of degenerate \glspl{gb}.

By comparing \gls{gbe} values of (unintentionally\footnote{To our knowledge, the presence of repeat \glspl{gb} were not mentioned in \cite{kimPhasefieldModeling3D2014} or \cite{kimIdentificationSchemeGrain2011}}) repeated \glspl{gb} in the Fe simulation dataset \cite{kimPhasefieldModeling3D2014}, we can estimate the intrinsic error of the \inpt{} data. For example, minimum and maximum deviations from the average value of a degenerate set are \SIlist{-0.2625;0.2625}{\joule\per\square\meter}, respectively, indicating that a repeated Fe \gls{gb} simulation from \cite{kimPhasefieldModeling3D2014} can vary by as much as \SI{0.525}{\joule\per\square\meter}, though rare. Additionally, \Gls{rmse} and \gls{mae} values can be obtained within each degenerate set by comparing against the set mean. Overall \gls{rmse} and \gls{mae} are then obtained by averaging and weighting by the number of \glspl{gb} in each degenerate set. Following this procedure, we obtain an average set-wise \gls{rmse} and \gls{mae} of \SIlist{0.06529;0.06190}{\joule\per\square\meter}, respectively, which is an approximate measure of the intrinsic error of the data. \cref{fig:kim-interp-degeneracy-results} shows histograms and parity plots of the intrinsic error. The overestimation of intrinsic error mentioned in the main text (\cref{sec:results:simulation}) could stem from bias as to what type of \glspl{gb} exhibit repeats based on the sampling scheme used in \cite{kimPhasefieldModeling3D2014} and/or that many of the degenerate sets contain a low number of repeats (\cref{fig:kim-interp-degeneracy-sets}).

Next, we see that by binning \glspl{gb} into degenerate sets, most degenerate sets have a degeneracy of fewer than 5 \cref{fig:kim-interp-degeneracy-sets}. We split the repeated data into sets with a degeneracy of fewer than 5 and greater than or equal to 5 and plot the errors (relative to the respective set mean) in both histogram form (\cref{fig:kim-interp-degeneracy-results}a and \cref{fig:kim-interp-degeneracy-results}c, respectively) and as hexagonally-binned parity plots (\cref{fig:kim-interp-degeneracy-results}b and \cref{fig:kim-interp-degeneracy-results}d, respectively). While heavily repeated \glspl{gb} tend to give similar results, occasionally repeated \glspl{gb} often have larger \gls{gbe} variability. This could have physical meaning: Certain types of (e.g. high-symmetry) \glspl{gb} tend to have less variation (i.e. fewer and/or more tightly distributed metastable states). However, it could also be an artifact of the simulation setup that produced this data (e.g. deterministic simulation output for certain types of \glspl{gb}).

\begin{figure}
    \centering
    \includegraphics[scale=1]{kim-interp-degeneracy-sets.png}
    \caption{Histogram of number of sets vs. number of degenerate \glspl{gb} per set for the Fe simulation dataset \cite{kimPhasefieldModeling3D2014}. Most sets have a degeneracy of fewer than 5.}
    \label{fig:kim-interp-degeneracy-sets}
\end{figure}

\begin{figure}
    \centering
    \includegraphics[scale=1]{kim-interp-degeneracy-results.png}
    \caption{Degenerate \glspl{gb} sets are split into those with a degeneracy of fewer than 5 and greater than or equal to 5 and plotted as ( (a) and (c), respectively) error histograms and ( (b) and (d), respectively) hexagonally-binned parity plots. Large degenerate sets tend to have very low error, whereas small degenerate sets tend to have higher error. In other words, \glspl{gb} that are more likely to be repeated many times based on the sampling scheme in \cite{kimPhasefieldModeling3D2014} tend to give similar results, whereas \glspl{gb} that are less likely to be repeated often have larger variability in the simulation output. We do not know if this has physical meaning or is an artifact of the simulation setup.}
    \label{fig:kim-interp-degeneracy-results}
\end{figure}

% \subsection{Uncertainty Quantification and the Posterior Distribution of \glsentrytitlecase{gprm}{long} }
% The use of \gls{gpr} in the \gls{gprm} model facilitates uncertainty quantification which we discuss in more detail. We find that use of \cref{eq:gprmix-sigma} on two models with differing amounts of data leads to a model with typically higher uncertainty for low \glspl{gbe} than high \glspl{gbe} (\cref{fig:kim-interp-posterior}a). By observing the \glspl{nn} relative to an arc $\overline{AB}$, we find that there is a large scatter of \glspl{gbe} for \glspl{gb} that are within a radius of $\sim$\SI{6}{\degree} (\cref{fig:kim-interp-posterior}b). By sampling from the posterior distribution, we find that 

% Uncertainty results, as well as posterior samples of the \gls{gprm} distribution of models are shown in \cref{fig:kim-interp-posterior}.

% \begin{figure}
%     \centering
%     \includegraphics[scale=1]{figures/kim-interp-posterior.png}
%     \caption{Interpolation results for a large Fe simulation database \cite{kimPhasefieldModeling3D2014} using \num{46883} \inpt{} \glspl{gb} and \num{11721} \outpt{} \glspl{gb} in an 80\%/20\% split and a \gls{gpr} mixture model to better approximate low \glspl{gbe}. (a) Parity plot colored by uncertainty standard deviation. (b) Prediction (black) and uncertainty standard deviation (grey band) of \gls{gpr} mixture model as a function of distance along a 1D arc ($\overline{AB}$) between two \glspl{vfzgbo} ($A$ and $B$). The first \inpt{} \gls{nn} is shown as a black, dashed line. The \inpt{} point \glspl{knn} ($k\in[1,2,3,4,5,6]$) relative to $\overline{AB}$ are colored and sized according to the distance to $\overline{AB}$ (\inpt{} points far from the line are small, dark red circles and \inpt{} points close to the line are large, dark blue circles). In other words, this shows the data in a small region of influence around $\overline{AB}$ that contributed to the model predictions where close data is emphasized (larger) than far-away data (smaller). (c)
%     Five models sampled from the posterior distribution of the \gls{gpr} mixture model. (d) Fe \gls{gpr} mixture model results and uncertainty standard deviation (grey band) overlaid with the Ni \gls{brk} model along the arc $\overline{AB}$. Coordinates for $A$ and $B$ are given in \cref{tab:tunnel-AB2} of the main paper and 300 equally spaced points are plotted for (b), (c), and (d).}
%     \label{fig:kim-interp-posterior}
% \end{figure}

\section{Olmsted Interpolation}

As illustrated in \cref{fig:olmsted-Ni-loocv}, \gls{loocv} interpolation results for \SI{0}{\kelvin} \gls{ms} low-noise Ni simulations using the \gls{gpr} method are similar to \gls{lkr} results reported in Figure 6a of \citet{chesserLearningGrainBoundary2020} (reproduced on the right of \cref{fig:olmsted-Ni-loocv} for convenience).

\begin{figure}
     \centering
     \begin{subfigure}[b]{0.5\textwidth}
         \centering
         \includegraphics[width=\textwidth]{figures/olmsted-Ni-loocv.png}
        %  \caption{}
         \label{fig:our-loocv}
     \hfill
     \end{subfigure}
          \begin{subfigure}[b]{0.4\textwidth}
         \centering
         \includegraphics[width=\textwidth]{figures/ChesserFigure.jpg}
        %  \caption{}
         \label{fig:chesser-loocv}
     \end{subfigure}
        \caption{(left) Hexagonally binned parity plot for Ni simulation \glsxtrfull{gbe} interpolation using \glsxtrshort{loocv}. (right) Parity plot for \glsxtrfull{loocv} interpolation results reproduced from Figure 6a of \citet{chesserLearningGrainBoundary2020} under CC-BY Creative Commons license. }
        \label{fig:olmsted-Ni-loocv}
\end{figure}

% \begin{figure}
%     \centering
%     \includegraphics{figures/olmsted-Ni-loocv.png}
%     \caption{Hexagonally binned parity plot for Ni simulation \glsxtrfull{gbe} interpolation using \glsxtrfull{loocv}. }
%     \label{fig:olmsted-Ni-loocv}
% \end{figure}

%The true, underlying \gls{brk} function is also shown (black line) and a shaded error band of uncertainty standard deviation. \glspl{knn} ($1<=k<=6$) \inpt{} points closest to $\overline{AB}$ are plotted and colored by distance to $\overline{AB}$.

\newpage
\clearpage %make sure figures don't go past glossaries, bibliography, etc.
\printglossaries
%need to manually clear cached files & logs in overleaf to get updated abbreviations to appear

\newpage
\bibliographystyle{elsarticle-num-names}
\bibliography{5dof-gb-energy.bib}

\end{document}